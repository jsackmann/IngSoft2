\begin{enumerate}
\def\labelenumi{\arabic{enumi})}
\setcounter{enumi}{1}
\item
  Como atleta quiero poder ver mi posición en el mapa en tiempo real
  mientras estoy en un seguimiento

  \begin{itemize}
  \itemsep1pt\parskip0pt\parsep0pt
  \item
    Business Value: 34
  \item
    Story Points: 13
  \item
    Criterio de Aceptación:

    \begin{itemize}
    \itemsep1pt\parskip0pt\parsep0pt
    \item
      El atleta puede ver su posición actualizada a intervalos regulares
      en la pantalla.
    \item
      Si se apaga la pantalla o bloquea el teléfono, al reanudar la
      aplicación la actualización de la posición se reanuda en forma
      automática.
    \item
      Si se pierde señal de geolocalización, se notifica al usuario.
    \item
      El atleta puede ver un \emph{timestamp} en cada lugar donde se
      actualizó su posición.
    \item
      El atleta puede desactivar geolocalización si lo desea.
    \end{itemize}
  \item
    Tareas
  \end{itemize}

  \begin{longtable}[c]{@{}ll@{}}
  \hline\noalign{\medskip}
  \begin{minipage}[b]{0.92\columnwidth}\raggedright
  Tarea
  \end{minipage} & \begin{minipage}[b]{0.08\columnwidth}\raggedright
  Tiempo
  \end{minipage}
  \\\noalign{\medskip}
  \hline\noalign{\medskip}
  \begin{minipage}[t]{0.92\columnwidth}\raggedright
  Diseñar las pantallas y vistas de la aplicación
  \end{minipage} & \begin{minipage}[t]{0.08\columnwidth}\raggedright
  1h
  \end{minipage}
  \\\noalign{\medskip}
  \begin{minipage}[t]{0.92\columnwidth}\raggedright
  Instalar y preparar el entorno de desarrollo
  \end{minipage} & \begin{minipage}[t]{0.08\columnwidth}\raggedright
  2h
  \end{minipage}
  \\\noalign{\medskip}
  \begin{minipage}[t]{0.92\columnwidth}\raggedright
  Armar proyecto git
  \end{minipage} & \begin{minipage}[t]{0.08\columnwidth}\raggedright
  30min
  \end{minipage}
  \\\noalign{\medskip}
  \begin{minipage}[t]{0.92\columnwidth}\raggedright
  Investigar el lenguaje Objective-C y la plataforma iOS
  \end{minipage} & \begin{minipage}[t]{0.08\columnwidth}\raggedright
  2h
  \end{minipage}
  \\\noalign{\medskip}
  \begin{minipage}[t]{0.92\columnwidth}\raggedright
  Investigar como obtener el tiempo actual del celular
  \end{minipage} & \begin{minipage}[t]{0.08\columnwidth}\raggedright
  1h
  \end{minipage}
  \\\noalign{\medskip}
  \begin{minipage}[t]{0.92\columnwidth}\raggedright
  Implementar la lógica para obtener la posición actual del teléfono
  \end{minipage} & \begin{minipage}[t]{0.08\columnwidth}\raggedright
  1h
  \end{minipage}
  \\\noalign{\medskip}
  \begin{minipage}[t]{0.92\columnwidth}\raggedright
  Agregar una vista con un mapa en la pantalla
  \end{minipage} & \begin{minipage}[t]{0.08\columnwidth}\raggedright
  1h
  \end{minipage}
  \\\noalign{\medskip}
  \begin{minipage}[t]{0.92\columnwidth}\raggedright
  Implementar lógica para centrar ese mapa en una posición indicada
  \end{minipage} & \begin{minipage}[t]{0.08\columnwidth}\raggedright
  1h
  \end{minipage}
  \\\noalign{\medskip}
  \begin{minipage}[t]{0.92\columnwidth}\raggedright
  Investigar qué pasa cuando se pasa la aplicación al background y
  cuando vuelve, implementar la lógica que mantenga actualizando la
  aplicación incluso en background.
  \end{minipage} & \begin{minipage}[t]{0.08\columnwidth}\raggedright
  2h
  \end{minipage}
  \\\noalign{\medskip}
  \begin{minipage}[t]{0.92\columnwidth}\raggedright
  Testear que el mapa se centre correctamente.
  \end{minipage} & \begin{minipage}[t]{0.08\columnwidth}\raggedright
  30min
  \end{minipage}
  \\\noalign{\medskip}
  \begin{minipage}[t]{0.92\columnwidth}\raggedright
  Testear que efectivamente la posición se actualice al desplazar el
  teléfono.
  \end{minipage} & \begin{minipage}[t]{0.08\columnwidth}\raggedright
  30min
  \end{minipage}
  \\\noalign{\medskip}
  \begin{minipage}[t]{0.92\columnwidth}\raggedright
  Implementar la lógica para dibujar un recorrido en el mapa dados los
  puntos y un timestamp de los mismos.
  \end{minipage} & \begin{minipage}[t]{0.08\columnwidth}\raggedright
  3h
  \end{minipage}
  \\\noalign{\medskip}
  \hline
  \end{longtable}
\item
  Como atleta quiero que la aplicación siga mi fase dentro del plan para
  que me avise si lo estoy siguiendo o tengo que modificar mi marcha.

  \begin{itemize}
  \item
    Business Value: 34
  \item
    Story Points: 8
  \item
    Criterio de Aceptación:

    \begin{itemize}
    \itemsep1pt\parskip0pt\parsep0pt
    \item
      El atleta debe poder saber que plan y que entrenamiento esta
      actualmente siguiendo.
    \item
      El atleta debe poder saber en que fase de que entrenamiento esta
      en este momento.
    \item
      El atleta debe saber a que velocidad esta corriendo actualmente.
    \item
      El atleta debe recibir una notificación de que debe aumentar la
      marcha si está yendo más lento que la fase.
    \item
      El atleta debe recibir una notificación de que debe disminuir la
      marcha si está yendo más despacio que la fase.
    \item
      El atleta debe recibir una notificación en intervalos regulares si
      se está manteniendo en una marcha aceptable.
    \item
      Cada alerta por marcha inválida se repetirá a intervalos regulares
      mientras persista la condición.
    \item
      En la interfaz gráfica debe aparecer la velocidad a la que el
      atleta debiera correr para llegar a la distancia implícita
      requerida (resultado de multiplicar la velocidad media de la fase
      por la duración de la misma).
    \end{itemize}
  \item
    Tareas
  \end{itemize}

  \begin{longtable}[c]{@{}ll@{}}
  \hline\noalign{\medskip}
  \begin{minipage}[b]{0.92\columnwidth}\raggedright
  Tarea
  \end{minipage} & \begin{minipage}[b]{0.08\columnwidth}\raggedright
  Tiempo
  \end{minipage}
  \\\noalign{\medskip}
  \hline\noalign{\medskip}
  \begin{minipage}[t]{0.92\columnwidth}\raggedright
  Implementar la lógica para calcular los datos de velocidad promedio
  \end{minipage} & \begin{minipage}[t]{0.08\columnwidth}\raggedright
  2h
  \end{minipage}
  \\\noalign{\medskip}
  \begin{minipage}[t]{0.92\columnwidth}\raggedright
  Investigar un algoritmo para lograr calcular la velocidad promedio a
  medida que llegan los datos.
  \end{minipage} & \begin{minipage}[t]{0.08\columnwidth}\raggedright
  2h
  \end{minipage}
  \\\noalign{\medskip}
  \begin{minipage}[t]{0.92\columnwidth}\raggedright
  Codificar la lógica para obtener la velocidad necesaria para llegar a
  la velocidad requerida.
  \end{minipage} & \begin{minipage}[t]{0.08\columnwidth}\raggedright
  2h
  \end{minipage}
  \\\noalign{\medskip}
  \begin{minipage}[t]{0.92\columnwidth}\raggedright
  Codificar la lógica para que si la velocidad no esta en el rango, se
  envié una alerta correspondiente a cuan ``grave'' es el nivel de
  alerta.
  \end{minipage} & \begin{minipage}[t]{0.08\columnwidth}\raggedright
  5h
  \end{minipage}
  \\\noalign{\medskip}
  \begin{minipage}[t]{0.92\columnwidth}\raggedright
  Investigar como reproducir una canción en cada formato estandar (mp3,
  wav, etc.)
  \end{minipage} & \begin{minipage}[t]{0.08\columnwidth}\raggedright
  2h
  \end{minipage}
  \\\noalign{\medskip}
  \begin{minipage}[t]{0.92\columnwidth}\raggedright
  Codificar la vista para que muestre la velocidad objetivo, duración y
  distancia recorrida en esta fase
  \end{minipage} & \begin{minipage}[t]{0.08\columnwidth}\raggedright
  5h
  \end{minipage}
  \\\noalign{\medskip}
  \begin{minipage}[t]{0.92\columnwidth}\raggedright
  Crear un mock de un entrenamiento de una fase para poder realizar las
  pruebas de seguimiento de velocidad y actualización de la vista
  \end{minipage} & \begin{minipage}[t]{0.08\columnwidth}\raggedright
  30min
  \end{minipage}
  \\\noalign{\medskip}
  \begin{minipage}[t]{0.92\columnwidth}\raggedright
  Crear un mock de batería y configurador para testear los tipos de
  notificaciones y la frecuencia de actualización
  \end{minipage} & \begin{minipage}[t]{0.08\columnwidth}\raggedright
  1h
  \end{minipage}
  \\\noalign{\medskip}
  \begin{minipage}[t]{0.92\columnwidth}\raggedright
  Codificar un sistema de administración de actualizaciones para proveer
  los datos de posición a los distintos módulos
  \end{minipage} & \begin{minipage}[t]{0.08\columnwidth}\raggedright
  2h
  \end{minipage}
  \\\noalign{\medskip}
  \begin{minipage}[t]{0.92\columnwidth}\raggedright
  Testear para los 3 tipos de condiciones de marcha válida e inválida
  \end{minipage} & \begin{minipage}[t]{0.08\columnwidth}\raggedright
  2h
  \end{minipage}
  \\\noalign{\medskip}
  \hline
  \end{longtable}
\item
  Como Secretaría de Deportes de la ciudad de Balvanera y San Cristóbal,
  quiero tener un documento detallado del diseño de la aplicación

  \begin{itemize}
  \itemsep1pt\parskip0pt\parsep0pt
  \item
    Business Value: 21
  \item
    Story Points: 13
  \item
    Criterio de Aceptación:

    \begin{itemize}
    \itemsep1pt\parskip0pt\parsep0pt
    \item
      El diagrama corresponderá al diseño orientado a objetos de la
      aplicación.
    \item
      Se incluirá un diagrama de clases mostrando la taxonomía de los
      objetos a implementar.
    \item
      Se incluirá un diagrama de secuencia para mostrar pasos
      algorítmicos de interés.
    \item
      El mismo utilizará la sintaxis de la materia Ingeniería del
      Software II de la carrera de Ciencias de la Computación de la UBA.
    \item
      Se incluirá un informe detallado de las decisiones de diseño
      tomadas.
    \end{itemize}
  \item
    Tareas:
  \end{itemize}

  \begin{longtable}[c]{@{}ll@{}}
  \hline\noalign{\medskip}
  \begin{minipage}[b]{0.92\columnwidth}\raggedright
  Tarea
  \end{minipage} & \begin{minipage}[b]{0.08\columnwidth}\raggedright
  Tiempo
  \end{minipage}
  \\\noalign{\medskip}
  \hline\noalign{\medskip}
  \begin{minipage}[t]{0.92\columnwidth}\raggedright
  Hacer el diagrama de secuencia del sistema de notificaciones en
  seguimiento
  \end{minipage} & \begin{minipage}[t]{0.08\columnwidth}\raggedright
  1h
  \end{minipage}
  \\\noalign{\medskip}
  \begin{minipage}[t]{0.92\columnwidth}\raggedright
  Realizar el informe dei diseño justificando las decisiones
  \end{minipage} & \begin{minipage}[t]{0.08\columnwidth}\raggedright
  6h
  \end{minipage}
  \\\noalign{\medskip}
  \begin{minipage}[t]{0.92\columnwidth}\raggedright
  Determinar cuales son y realizar los diagramas de secuencia
  pertinentes
  \end{minipage} & \begin{minipage}[t]{0.08\columnwidth}\raggedright
  8h
  \end{minipage}
  \\\noalign{\medskip}
  \begin{minipage}[t]{0.92\columnwidth}\raggedright
  Hacer diagrama de clases para el mecanismo de actualización de datos
  en un seguimiento
  \end{minipage} & \begin{minipage}[t]{0.08\columnwidth}\raggedright
  8h
  \end{minipage}
  \\\noalign{\medskip}
  \begin{minipage}[t]{0.92\columnwidth}\raggedright
  Hacer el diagrama de clases de los controladores y su relación con los
  objetos de negocio
  \end{minipage} & \begin{minipage}[t]{0.08\columnwidth}\raggedright
  8h
  \end{minipage}
  \\\noalign{\medskip}
  \begin{minipage}[t]{0.92\columnwidth}\raggedright
  Hacer el diagrama de clases para el mecanismo de creación de batería y
  su impacto
  \end{minipage} & \begin{minipage}[t]{0.08\columnwidth}\raggedright
  3h
  \end{minipage}
  \\\noalign{\medskip}
  \begin{minipage}[t]{0.92\columnwidth}\raggedright
  Hacer el diagrama para el mecanismo de craeción de planes básicos
  \end{minipage} & \begin{minipage}[t]{0.08\columnwidth}\raggedright
  2h
  \end{minipage}
  \\\noalign{\medskip}
  \begin{minipage}[t]{0.92\columnwidth}\raggedright
  Hacer diagrama de clases para el mecanismo de estadísticas de
  entrenamientos y sus almacenamientos.
  \end{minipage} & \begin{minipage}[t]{0.08\columnwidth}\raggedright
  3h
  \end{minipage}
  \\\noalign{\medskip}
  \hline
  \end{longtable}
\item
  Como corredor quiero poder inicializar el seguimiento de un
  entrenamiento de los que me dio la aplicación para poder empezar a
  correr bajo el plan obtenido.

  \begin{itemize}
  \itemsep1pt\parskip0pt\parsep0pt
  \item
    Business Value: 21
  \item
    Story Points: 13
  \item
    Criterio de Aceptación:

    \begin{itemize}
    \itemsep1pt\parskip0pt\parsep0pt
    \item
      El atleta puede elegir un plan de la lista de disponibles.
    \item
      El atleta puede seleccionar que el plan empiece a correr, y el
      mismo empezara el seguimiento en la primer fase.
    \item
      El atleta puede detener el entrenamiento en cualquier momento.
    \item
      El seguimiento termina de acuerdo a las fases del plan.
    \item
      El atleta puede elegir un entrenamiento si este esta disponible,
      es decir si el plazo para el objetivo no ha expirado.
    \end{itemize}
  \item
    Tareas
  \end{itemize}

  \begin{longtable}[c]{@{}ll@{}}
  \hline\noalign{\medskip}
  \begin{minipage}[b]{0.92\columnwidth}\raggedright
  Tarea
  \end{minipage} & \begin{minipage}[b]{0.08\columnwidth}\raggedright
  Tiempo
  \end{minipage}
  \\\noalign{\medskip}
  \hline\noalign{\medskip}
  \begin{minipage}[t]{0.92\columnwidth}\raggedright
  Implementar el controlador para que al apretar el boton de iniciar el
  seguimiento, inicie el seguimiento para el entrenamiento indicado con
  los datos del mismo
  \end{minipage} & \begin{minipage}[t]{0.08\columnwidth}\raggedright
  5h
  \end{minipage}
  \\\noalign{\medskip}
  \begin{minipage}[t]{0.92\columnwidth}\raggedright
  Implementar que se pueda detener un plan de entrenamiento en cualquier
  momento del mismo
  \end{minipage} & \begin{minipage}[t]{0.08\columnwidth}\raggedright
  3h
  \end{minipage}
  \\\noalign{\medskip}
  \begin{minipage}[t]{0.92\columnwidth}\raggedright
  Investigar maneras de ordenar la lista según relevancia de
  entrenamientos
  \end{minipage} & \begin{minipage}[t]{0.08\columnwidth}\raggedright
  2h
  \end{minipage}
  \\\noalign{\medskip}
  \begin{minipage}[t]{0.92\columnwidth}\raggedright
  Crear un mock de un okan ara testear a vsta de entrenamientos e
  indicar para empezar
  \end{minipage} & \begin{minipage}[t]{0.08\columnwidth}\raggedright
  3h
  \end{minipage}
  \\\noalign{\medskip}
  \begin{minipage}[t]{0.92\columnwidth}\raggedright
  Utilizar el mock de entrenamientos en el mock de planes para asociar
  un plan a entrenamientos de prueba
  \end{minipage} & \begin{minipage}[t]{0.08\columnwidth}\raggedright
  30min
  \end{minipage}
  \\\noalign{\medskip}
  \begin{minipage}[t]{0.92\columnwidth}\raggedright
  Implementar la vista para permitir iniciar un entrenamiento y mostrar
  sus fases y datos, con un botón para iniciar.
  \end{minipage} & \begin{minipage}[t]{0.08\columnwidth}\raggedright
  2h
  \end{minipage}
  \\\noalign{\medskip}
  \begin{minipage}[t]{0.92\columnwidth}\raggedright
  Implementar la vista de planes disponibles para el usuario
  \end{minipage} & \begin{minipage}[t]{0.08\columnwidth}\raggedright
  1h
  \end{minipage}
  \\\noalign{\medskip}
  \begin{minipage}[t]{0.92\columnwidth}\raggedright
  Testear que el seguimiento sea inicializado con los datos del
  entrenamiento seleccionado
  \end{minipage} & \begin{minipage}[t]{0.08\columnwidth}\raggedright
  1h
  \end{minipage}
  \\\noalign{\medskip}
  \hline
  \end{longtable}
\item
  Como Secretaría de Deportes de la ciudad de Balvanera y San Cristóbal,
  quiero tener un documento detallado del diseño de la aplicación

  \begin{itemize}
  \itemsep1pt\parskip0pt\parsep0pt
  \item
    Business Value: 21
  \item
    Story Points: 13
  \item
    Criterio de Aceptación:

    \begin{itemize}
    \itemsep1pt\parskip0pt\parsep0pt
    \item
      El diagrama corresponderá al diseño orientado a objetos de la
      aplicación.
    \item
      Se incluirá un diagrama de clases mostrando la taxonomía de los
      objetos a implementar.
    \item
      Se incluirá un diagrama de secuencia para mostrar pasos
      algorítmicos de interés.
    \item
      El mismo utilizará la sintaxis de la materia Ingeniería del
      Software II de la carrera de Ciencias de la Computación de la UBA.
    \item
      Se incluirá un informe detallado de las decisiones de diseño
      tomadas.
    \end{itemize}
  \item
    Tareas
  \end{itemize}

  \begin{longtable}[c]{@{}lr@{}}
  \hline\noalign{\medskip}
  \begin{minipage}[b]{0.89\columnwidth}\raggedright
  Tarea
  \end{minipage} & \begin{minipage}[b]{0.11\columnwidth}\raggedleft
  Tiempo
  \end{minipage}
  \\\noalign{\medskip}
  \hline\noalign{\medskip}
  \begin{minipage}[t]{0.89\columnwidth}\raggedright
  Conseguir dispositivo de prueba
  \end{minipage} & \begin{minipage}[t]{0.11\columnwidth}\raggedleft
  2h
  \end{minipage}
  \\\noalign{\medskip}
  \begin{minipage}[t]{0.89\columnwidth}\raggedright
  Preparar la presentación del producto
  \end{minipage} & \begin{minipage}[t]{0.11\columnwidth}\raggedleft
  2h
  \end{minipage}
  \\\noalign{\medskip}
  \begin{minipage}[t]{0.89\columnwidth}\raggedright
  Preparar informe de funcionalidades
  \end{minipage} & \begin{minipage}[t]{0.11\columnwidth}\raggedleft
  3h
  \end{minipage}
  \\\noalign{\medskip}
  \begin{minipage}[t]{0.89\columnwidth}\raggedright
  Generar y probar el instalable de la demo para el dispositivo
  \end{minipage} & \begin{minipage}[t]{0.11\columnwidth}\raggedleft
  30min
  \end{minipage}
  \\\noalign{\medskip}
  \begin{minipage}[t]{0.89\columnwidth}\raggedright
  Preparar informe de seguimiento de trabajo realizado
  \end{minipage} & \begin{minipage}[t]{0.11\columnwidth}\raggedleft
  2h
  \end{minipage}
  \\\noalign{\medskip}
  \begin{minipage}[t]{0.89\columnwidth}\raggedright
  Preparar la charla de presentación
  \end{minipage} & \begin{minipage}[t]{0.11\columnwidth}\raggedleft
  2h30min
  \end{minipage}
  \\\noalign{\medskip}
  \begin{minipage}[t]{0.89\columnwidth}\raggedright
  Preparar la restrospectiva sobre el proyecto
  \end{minipage} & \begin{minipage}[t]{0.11\columnwidth}\raggedleft
  1h
  \end{minipage}
  \\\noalign{\medskip}
  \hline
  \end{longtable}
\item
  Como corredor quiero poder ver la velocidad promedio y la duración de
  cada fase de un entrenamiento para saber el criterio con el que la
  aplicación mide mi performance corriendo.

  \begin{itemize}
  \itemsep1pt\parskip0pt\parsep0pt
  \item
    Business Value: 13
  \item
    Story Points: 8
  \item
    Criterio de Aceptación:

    \begin{itemize}
    \itemsep1pt\parskip0pt\parsep0pt
    \item
      El atleta podrá entrar a los planes que tiene creados.
    \item
      El atleta debe poder ver los entrenamientos que tiene listos
      dentro de ese plan.
    \item
      El atleta debe poder elegir un entrenamiento de los que la
      aplicación ha preparado.
    \item
      El atleta debe poder examinar las fases de un entrenamiento.
    \item
      El atleta debe poder elegir una fase para ver el detalle de la
      misma.
    \item
      El atleta debe ver para la fase elegida un rango de velocidades en
      km/h que son aceptables.
    \item
      El atleta debe poder ver para la fase elegida cuanto tiempo dura
      la misma en minutos.
    \end{itemize}
  \item
    Tareas
  \end{itemize}

  \begin{longtable}[c]{@{}ll@{}}
  \hline\noalign{\medskip}
  \begin{minipage}[b]{0.92\columnwidth}\raggedright
  Tarea
  \end{minipage} & \begin{minipage}[b]{0.08\columnwidth}\raggedright
  Tiempo
  \end{minipage}
  \\\noalign{\medskip}
  \hline\noalign{\medskip}
  \begin{minipage}[t]{0.92\columnwidth}\raggedright
  Implementar modelos Corredor, Plan, Fase y Entrenamiento
  \end{minipage} & \begin{minipage}[t]{0.08\columnwidth}\raggedright
  1h
  \end{minipage}
  \\\noalign{\medskip}
  \begin{minipage}[t]{0.92\columnwidth}\raggedright
  Mockear datos de los modelos
  \end{minipage} & \begin{minipage}[t]{0.08\columnwidth}\raggedright
  30min
  \end{minipage}
  \\\noalign{\medskip}
  \begin{minipage}[t]{0.92\columnwidth}\raggedright
  Crear una vista para mostrar los datos de las fases de entrenamiento
  que incluyan la duración y rango de velocidades.
  \end{minipage} & \begin{minipage}[t]{0.08\columnwidth}\raggedright
  2h
  \end{minipage}
  \\\noalign{\medskip}
  \begin{minipage}[t]{0.92\columnwidth}\raggedright
  Investigar como mostrar datos numéricos de velocidad y duración por la
  interfaz del celular.
  \end{minipage} & \begin{minipage}[t]{0.08\columnwidth}\raggedright
  30min
  \end{minipage}
  \\\noalign{\medskip}
  \begin{minipage}[t]{0.92\columnwidth}\raggedright
  Crear una lista que muestre y permita elegir los planes de
  entrenamiento
  \end{minipage} & \begin{minipage}[t]{0.08\columnwidth}\raggedright
  1h
  \end{minipage}
  \\\noalign{\medskip}
  \begin{minipage}[t]{0.92\columnwidth}\raggedright
  Crear una lista para que muestre y permita elegir los entrenamientos
  del usuario
  \end{minipage} & \begin{minipage}[t]{0.08\columnwidth}\raggedright
  1h
  \end{minipage}
  \\\noalign{\medskip}
  \begin{minipage}[t]{0.92\columnwidth}\raggedright
  Crear una lista que muestre las fases de un entrenamiento
  \end{minipage} & \begin{minipage}[t]{0.08\columnwidth}\raggedright
  1h
  \end{minipage}
  \\\noalign{\medskip}
  \begin{minipage}[t]{0.92\columnwidth}\raggedright
  Agregar funcionalidad para que al seleccionar un plan de entrenamiento
  de la lista, se muestren la lista de los entrenamientos relacionados
  al plan seleccionado.
  \end{minipage} & \begin{minipage}[t]{0.08\columnwidth}\raggedright
  30min
  \end{minipage}
  \\\noalign{\medskip}
  \begin{minipage}[t]{0.92\columnwidth}\raggedright
  Agregar funcionalidad para que al seleccionar un entrenamiento de la
  lista se muestren las fases asociadas a ese entrenamiento
  \end{minipage} & \begin{minipage}[t]{0.08\columnwidth}\raggedright
  30min
  \end{minipage}
  \\\noalign{\medskip}
  \hline
  \end{longtable}
\end{enumerate}
