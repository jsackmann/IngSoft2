\begin{enumerate}
\def\labelenumi{\arabic{enumi})}
\item
  Como atleta quiero~tener disponible la lista de entrenamientos que ya
  concluí

  \begin{itemize}
  \item
    Story Points:~8
  \item
    Value:~5
  \item
    Criterio de Aceptación:

    \begin{itemize}
    \itemsep1pt\parskip0pt\parsep0pt
    \item
      El atleta puede ver una lista donde tiene cada entrenamiento.
    \item
      Al elegir uno de ellos con su celular, se mostrará el detalle del
      mismo
    \end{itemize}
  \item
    Tareas:

    \begin{itemize}
    \itemsep1pt\parskip0pt\parsep0pt
    \item
      Concebir cómo almacenar los datos de un entrenamiento
    \item
      Investigar opciones para mostrar la información de cada
      entrenamiento.
    \item
      Crear una vista con una \emph{table view} para mostrar la lista de
      entrenamientos concluidos.
    \item
      Crear una vista para mostrar el detalle de entrenamiento.
    \item
      Agregar funcionalidad para que al seleccionar un entrenamiento de
      la lista de entrenamiento se redireccione a la vista de detalle.
    \end{itemize}
  \end{itemize}
\item
  Como atleta quiero poder ver estadísticas de los entrenamientos que ya
  hice

  \begin{itemize}
  \item
    Story Points: 13
  \item
    Value:~5
  \item
    Criterio de Aceptación:

    \begin{itemize}
    \item
      El atleta puede ver para una fase de un entrenamiento realizado el
      recorrido que hizo y verlo en un mapa similar al que tiene
      disponible cuando hace el recorrido en una fase.
    \item
      El atleta puede ver para cada fase de un entrenamiento,
      estadisticas sobre su velocidad y distancia.
    \item
      El atleta puede ver cuánta distancia recorrió para cada
      entrenamiento, cuanto tardo en cada uno, y cual fue su velocidad
      máxima.
    \item
      Todos estos datos estarán disponibles para el atleta para cada
      entrenamiento en el detalle de los mismos que obtiene al
      seleccionarlo en la lista.
    \end{itemize}
  \item
    Tareas:

    \begin{itemize}
    \item
      Investigar si hay una librería para obtener estadísticas en base a
      los datos ingresados y que cantidad de datos necesita.
    \item
      Investigar el almacenamiento necesario para los datos que se deben
      obtener para el cálculo de cada estadística.
    \item
      Programar la lógica para calcular estas estadísticas en base
    \item
      Testear que las estadísticas obtenidas sean correctas.
    \item
      Probar de intentar sacar estadísticas de datos absurdos.
    \item
      Investigar como mostrar una ventana con las estadísticas que
      aparezca al seleccionar el entrenamiento de la lista.
    \end{itemize}
  \end{itemize}
\item
  Como atleta~quiero que la aplicación siga mi fase dentro del plan para
  que me avise si lo estoy siguiendo o tengo que modificar mi marcha.

  \begin{itemize}
  \item
    Story Points: 8
  \item
    Value:~21
  \item
    Criterio de aceptación:

    \begin{itemize}
    \item
      El atleta debe poder saber en que fase de que entrenamiento esta.
    \item
      El atleta debe recibir una notificación de que debe aumentar la
      marcha si está yendo más lento que el plan.
    \item
      El atleta debe recibir una notificación de que debe disminuir la
      marcha si está yendo más despacio que el plan.
    \item
      El atleta debe recibir una notificación en intervalos regulares si
      se está manteniendo en una marcha aceptable.
    \item
      Cada alerta por marcha inválida se repetirá a intervalos regulares
      mientras persista la condición.
    \item
      En la interfaz gráfica debe aparecer un detalle de la razón de la
      marcha inválida (es decir, debe indicar si esta yendo más lento o
      rápido, y la diferencia entre la marcha ideal y la que lleva).
    \end{itemize}
  \item
    Tareas:

    \begin{itemize}
    \item
      Investigar cómo reproducir una canción en cada formato estándar
      (mp3, wav, etc) dado.
    \item
      Codificar la lógica para que según el tipo de alerta se eliga un
      sonido a reproducir (potencialmente leyendo de almacenamiento el
      mismo) y se lo reproduzca.
    \item
      Codificar la lógica para según el tipo de alerta se eliga una
      canción aleatoria de las disponibles, se la lea de disco y
      reproduzca.
    \item
      Codificar la lógica para que si la velocidad no esta en el rango,
      se envíe una alerta y se muestre en pantalla la diferencia y según
      ese rango, cuan ``grave'' es el nivel de alerta.
    \item
      Testear para los 3 tipos de condiciones de marcha válida e
      inválida.
    \end{itemize}
  \end{itemize}
\item
  Como atleta quiero poder publicar los recorridos de los entrenamientos
  en redes sociales y aplicaciones de geolocalización.

  \begin{itemize}
  \item
    Story Points: 10
  \item
    Value:~1
  \item
    Criterio de aceptación

    \begin{itemize}
    \item
      El atleta puede seleccionar en qué red social o aplicación
      publicar.
    \item
      El atleta puede seleccionar qué recorrido puede publicar.
    \item
      El atleta puede escribir un mensaje a agregar además de los datos
      de su entrenamiento.
    \item
      La publicación es visible por los demás miembros de la red social
      de acuerdo a las reglas de privacidad de la misma.
    \item
      Solo los datos explícitamente indicados por el usuario son
      publicados en la red social correspondiente.
    \item
      El atleta puede decidir si quiere que se muestre el recorrido que
      realizó, la velocidad con la que corrió, etc., para cada tipo de
      dato a compartir.
    \end{itemize}
  \item
    Tareas:

    \begin{itemize}
    \itemsep1pt\parskip0pt\parsep0pt
    \item
      Investigar como funcionan la APIs de Facebook
    \item
      Investigar como funcionan la APIs de Google+
    \item
      Investigar como funcionan la APIs de Foursquare
    \item
      Investigar como funcionan la APIs de Nike Run
    \item
      Investigar como funcionan la APIs de Hi5
    \item
      Investigar como funcionan la APIs de MySpace.
    \item
      Codificar las funciones de publicación que hacen uso de cada API
    \item
      Investigar opciones de privacidad para proteger los datos del
      teléfono.
    \item
      Codificar la funcionalidad de configuración de datos compartidos.
    \end{itemize}
  \end{itemize}
\item
  ~Como atleta~quiero poder ingresar los datos de mi estado físico.

  \begin{itemize}
  \item
    Story Points: 5
  \item
    Value:~8
  \item
    Criterio de aceptación

    \begin{itemize}
    \item
      El atleta podrá ingresar su peso en kilogramos.
    \item
      El atleta podrá ingresar su altura en cm.
    \item
      El atleta podrá especificar mayores detalles usando categorías
      basadas en si ya corrió una carrera o no, ya corrió un maratón o
      no, su mejor marca de distancia en una carrera y en un maratón.
    \item
      La aplicación guardará registro del valor actual de ambos datos.
    \end{itemize}
  \item
    Tareas:

    \begin{itemize}
    \item
      Investigar cómo crear un formulario y guardar los datos en la
      aplicación
    \item
      Investigar cómo validar los campos de acuerdo al tipo de datos
      pedidos.
    \item
      Pensar cómo representar esos datos del usuario y en donde
      almacenar los mismos.
    \item
      Testear ingresando datos inválidos / absurdos para verificar la
      consistencia de los mismos dentro de la aplicación.
    \item
      Documentar límites de las validaciones.
    \end{itemize}
  \end{itemize}
\item
  ~Como atleta~quiero poder ingresar mi frecuencia semanal con la que
  puedo entrenar.

  \begin{itemize}
  \item
    Story Points:~1
  \item
    Value:~5
  \item
    Criterio de aceptación

    \begin{itemize}
    \item
      El atleta debe ingresar la frecuencia semanal con la que puede
      entrenar.
    \item
      Los valores ingresados deben ser una cantidad de días entre 1 y 7.
    \item
      La aplicación guardará registro del valor ingresado.
    \end{itemize}
  \item
    Tareas:

    \begin{itemize}
    \item
      Investigar implementación de fechas / calendarios en iOS
    \item
      Testear con datos inválidos.
    \item
      Crear una vista para que el usuario seleccione la frecuencia
      semanal.
    \item
      Guardar la información ingresada por el usuario
    \end{itemize}
  \end{itemize}
\item
  ~Como atleta~quiero poder ingresar el objetivo de mi entrenamiento.

  \begin{itemize}
  \item
    Story Points: 8
  \item
    Value:~13
  \item
    Criterio de aceptación

    \begin{itemize}
    \item
      El atleta debe ingresar sus objetivos propuestos entre las
      opciones:

      \begin{itemize}
      \item
        Correr 5 km sin tiempo.
      \item
        Terminar un maratón olímpico.
      \item
        Correr 7 km en 35 minutos.
      \end{itemize}
    \end{itemize}

    y otras opciones y posibilidades decididas durante la implementación
  \item
    Tareas:

    \begin{itemize}
    \item
      Codificar la lista de tareas.
    \item
      Investigar forma de agregar elementos a la lista si se fuera a
      actualizar la aplicación.
    \item
      Testear agregar objetivos a los ya presentados.
    \item
      Investigar que otros objetivos posibles serían interesantes para
      un potencial usuario.
    \end{itemize}
  \end{itemize}
\item
  ~Como atleta~quiero poder ingresar el plazo estipulado para mi
  entrenamiento si así lo deseo.

  \begin{itemize}
  \item
    Story Points:~2
  \item
    Value:~5
  \item
    Criterio de aceptación

    \begin{itemize}
    \itemsep1pt\parskip0pt\parsep0pt
    \item
      El atleta puede elegir un plazo estipulado para la finalización de
      cada uno de sus objetivos.
    \end{itemize}
  \item
    Tareas:

    \begin{itemize}
    \itemsep1pt\parskip0pt\parsep0pt
    \item
      Investigar que tipo de granularidad y que tipo de duraciones se
      pueden soportar: intervalos válidos, etc.
    \item
      Investigar posibles interfaces de usuario para ingresar tiempos,
      por ejemplo utilizar un calendario para poner una fecha de
      finalización.
    \item
      Codificar validacion de plazos si se utilizan fechas.
    \end{itemize}
  \end{itemize}
\item
  Como atleta~quiero que las notificaciones de velocidad de la
  aplicación sean acordes al de batería seleccionado.

  \begin{itemize}
  \item
    Story Points: 8
  \item
    Value:~5
  \item
    Criterio de aceptación:

    \begin{itemize}
    \item
      Si el atleta eligió un consumo bajo, las notificaciones son
      pitidos y ocurren cada 1 minuto.
    \item
      Si el atleta eligió un consumo alto de batería, las notificaciones
      son temas musicales preelegidos por la app y ocurren cada 10
      segundos.
    \item
      Para otros niveles de batería se determinará también una
      frecuencia de notificaciones y calidad de las mismas en el momento
      de la implementación.
    \end{itemize}
  \item
    Tareas:

    \begin{itemize}
    \item
      Investigar cómo obtener la velocidad actual a la que se desplaza
      el teléfono.
    \item
      Aproximar el consumo de batería de las actualizaciones en función
      de la frecuencia de la misma.
    \item
      Investigar como controlar la frecuencia de muestreo de velocidad
      del dispositivo.
    \item
      Escribir el código que permita modificar la frecuencia de
      muestreo.
    \item
      Escribir el código que obtenga la velocidad promedio del usuario
      dentro del intervalo de muestreo.
    \end{itemize}
  \end{itemize}
\item
  Como atleta~quiero que las actualizaciones de posición sean acordes al
  nivel de batería seleccionado.

  \begin{itemize}
  \item
    Story Points: 8
  \item
    Value:~2
  \item
    Criterio de aceptación:

    \begin{itemize}
    \itemsep1pt\parskip0pt\parsep0pt
    \item
      La posición se actualiza cada 10 segundos si el nivel de consumo
      batería elegido es alto.
    \item
      La posición se actualiza cada minuto si el nivel de consumo de
      batería es bajo.
    \item
      Para los demás niveles de batería también se indicará una
      frecuencia de actualización de posición al momento de la
      implementación.
    \end{itemize}
  \item
    Tareas:

    \begin{itemize}
    \itemsep1pt\parskip0pt\parsep0pt
    \item
      Investigar si se puede, y en caso de que se pueda como,
      seleccionar el nivel de actualización del GPS.
    \item
      Investigar cuanto consume la actualización del GPS en función de
      la frecuencia de refresco establecida.
    \item
      Codificar la funcionalidad de ajuste de frecuencia de
      actualizaciones del GPS.
    \end{itemize}
  \end{itemize}
\item
  Como atleta quiero poder ajustar la opción de consumo de batería.

  \begin{itemize}
  \item
    Story Points: 3
  \item
    Value:~2
  \item
    Criterio de aceptación:

    \begin{itemize}
    \itemsep1pt\parskip0pt\parsep0pt
    \item
      El atleta puede seleccionar dentro de los niveles disponibles,
      como mínimo bajo, medio y alto.
    \item
      La aplicación debe poder correr más tiempo bajo un plan de consumo
      bajo que en uno alto.
    \item
      El atleta puede determinar que impacto tiene en las
      funcionalidades de la aplicación el cambio de consumo de batería.
    \end{itemize}
  \item
    Tareas:

    \begin{itemize}
    \itemsep1pt\parskip0pt\parsep0pt
    \item
      Investigar que niveles de batería permite el dispositivo.
    \item
      Programar una opción seleccionable para cada nivel de batería y
      que sea accesible por los demás módulos
    \item
      Documentar para el usuario cual es el impacto de cada nivel de
      batería en la funcionalidad de la aplicación y en la duración del
      teléfono (usando estimativos de ser necesario).
    \item
      Incluir esta documentación como ayuda dentro de la aplicación
    \end{itemize}
  \end{itemize}
\item
  ~Como atleta quiero que la aplicación me de un plan de entrenamiento
  en base a los datos.

  \begin{itemize}
  \item
    Story Points: 13
  \item
    Value:~21
  \item
    Criterio de aceptación

    \begin{itemize}
    \item
      Si el atleta estableció como objetivo que desea correr una
      maratón, el sistema creará un plan concentrado en larga duración y
      velocidad constante.
    \item
      Si el atleta estableció que desea correr una determinada cantidad
      de kilómetros en un cierto tiempo, el sistema creará un plan con
      entrenamientos de velocidad progresivamente más difíciles hasta
      alcanzar el objetivo.
    \item
      Si el atleta no estableció requerimientos ni de distancia ni de
      tiempo, el programa devolverá una serie de entrenamientos
      recreativos.
    \item
      Si el atleta se encuentra en buen estado físico, los
      entrenamientos constarán de fases con mayor exigencia.
    \item
      La duración y velocidad devueltas serán inversamente
      proporcionales al peso del corredor, de acuerdo a los criterios de
      salud vigentes.
    \item
      La frecuencia semanal del plan de entrenamiento se corresponderá
      con la ingresada por el corredor al momento de dar la
      especificación de sus objetivos.
    \end{itemize}
  \item
    Tareas:

    \begin{itemize}
    \item
      Determinar distancias, velocidades y duraciones para las fases
      acordes a un plan de entrenamiento, peso y objetivos, consultando
      a médicos y entrenadores físicos posiblemente.
    \item
      Determinar cómo leer del celular las opciones almacenadas por el
      corredor.
    \item
      Codificar las reglas de asignación de planes en base a los
      parámetros indicados en el criterio.
    \item
      Codificar la lógica para un asignador de fases a días de semana de
      acuerdo a la frecuencia ingresada.
    \end{itemize}
  \end{itemize}
\item
  Como corredor quiero poder ver la velocidad promedio y la duración de
  cada fase de un entrenamiento para saber el criterio con el que la
  aplicación mide mi performance corriendo.

  \begin{itemize}
  \item
    Story Points:~8
  \item
    Value:~13
  \item
    Criterio de aceptación

    \begin{itemize}
    \item
      El atleta debe poder elegir un entrenamiento de los que la
      aplicación ha preparado.
    \item
      El atleta debe poder examinar las fases de un entrenamiento.
    \item
      El atleta debe ver para cada fase un rango de velocidades en km/h
      que son aceptables.
    \item
      El atleta debe poder ver para cada fase, cuanto tiempo dura la
      misma en minutos.
    \end{itemize}
  \item
    Tareas:

    \begin{itemize}
    \item
      Investigar cómo mostrar datos numéricos de velocidad y duración
      por la interfaz del celular, y como actualizar la vista cuando
      estos cambian.
    \item
      Investigar un algoritmo para lograr calcular la velocidad promedio
      a medida que llegan los datos.
    \item
      Testear que el promedio calculado es correcto incluso considerando
      actualizaciones de velocidad y tiempo poco frecuentes (por ejemplo
      en un modo de batería bajo).
    \item
      Crear una vista para mostrar los datos
    \item
      Implementar la lógica para calcular los datos de velocidad
      promedio.
    \end{itemize}
  \end{itemize}
\item
  Como atleta quiero que la aplicación me avise de la próxima fase del
  plan si ya pasó el tiempo.

  \begin{itemize}
  \item
    Story Points: 8
  \item
    Value:~21
  \item
    Criterio de aceptación

    \begin{itemize}
    \item
      La aplicación genera una notificación auditiva cuando se termine
      el tiempo de la fase actual y se pase a otra.
    \item
      La aplicación genera una notificación cuando se terminó la última
      fase del entrenamiento.
    \item
      No se genera esa notificación particular por otro motivo.
    \end{itemize}
  \item
    Tareas:

    \begin{itemize}
    \item
      Investigar cómo generar alertas auditivas sencillas.
    \item
      Investigar cómo medir el paso del tiempo en el dispositivo (o al
      menos generar acciones a intervalos de tiempo regular)
    \end{itemize}
  \end{itemize}
\item
  Como atleta quiero poder ver mi posición en el mapa en tiempo real.

  \begin{itemize}
  \item
    Story Points: 13
  \item
    Value:~21
  \item
    Criterio de aceptación:

    \begin{itemize}
    \item
      El atleta puede ver su posición actualizada a intervalos regulares
      en la pantalla.
    \item
      Si se apaga la pantalla o bloquea el teléfono, al reanudar la
      aplicación la actualización de la posición se reanuda en forma
      automática.
    \item
      Si se pierde señal de geolocalización, se notifica al usuario.
    \item
      El atleta puede ver un \emph{timestamp} en cada lugar donde paso.
    \end{itemize}
  \item
    Tareas:

    \begin{itemize}
    \item
      Investigar como obtener el tiempo actual del celular.
    \item
      Implementar la lógica para obtener la posición actual del teléfono
    \item
      Incorporar el uso de mapas de otras fuentes en la aplicación.
    \item
      Agregar una vista con un mapa en la pantalla.
    \item
      Implementar lógica para centrar ese mapa en una posición indicada.
    \item
      Investigar qué pasa cuando se pasa la aplicación al background y
      cuando vuelve, implementar la lógica que mantenga actualizando la
      aplicación incluso en background.
    \item
      Testear que efectivamente la posición se actualice al desplazar el
      teléfono
    \item
      Testear que el mapa se centre correctamente.
    \item
      Implementar la lógica para dibujar un recorrido en el mapa dados
      los puntos y un \emph{timestamp} de los mismos.
    \end{itemize}
  \end{itemize}
\end{enumerate}
