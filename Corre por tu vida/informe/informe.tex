%BEGIN COPYPASTE EL INFORME DEL INFO
\documentclass[10pt, a4paper,english,spanish]{article}
\usepackage{subfig}

\parindent=20pt
\parskip=8pt
\usepackage[width=15.5cm, left=3cm, top=2.5cm, height= 24.5cm]{geometry}

\usepackage{ccfonts,eulervm} 
\usepackage[T1]{fontenc}
\usepackage{epigraph}
\usepackage{amsmath}
\usepackage{amsfonts}
\usepackage{amssymb}
\usepackage[activeacute, spanish,english]{babel}
\usepackage{cancel}
\usepackage[utf8]{inputenc}
\usepackage{algorithm}
%\usepackage{algpseudocode}
\usepackage{afterpage}
\usepackage{caratula}
\usepackage{url}
\usepackage{fancyhdr}
\usepackage{listings}
\usepackage{ulem}
\usepackage{dashrule}
\usepackage{pdflscape}
\usepackage{pgf}
\usepackage{tikz}
\usetikzlibrary{arrows,automata}


\floatname{algorithm}{Algoritmo}

\newtheorem{theorem}{Teorema}[section]
\newtheorem{lemma}[theorem]{Lema}
\newtheorem{proposition}[theorem]{Proposici\'on}
\newtheorem{corollary}[theorem]{Corolario}

\newcommand{\Var}{\textbf{var }}
\newcommand{\True}{\textbf{true }}
\newcommand{\False}{\textbf{false }}
\newcommand{\Break}{\textbf{break }}
\newcommand{\Continue}{\textbf{continue }}
\newcommand{\Param}{\textbf{param }}




\parindent 0em
%\algrenewcommand{\algorithmiccomment}[1]{//\textit{#1} }

\lstset{language=sql,numbers=left,tabsize=2,
	morekeywords={BEGIN,DECLARE,FOR,CREATE,PROCEDURE,RAISEERROR,EACH,ROW,BEFORE,AFTER,MINUS,IF},
	breaklines=true,breakatwhitespace=true}

\pagestyle{fancy}
\thispagestyle{fancy}
\addtolength{\headheight}{1pt}
\lhead{BD - TP1}
\rhead{Grupo 1}
\cfoot{\thepage}
\renewcommand{\footrulewidth}{0.4pt}
\newcommand{\hblacksquare}{\hfill \blacksquare}
%FIN COPYPASTE EL INFORME DEL INFO
\begin{document}

\materia{Ingenier\'ia del Software 2}
\submateria{Segundo Cuatrim\'estre de 2013}
\titulo{Trabajo Pr\'actico 1}
\subtitulo{User Stories - Tutor: Nicol\'as Rinaldi}
\grupo{Grupo 1}
\integrante{Juli\'an Sackmann}{540/09}{jsackmann@gmail.com}
\integrante{Juan Pablo Darago}{272/10}{jpdarago@gmail.com}
\integrante{Vanesa Stricker}{159/09}{vanesastricker@gmail.com}

\maketitle
\pagebreak

\tableofcontents
\pagebreak

\section{Sobre la herramienta}

Para el armado del Product y Sprint Backlogs utizamos la herramienta Agilefant,
accesible mediante la URL \url{http://agilefant.monits.com}. Se
dispone de un usuario para el tutor, el cual tiene los siguientes datos de acceso.

\begin{itemize}
	\item Nombre de usuario: nrinaldi
	\item Contrase\~na: monits
\end{itemize}

\section{Product backlog}

Incluimos a continuaci\'on las \textit{user stories} con su valor, \textit{story points} (que usamos
como medida de su dificultad), criterio de aceptaci\'on y tareas asociadas. Estas son las 
correspondientes al \textit{product backlog}.

\begin{enumerate}
\def\labelenumi{\arabic{enumi})}
\item
  Como atleta quiero que la aplicación siga mi fase dentro del plan para
  que me avise si lo estoy siguiendo o tengo que modificar mi marcha.

  \begin{itemize}
  \item
    Business Value: 34
  \item
    Story Points: 8
  \item
    Criterio de Aceptación:

    \begin{itemize}
    \itemsep1pt\parskip0pt\parsep0pt
    \item
      El atleta debe poder saber que plan y que entrenamiento esta
      actualmente siguiendo.
    \item
      El atleta debe poder saber en que fase de que entrenamiento esta
      en este momento.
    \item
      El atleta debe saber a que velocidad esta corriendo actualmente.
    \item
      El atleta debe recibir una notificación de que debe aumentar la
      marcha si está yendo más lento que la fase.
    \item
      El atleta debe recibir una notificación de que debe disminuir la
      marcha si está yendo más despacio que la fase.
    \item
      El atleta debe recibir una notificación en intervalos regulares si
      se está manteniendo en una marcha aceptable.
    \item
      Cada alerta por marcha inválida se repetirá a intervalos regulares
      mientras persista la condición.
    \item
      En la interfaz gráfica debe aparecer la velocidad a la que el
      atleta debiera correr para llegar a la distancia implícita
      requerida (resultado de multiplicar la velocidad media de la fase
      por la duración de la misma).
    \end{itemize}
  \end{itemize}
\item
  Como atleta quiero poder ver mi posición en el mapa en tiempo real
  mientras estoy en un seguimiento

  \begin{itemize}
  \itemsep1pt\parskip0pt\parsep0pt
  \item
    Business Value: 34
  \item
    Story Points: 13
  \item
    Criterio de Aceptación:

    \begin{itemize}
    \itemsep1pt\parskip0pt\parsep0pt
    \item
      El atleta puede ver su posición actualizada a intervalos regulares
      en la pantalla.
    \item
      Si se apaga la pantalla o bloquea el teléfono, al reanudar la
      aplicación la actualización de la posición se reanuda en forma
      automática.
    \item
      Si se pierde señal de geolocalización, se notifica al usuario.
    \item
      El atleta puede ver un \emph{timestamp} en cada lugar donde se
      actualizó su posición.
    \item
      El atleta puede desactivar geolocalización si lo desea.
    \end{itemize}
  \end{itemize}
\item
  Como atleta quiero que la aplicación me de un plan de entrenamiento en
  base a los datos ingresados.

  \begin{itemize}
  \itemsep1pt\parskip0pt\parsep0pt
  \item
    Business Value: 21
  \item
    Story Points: 13
  \item
    Criterio de Aceptación:

    \begin{itemize}
    \itemsep1pt\parskip0pt\parsep0pt
    \item
      Si el atleta estableció como objetivo que desea correr una
      maratón, el sistema creará un plan concentrado en larga duración y
      velocidad constante.
    \item
      Si el atleta estableció que desea correr una determinada cantidad
      de kilómetros en un cierto tiempo, el sistema creará un plan con
      entrenamientos de velocidad progresivamente más difíciles hasta
      alcanzar el objetivo.
    \item
      Si el atleta no estableció requerimientos ni de distancia ni de
      tiempo, el programa devolverá una serie de entrenamientos
      recreativos.
    \item
      Si el atleta se encuentra en buen estado físico, los
      entrenamientos constarán de fases con mayor exigencia.
    \item
      La duración y velocidad devueltas serán inversamente
      proporcionales al peso.
    \end{itemize}
  \end{itemize}
\item
  Como corredor quiero poder inicializar el seguimiento de un
  entrenamiento de los que me dio la aplicación para poder empezar a
  correr bajo el plan obtenido.

  \begin{itemize}
  \itemsep1pt\parskip0pt\parsep0pt
  \item
    Business Value: 21
  \item
    Story Points: 13
  \item
    Criterio de Aceptación:

    \begin{itemize}
    \itemsep1pt\parskip0pt\parsep0pt
    \item
      El atleta puede elegir un plan de la lista de disponibles.
    \item
      El atleta puede seleccionar que el plan empiece a correr, y el
      mismo empezara el seguimiento en la primer fase.
    \item
      El atleta puede detener el entrenamiento en cualquier momento.
    \item
      El seguimiento termina de acuerdo a las fases del plan.
    \item
      El atleta puede elegir un entrenamiento si este esta disponible,
      es decir si el plazo para el objetivo no ha expirado.
    \end{itemize}
  \end{itemize}
\item
  Como Secretaría de Deportes de la ciudad de Balvanera y San Cristóbal,
  quiero tener un documento detallado del diseño de la aplicación

  \begin{itemize}
  \itemsep1pt\parskip0pt\parsep0pt
  \item
    Business Value: 21
  \item
    Story Points: 13
  \item
    Criterio de Aceptación:

    \begin{itemize}
    \itemsep1pt\parskip0pt\parsep0pt
    \item
      El diagrama corresponderá al diseño orientado a objetos de la
      aplicación.
    \item
      Se incluirá un diagrama de clases mostrando la taxonomía de los
      objetos a implementar.
    \item
      Se incluirá un diagrama de secuencia para mostrar pasos
      algorítmicos de interés.
    \item
      El mismo utilizará la sintaxis de la materia Ingeniería del
      Software II de la carrera de Ciencias de la Computación de la UBA.
    \item
      Se incluirá un informe detallado de las decisiones de diseño
      tomadas.
    \end{itemize}
  \end{itemize}
\item
  Como atleta quiero poder ver estadísticas de los entrenamientos que ya
  hice

  \begin{itemize}
  \itemsep1pt\parskip0pt\parsep0pt
  \item
    Business Value: 21
  \item
    Story Points: 13
  \item
    Criterio de Aceptación:

    \begin{itemize}
    \itemsep1pt\parskip0pt\parsep0pt
    \item
      El atleta puede entrar a ``Mis estadísticas'' desde el menú
      principal.
    \item
      En ``Mis estadísticas'' tendrá una lista de los entrenamientos que
      ya realizó.
    \item
      Al elegir uno de estos entrenamientos de la lista se mostrará el
      detalle en otra pantalla.
    \item
      El atleta puede ver cuánta distancia recorrió para el
      entrenamiento elegido, cuanto tardo en cada uno, y cual fue su
      velocidad máxima.
    \item
      El atleta puede ver el recorrido que hizo en el entrenamiento en
      un mapa.
    \end{itemize}
  \end{itemize}
\item
  Como Secretaría de Deportes de la ciudad de Balvanera y San Cristóbal,
  quiero tener una demo de funcionalidad básica del producto presentada
  por los desarrolladores.

  \begin{itemize}
  \itemsep1pt\parskip0pt\parsep0pt
  \item
    Business Value: 21
  \item
    Story Points: 8
  \item
    Criterio de Aceptación:

    \begin{itemize}
    \itemsep1pt\parskip0pt\parsep0pt
    \item
      La demo debe permitir realizar el seguimiento de un entrenamiento,
      indicando para una fase si se debe aumentar o disminuir la
      velocidad.
    \item
      La demo debe incluir una presentación de funcionalidad y un
      informe de funcionalidades
    \item
      La demo debe mostrar el recorrido en un mapa.
    \item
      La demo debe correr en el entorno movil de iOS.
    \end{itemize}
  \end{itemize}
\item
  Como atleta quiero que la aplicación siga en que fase del
  entrenamiento me encuentro y me avise si esta cambia

  \begin{itemize}
  \itemsep1pt\parskip0pt\parsep0pt
  \item
    Business Value: 21
  \item
    Story Points: 8
  \item
    Criterio de Aceptación:

    \begin{itemize}
    \itemsep1pt\parskip0pt\parsep0pt
    \item
      En un seguimiento, se lleva cuenta de cual es la fase actual según
      la duración.
    \item
      Cuando el tiempo de la fase actual se acaba, el entrenamiento pasa
      a la siguiente fase o termina si no hay.
    \item
      Los datos de seguimiento se actualizan cuando se actualiza la
      fase.
    \item
      La aplicación genera una notificación auditiva cuando se termine
      el tiempo de la fase actual y se pase a otra.
    \item
      La aplicación genera una notificación cuando se terminó la última
      fase del entrenamiento.
    \item
      No se genera esa notificación particular por otro motivo.
    \end{itemize}
  \end{itemize}
\item
  Como atleta quiero poder ingresar el objetivo de mi entrenamiento.

  \begin{itemize}
  \itemsep1pt\parskip0pt\parsep0pt
  \item
    Business Value: 13
  \item
    Story Points: 8
  \item
    Criterio de Aceptación:

    \begin{itemize}
    \item
      El atleta al ingresar a ``Crear plan'' tendrá la opción
      ``objetivo''
    \item
      El atleta recibirá una lista de objetivos posibles.
    \item
      El atleta debe ingresar sus objetivos propuestos entre las
      opciones:
    \item
      \begin{itemize}
      \itemsep1pt\parskip0pt\parsep0pt
      \item
        Correr 5 km sin tiempo.
      \end{itemize}
    \item
      \begin{itemize}
      \itemsep1pt\parskip0pt\parsep0pt
      \item
        Terminar un maratón olímpico.
      \end{itemize}
    \item
      \begin{itemize}
      \itemsep1pt\parskip0pt\parsep0pt
      \item
        Correr 7 km en 35 minutos. y otras opciones y posibilidades
        decididas durante la implementación
      \end{itemize}
    \item
      Los objetivos corresponderán a una distancia a recorrer, y
      potencialmente un tiempo en el que se debe correr esta distancia.
    \end{itemize}
  \end{itemize}
\item
  Como corredor quiero tener estadisticas generales de mi performance
  como atleta

  \begin{itemize}
  \itemsep1pt\parskip0pt\parsep0pt
  \item
    Business Value: 13
  \item
    Story Points: 8
  \item
    Criterio de Aceptación:

    \begin{itemize}
    \itemsep1pt\parskip0pt\parsep0pt
    \item
      El atleta podrá entrar a ``Mis estadísticas'' dentro del menu
      principal.
    \item
      El atleta podrá entrar a ``Estadísticas generales'' dentro de la
      vista de estadísticas.
    \item
      Las estadísticas generales se calcularán en base a todos los
      entrenamientos del usuario.
    \item
      Estas estadísticas estarán presentadas cuando el atleta presione
      ``Estadísticas generales'' en una nueva vista.
    \item
      Cada estadística sera presentada con una descripción del valor
      obtenido.
    \item
      Estas estadísticas contendran como mínimo: Kilometros totales
      recorridos y velocidad máxima histórica.
    \end{itemize}
  \end{itemize}
\item
  Como corredor quiero poder ver la velocidad promedio y la duración de
  cada fase de un entrenamiento para saber el criterio con el que la
  aplicación mide mi performance corriendo.

  \begin{itemize}
  \itemsep1pt\parskip0pt\parsep0pt
  \item
    Business Value: 13
  \item
    Story Points: 8
  \item
    Criterio de Aceptación:

    \begin{itemize}
    \itemsep1pt\parskip0pt\parsep0pt
    \item
      El atleta podrá entrar a los planes que tiene creados.
    \item
      El atleta debe poder ver los entrenamientos que tiene listos
      dentro de ese plan.
    \item
      El atleta debe poder elegir un entrenamiento de los que la
      aplicación ha preparado.
    \item
      El atleta debe poder examinar las fases de un entrenamiento.
    \item
      El atleta debe poder elegir una fase para ver el detalle de la
      misma.
    \item
      El atleta debe ver para la fase elegida un rango de velocidades en
      km/h que son aceptables.
    \item
      El atleta debe poder ver para la fase elegida cuanto tiempo dura
      la misma en minutos.
    \end{itemize}
  \end{itemize}
\item
  Como atleta quiero poder ingresar los datos de mi estado físico.

  \begin{itemize}
  \itemsep1pt\parskip0pt\parsep0pt
  \item
    Business Value: 8
  \item
    Story Points: 5
  \item
    Criterio de Aceptación:

    \begin{itemize}
    \itemsep1pt\parskip0pt\parsep0pt
    \item
      El atleta podrá ingresar a ``Mi perfil'' desde el menú principal
    \item
      El atleta podrá ingresar su peso en kilogramos.
    \item
      El atleta podrá ingresar su altura en cm.
    \item
      El atleta podrá especificar mayores detalles usando categorías
      basadas en si ya corrió una carrera o no, ya corrió un maratón o
      no, su mejor marca de distancia en una carrera y en un maratón.
    \item
      La aplicación guardará registro del valor actual de ambos datos.
    \end{itemize}
  \end{itemize}
\item
  Como atleta quiero poder ajustar el consumo de batería

  \begin{itemize}
  \itemsep1pt\parskip0pt\parsep0pt
  \item
    Business Value: 8
  \item
    Story Points: 3
  \item
    Criterio de Aceptación:

    \begin{itemize}
    \itemsep1pt\parskip0pt\parsep0pt
    \item
      El atleta puede entrar a ``Configuración'' desde el menu
      principal.
    \item
      El atleta dispondrá en esta ventana de los distintos tipo de
      batería disponibles.
    \item
      El atleta puede seleccionar dentro de los niveles disponibles,
      como mínimo bajo, medio y alto.
    \item
      La aplicación debe poder correr más tiempo bajo un plan de consumo
      bajo que en uno alto.
    \item
      El atleta puede determinar que impacto tiene en las
      funcionalidades de la aplicación el cambio de consumo de batería.
    \end{itemize}
  \end{itemize}
\item
  Como atleta quiero poder ingresar el plazo estipulado para mi
  entrenamiento si así lo deseo.

  \begin{itemize}
  \itemsep1pt\parskip0pt\parsep0pt
  \item
    Business Value: 5
  \item
    Story Points: 2
  \item
    Criterio de Aceptación:

    \begin{itemize}
    \itemsep1pt\parskip0pt\parsep0pt
    \item
      El atleta puede estipular ``Plazo'' dentro de la selección de
      ``Crear Plan''
    \item
      El atleta puede elegir un plazo estipulado para la finalización de
      cada uno de sus objetivos.
    \item
      Los planes consistirán de una fecha límite en la que se quiere
      lograr el objetivo.
    \end{itemize}
  \end{itemize}
\item
  Como atleta quiero poder ingresar mi frecuencia semanal con la que
  puedo entrenar.

  \begin{itemize}
  \itemsep1pt\parskip0pt\parsep0pt
  \item
    Business Value: 5
  \item
    Story Points: 1
  \item
    Criterio de Aceptación:

    \begin{itemize}
    \itemsep1pt\parskip0pt\parsep0pt
    \item
      El atleta podrá ingresar ``Frecuencia semanal'' dentro de sus
      datos de ``Mi Perfil''
    \item
      El atleta debe ingresar la frecuencia semanal con la que puede
      entrenar.
    \item
      Los valores ingresados deben ser una cantidad de días entre 1 y 7.
    \item
      La aplicación guardará registro del valor.
    \end{itemize}
  \end{itemize}
\item
  Como atleta quiero que las actualizaciones de posición sean acordes al
  nivel de batería seleccionado.

  \begin{itemize}
  \itemsep1pt\parskip0pt\parsep0pt
  \item
    Business Value: 5
  \item
    Story Points: 8
  \item
    Criterio de Aceptación:

    \begin{itemize}
    \itemsep1pt\parskip0pt\parsep0pt
    \item
      La posición se actualiza cada 10 segundos si el nivel de consumo
      batería elegido es alto.
    \item
      La posición se actualiza cada minuto si el nivel de consumo de
      batería es bajo.
    \item
      Para los demás niveles de batería también se indicará una
      frecuencia de actualización de posición al momento de la
      implementación.
    \end{itemize}
  \end{itemize}
\item
  Como atleta quiero que las notificaciones de velocidad de la
  aplicación sean acordes al de batería seleccionado.

  \begin{itemize}
  \itemsep1pt\parskip0pt\parsep0pt
  \item
    Business Value: 5
  \item
    Story Points: 8
  \item
    Criterio de Aceptación:

    \begin{itemize}
    \itemsep1pt\parskip0pt\parsep0pt
    \item
      Si el atleta eligió un consumo bajo, las notificaciones son
      pitidos y ocurren cada 1 minuto.
    \item
      Si el atleta eligió un consumo alto de batería, las notificaciones
      son temas musicales preelegidos por la app y ocurren cada 10
      segundos.
    \item
      Para otros niveles de batería se determinará también una
      frecuencia de notificaciones y calidad de las mismas en el momento
      de la implementación.
    \end{itemize}
  \end{itemize}
\item
  Como atleta quiero poder publicar los recorridos de los entrenamientos
  en redes sociales y aplicaciones de geolocalización.

  \begin{itemize}
  \itemsep1pt\parskip0pt\parsep0pt
  \item
    Business Value: 1
  \item
    Story Points: 10
  \item
    Criterio de Aceptación:

    \begin{itemize}
    \itemsep1pt\parskip0pt\parsep0pt
    \item
      El atleta puede seleccionar en qué red social o aplicación
      publicar.
    \item
      El atleta puede seleccionar qué recorrido puede publicar.
    \item
      El atleta puede escribir un mensaje a agregar además de los datos
      de su entrenamiento.
    \item
      La publicación es visible por los demás miembros de la red social
      de acuerdo a las reglas de privacidad de la misma.
    \item
      Solo los datos explícitamente indicados por el usuario son
      publicados en la red social correspondiente.
    \item
      El atleta puede decidir si quiere que se muestre el recorrido que
      realizó, la velocidad con la que corrió, etc., para cada tipo de
      dato a compartir.
    \end{itemize}
  \end{itemize}
\end{enumerate}


\section{Sprint backlog}

A continuaci\'on detallamos las \textit{user stories} que incluimos en el actual \textit{sprint}. Las
mismas son:

\begin{itemize}
	\item 3) Como atleta quiero que la aplicación siga mi fase dentro del plan para que me avise si lo estoy siguiendo o tengo que modificar mi marcha.
	\item 14) Como corredor quiero poder ver la velocidad promedio y la duración de cada fase de un entrenamiento para saber el criterio con el que la aplicación mide mi performance corriendo.
	\item 16) Como atleta quiero poder ver mi posición en el mapa en tiempo real. 
\end{itemize}

y a continuaci\'on incluimos el detalles de las tareas involucradas y la dificultad asociada a cada una.

\begin{itemize}
	\itemsep 0em
	\item Investigar cómo reproducir una canción en cada formato estándar (mp3,
	wav, etc).
	\item[] \hfill \textit{Dificultad: 3}
	\item Investigar cómo guardar canciones en el teléfono.
	\item[] \hfill \textit{Dificultad: 3}
	\item Codificar la lógica para que si la velocidad no esta en el rango, se
	envíe una alerta.
	\item[] \hfill \textit{Dificultad: 5}
	\item Testear para los 3 tipos de condiciones de marcha válida e inválida.
	\item[] \hfill \textit{Dificultad: 2}
	\item Investigar cómo mostrar datos numéricos de velocidad y duración por la interfaz del celular, y como actualizar
	la vista cuando estos cambian.
	\item[] \hfill \textit{Dificultad: 3}
	\item Investigar un algoritmo para lograr calcular la velocidad promedio a medida que llegan los datos.
	\item[] \hfill \textit{Dificultad: 2}
	\item Testear que el promedio calculado es correcto incluso considerando actualizaciones de velocidad y tiempo poco frecuentes 
		(por ejemplo en un modo de batería bajo).
	\item[] \hfill \textit{Dificultad: 5}
	\item Crear una vista para mostrar los datos.
	\item[] \hfill \textit{Dificultad: 3}
	\item Implementar la lógica para calcular los datos de posición actual
	\item[] \hfill \textit{Dificultad: 5}
	\item Implementar la lógica para obtener la posición actual del teléfono.
	\item[] \hfill \textit{Dificultad: 8}
	\item Incorporar el uso de mapas de otras fuentes dentro de la aplicación
	\item[] \hfill \textit{Dificultad: 3}
	\item Agregar una vista con un mapa en la pantalla.
	\item[] \hfill \textit{Dificultad: 5}
	\item Implementar lógica para centrar ese mapa en una posición indicada.
	\item[] \hfill \textit{Dificultad: 3}
	\item Investigar qué pasa cuando se pasa la aplicación al \textit{background} y cuando vuelve, 
			implementar la lógica que mantenga actualizando a la aplicación incluso en background.
	\item[] \hfill \textit{Dificultad: 3}
	\item Testear que efectivamente la posición se actualice al desplazar el teléfono.
	\item[] \hfill \textit{Dificultad: 2}
	\item Testear que el mapa se centre correctamente.
	\item[] \hfill \textit{Dificultad: 2}
	\item Implementar la lógica para dibujar un recorrido en el mapa dados los puntos y un \textit{timestamp} para cada uno.
	\item[] \hfill \textit{Dificultad: 5}
\end{itemize}

\end{document}
