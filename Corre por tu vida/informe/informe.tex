%BEGIN COPYPASTE EL INFORME DEL INFO
\documentclass[10pt, a4paper,english,spanish]{article}
\usepackage{subfig}

\parindent=20pt
\parskip=8pt
\usepackage[width=15.5cm, left=3cm, top=2.5cm, height= 24.5cm]{geometry}

% \usepackage{ccfonts,eulervm} 
% \usepackage[T1]{fontenc}

\usepackage{longtable}
\usepackage{ccfonts,eulervm} 
\usepackage[T1]{fontenc}
% \usepackage{ccfonts,eulervm} 
% \usepackage[T1]{fontenc}
\usepackage{longtable}
\usepackage{epigraph}
\usepackage{amsmath}
\usepackage{amsfonts}
\usepackage{amssymb}
\usepackage[activeacute, spanish,english]{babel}
\usepackage{cancel}
\usepackage[utf8]{inputenc}
\usepackage{algorithm}
%\usepackage{algpseudocode}
\usepackage{afterpage}
\usepackage{caratula}
\usepackage{url}
\usepackage{fancyhdr}
\usepackage{listings}
\usepackage{ulem}
\usepackage{dashrule}
\usepackage{pdflscape}
\usepackage{pgf}
\usepackage{tikz}
\usetikzlibrary{arrows,automata}


\floatname{algorithm}{Algoritmo}

\newtheorem{theorem}{Teorema}[section]
\newtheorem{lemma}[theorem]{Lema}
\newtheorem{proposition}[theorem]{Proposici\'on}
\newtheorem{corollary}[theorem]{Corolario}

\newcommand{\Var}{\textbf{var }}
\newcommand{\True}{\textbf{true }}
\newcommand{\False}{\textbf{false }}
\newcommand{\Break}{\textbf{break }}
\newcommand{\Continue}{\textbf{continue }}
\newcommand{\Param}{\textbf{param }}




% \parindent 0em
%\algrenewcommand{\algorithmiccomment}[1]{//\textit{#1} }

\renewcommand{\emph}[1]{\textit{#1}}
\pagestyle{fancy}
\thispagestyle{fancy}
\addtolength{\headheight}{1pt}
\lhead{IS2 - TP1}
\rhead{Grupo 8}
\cfoot{\thepage}
\renewcommand{\footrulewidth}{0.4pt}
\newcommand{\hblacksquare}{\hfill \blacksquare}
%FIN COPYPASTE EL INFORME DEL INFO
\begin{document}

\materia{Ingenier\'ia del Software 2}
\submateria{Segundo Cuatrim\'estre de 2013}
\titulo{Trabajo Pr\'actico 1}
\subtitulo{User Stories - Tutor: Nicol\'as Rinaldi}
\grupo{Grupo 8}
\integrante{Juli\'an Sackmann}{540/09}{jsackmann@gmail.com}
\integrante{Juan Pablo Darago}{272/10}{jpdarago@gmail.com}
\integrante{Vanesa Stricker}{159/09}{vanesastricker@gmail.com}
\integrante{Mat\'ias Barbeito}{179/08}{matiasbarbeito@gmail.com}

\maketitle
\pagebreak

\tableofcontents
\pagebreak

\section{User stories}

\subsection{Sobre la herramienta}

Para el armado del Product y Sprint Backlogs utizamos la herramienta Agilefant,
accesible mediante la URL \url{http://agilefant.monits.com}. Se
dispone de un usuario para el tutor, el cual tiene los siguientes datos de acceso.

\begin{itemize}
	\item Nombre de usuario: nrinaldi
	\item Contrase\~na: monits
\end{itemize}

\subsection{Product backlog}

Incluimos a continuaci\'on las \textit{user stories} con su valor, \textit{story points} (que usamos
como medida de su dificultad), criterio de aceptaci\'on y tareas asociadas. Estas son las 
correspondientes al \textit{product backlog}.

\begin{enumerate}
\def\labelenumi{\arabic{enumi})}
\item
  Como atleta quiero que la aplicación siga mi fase dentro del plan para
  que me avise si lo estoy siguiendo o tengo que modificar mi marcha.

  \begin{itemize}
  \item
    Business Value: 34
  \item
    Story Points: 8
  \item
    Criterio de Aceptación:

    \begin{itemize}
    \itemsep1pt\parskip0pt\parsep0pt
    \item
      El atleta debe poder saber que plan y que entrenamiento esta
      actualmente siguiendo.
    \item
      El atleta debe poder saber en que fase de que entrenamiento esta
      en este momento.
    \item
      El atleta debe saber a que velocidad esta corriendo actualmente.
    \item
      El atleta debe recibir una notificación de que debe aumentar la
      marcha si está yendo más lento que la fase.
    \item
      El atleta debe recibir una notificación de que debe disminuir la
      marcha si está yendo más despacio que la fase.
    \item
      El atleta debe recibir una notificación en intervalos regulares si
      se está manteniendo en una marcha aceptable.
    \item
      Cada alerta por marcha inválida se repetirá a intervalos regulares
      mientras persista la condición.
    \item
      En la interfaz gráfica debe aparecer la velocidad a la que el
      atleta debiera correr para llegar a la distancia implícita
      requerida (resultado de multiplicar la velocidad media de la fase
      por la duración de la misma).
    \end{itemize}
  \end{itemize}
\item
  Como atleta quiero poder ver mi posición en el mapa en tiempo real
  mientras estoy en un seguimiento

  \begin{itemize}
  \itemsep1pt\parskip0pt\parsep0pt
  \item
    Business Value: 34
  \item
    Story Points: 13
  \item
    Criterio de Aceptación:

    \begin{itemize}
    \itemsep1pt\parskip0pt\parsep0pt
    \item
      El atleta puede ver su posición actualizada a intervalos regulares
      en la pantalla.
    \item
      Si se apaga la pantalla o bloquea el teléfono, al reanudar la
      aplicación la actualización de la posición se reanuda en forma
      automática.
    \item
      Si se pierde señal de geolocalización, se notifica al usuario.
    \item
      El atleta puede ver un \emph{timestamp} en cada lugar donde se
      actualizó su posición.
    \item
      El atleta puede desactivar geolocalización si lo desea.
    \end{itemize}
  \end{itemize}
\item
  Como atleta quiero que la aplicación me de un plan de entrenamiento en
  base a los datos ingresados.

  \begin{itemize}
  \itemsep1pt\parskip0pt\parsep0pt
  \item
    Business Value: 21
  \item
    Story Points: 13
  \item
    Criterio de Aceptación:

    \begin{itemize}
    \itemsep1pt\parskip0pt\parsep0pt
    \item
      Si el atleta estableció como objetivo que desea correr una
      maratón, el sistema creará un plan concentrado en larga duración y
      velocidad constante.
    \item
      Si el atleta estableció que desea correr una determinada cantidad
      de kilómetros en un cierto tiempo, el sistema creará un plan con
      entrenamientos de velocidad progresivamente más difíciles hasta
      alcanzar el objetivo.
    \item
      Si el atleta no estableció requerimientos ni de distancia ni de
      tiempo, el programa devolverá una serie de entrenamientos
      recreativos.
    \item
      Si el atleta se encuentra en buen estado físico, los
      entrenamientos constarán de fases con mayor exigencia.
    \item
      La duración y velocidad devueltas serán inversamente
      proporcionales al peso.
    \end{itemize}
  \end{itemize}
\item
  Como corredor quiero poder inicializar el seguimiento de un
  entrenamiento de los que me dio la aplicación para poder empezar a
  correr bajo el plan obtenido.

  \begin{itemize}
  \itemsep1pt\parskip0pt\parsep0pt
  \item
    Business Value: 21
  \item
    Story Points: 13
  \item
    Criterio de Aceptación:

    \begin{itemize}
    \itemsep1pt\parskip0pt\parsep0pt
    \item
      El atleta puede elegir un plan de la lista de disponibles.
    \item
      El atleta puede seleccionar que el plan empiece a correr, y el
      mismo empezara el seguimiento en la primer fase.
    \item
      El atleta puede detener el entrenamiento en cualquier momento.
    \item
      El seguimiento termina de acuerdo a las fases del plan.
    \item
      El atleta puede elegir un entrenamiento si este esta disponible,
      es decir si el plazo para el objetivo no ha expirado.
    \end{itemize}
  \end{itemize}
\item
  Como Secretaría de Deportes de la ciudad de Balvanera y San Cristóbal,
  quiero tener un documento detallado del diseño de la aplicación

  \begin{itemize}
  \itemsep1pt\parskip0pt\parsep0pt
  \item
    Business Value: 21
  \item
    Story Points: 13
  \item
    Criterio de Aceptación:

    \begin{itemize}
    \itemsep1pt\parskip0pt\parsep0pt
    \item
      El diagrama corresponderá al diseño orientado a objetos de la
      aplicación.
    \item
      Se incluirá un diagrama de clases mostrando la taxonomía de los
      objetos a implementar.
    \item
      Se incluirá un diagrama de secuencia para mostrar pasos
      algorítmicos de interés.
    \item
      El mismo utilizará la sintaxis de la materia Ingeniería del
      Software II de la carrera de Ciencias de la Computación de la UBA.
    \item
      Se incluirá un informe detallado de las decisiones de diseño
      tomadas.
    \end{itemize}
  \end{itemize}
\item
  Como atleta quiero poder ver estadísticas de los entrenamientos que ya
  hice

  \begin{itemize}
  \itemsep1pt\parskip0pt\parsep0pt
  \item
    Business Value: 21
  \item
    Story Points: 13
  \item
    Criterio de Aceptación:

    \begin{itemize}
    \itemsep1pt\parskip0pt\parsep0pt
    \item
      El atleta puede entrar a ``Mis estadísticas'' desde el menú
      principal.
    \item
      En ``Mis estadísticas'' tendrá una lista de los entrenamientos que
      ya realizó.
    \item
      Al elegir uno de estos entrenamientos de la lista se mostrará el
      detalle en otra pantalla.
    \item
      El atleta puede ver cuánta distancia recorrió para el
      entrenamiento elegido, cuanto tardo en cada uno, y cual fue su
      velocidad máxima.
    \item
      El atleta puede ver el recorrido que hizo en el entrenamiento en
      un mapa.
    \end{itemize}
  \end{itemize}
\item
  Como Secretaría de Deportes de la ciudad de Balvanera y San Cristóbal,
  quiero tener una demo de funcionalidad básica del producto presentada
  por los desarrolladores.

  \begin{itemize}
  \itemsep1pt\parskip0pt\parsep0pt
  \item
    Business Value: 21
  \item
    Story Points: 8
  \item
    Criterio de Aceptación:

    \begin{itemize}
    \itemsep1pt\parskip0pt\parsep0pt
    \item
      La demo debe permitir realizar el seguimiento de un entrenamiento,
      indicando para una fase si se debe aumentar o disminuir la
      velocidad.
    \item
      La demo debe incluir una presentación de funcionalidad y un
      informe de funcionalidades
    \item
      La demo debe mostrar el recorrido en un mapa.
    \item
      La demo debe correr en el entorno movil de iOS.
    \end{itemize}
  \end{itemize}
\item
  Como atleta quiero que la aplicación siga en que fase del
  entrenamiento me encuentro y me avise si esta cambia

  \begin{itemize}
  \itemsep1pt\parskip0pt\parsep0pt
  \item
    Business Value: 21
  \item
    Story Points: 8
  \item
    Criterio de Aceptación:

    \begin{itemize}
    \itemsep1pt\parskip0pt\parsep0pt
    \item
      En un seguimiento, se lleva cuenta de cual es la fase actual según
      la duración.
    \item
      Cuando el tiempo de la fase actual se acaba, el entrenamiento pasa
      a la siguiente fase o termina si no hay.
    \item
      Los datos de seguimiento se actualizan cuando se actualiza la
      fase.
    \item
      La aplicación genera una notificación auditiva cuando se termine
      el tiempo de la fase actual y se pase a otra.
    \item
      La aplicación genera una notificación cuando se terminó la última
      fase del entrenamiento.
    \item
      No se genera esa notificación particular por otro motivo.
    \end{itemize}
  \end{itemize}
\item
  Como atleta quiero poder ingresar el objetivo de mi entrenamiento.

  \begin{itemize}
  \itemsep1pt\parskip0pt\parsep0pt
  \item
    Business Value: 13
  \item
    Story Points: 8
  \item
    Criterio de Aceptación:

    \begin{itemize}
    \item
      El atleta al ingresar a ``Crear plan'' tendrá la opción
      ``objetivo''
    \item
      El atleta recibirá una lista de objetivos posibles.
    \item
      El atleta debe ingresar sus objetivos propuestos entre las
      opciones:
    \item
      \begin{itemize}
      \itemsep1pt\parskip0pt\parsep0pt
      \item
        Correr 5 km sin tiempo.
      \end{itemize}
    \item
      \begin{itemize}
      \itemsep1pt\parskip0pt\parsep0pt
      \item
        Terminar un maratón olímpico.
      \end{itemize}
    \item
      \begin{itemize}
      \itemsep1pt\parskip0pt\parsep0pt
      \item
        Correr 7 km en 35 minutos. y otras opciones y posibilidades
        decididas durante la implementación
      \end{itemize}
    \item
      Los objetivos corresponderán a una distancia a recorrer, y
      potencialmente un tiempo en el que se debe correr esta distancia.
    \end{itemize}
  \end{itemize}
\item
  Como corredor quiero tener estadisticas generales de mi performance
  como atleta

  \begin{itemize}
  \itemsep1pt\parskip0pt\parsep0pt
  \item
    Business Value: 13
  \item
    Story Points: 8
  \item
    Criterio de Aceptación:

    \begin{itemize}
    \itemsep1pt\parskip0pt\parsep0pt
    \item
      El atleta podrá entrar a ``Mis estadísticas'' dentro del menu
      principal.
    \item
      El atleta podrá entrar a ``Estadísticas generales'' dentro de la
      vista de estadísticas.
    \item
      Las estadísticas generales se calcularán en base a todos los
      entrenamientos del usuario.
    \item
      Estas estadísticas estarán presentadas cuando el atleta presione
      ``Estadísticas generales'' en una nueva vista.
    \item
      Cada estadística sera presentada con una descripción del valor
      obtenido.
    \item
      Estas estadísticas contendran como mínimo: Kilometros totales
      recorridos y velocidad máxima histórica.
    \end{itemize}
  \end{itemize}
\item
  Como corredor quiero poder ver la velocidad promedio y la duración de
  cada fase de un entrenamiento para saber el criterio con el que la
  aplicación mide mi performance corriendo.

  \begin{itemize}
  \itemsep1pt\parskip0pt\parsep0pt
  \item
    Business Value: 13
  \item
    Story Points: 8
  \item
    Criterio de Aceptación:

    \begin{itemize}
    \itemsep1pt\parskip0pt\parsep0pt
    \item
      El atleta podrá entrar a los planes que tiene creados.
    \item
      El atleta debe poder ver los entrenamientos que tiene listos
      dentro de ese plan.
    \item
      El atleta debe poder elegir un entrenamiento de los que la
      aplicación ha preparado.
    \item
      El atleta debe poder examinar las fases de un entrenamiento.
    \item
      El atleta debe poder elegir una fase para ver el detalle de la
      misma.
    \item
      El atleta debe ver para la fase elegida un rango de velocidades en
      km/h que son aceptables.
    \item
      El atleta debe poder ver para la fase elegida cuanto tiempo dura
      la misma en minutos.
    \end{itemize}
  \end{itemize}
\item
  Como atleta quiero poder ingresar los datos de mi estado físico.

  \begin{itemize}
  \itemsep1pt\parskip0pt\parsep0pt
  \item
    Business Value: 8
  \item
    Story Points: 5
  \item
    Criterio de Aceptación:

    \begin{itemize}
    \itemsep1pt\parskip0pt\parsep0pt
    \item
      El atleta podrá ingresar a ``Mi perfil'' desde el menú principal
    \item
      El atleta podrá ingresar su peso en kilogramos.
    \item
      El atleta podrá ingresar su altura en cm.
    \item
      El atleta podrá especificar mayores detalles usando categorías
      basadas en si ya corrió una carrera o no, ya corrió un maratón o
      no, su mejor marca de distancia en una carrera y en un maratón.
    \item
      La aplicación guardará registro del valor actual de ambos datos.
    \end{itemize}
  \end{itemize}
\item
  Como atleta quiero poder ajustar el consumo de batería

  \begin{itemize}
  \itemsep1pt\parskip0pt\parsep0pt
  \item
    Business Value: 8
  \item
    Story Points: 3
  \item
    Criterio de Aceptación:

    \begin{itemize}
    \itemsep1pt\parskip0pt\parsep0pt
    \item
      El atleta puede entrar a ``Configuración'' desde el menu
      principal.
    \item
      El atleta dispondrá en esta ventana de los distintos tipo de
      batería disponibles.
    \item
      El atleta puede seleccionar dentro de los niveles disponibles,
      como mínimo bajo, medio y alto.
    \item
      La aplicación debe poder correr más tiempo bajo un plan de consumo
      bajo que en uno alto.
    \item
      El atleta puede determinar que impacto tiene en las
      funcionalidades de la aplicación el cambio de consumo de batería.
    \end{itemize}
  \end{itemize}
\item
  Como atleta quiero poder ingresar el plazo estipulado para mi
  entrenamiento si así lo deseo.

  \begin{itemize}
  \itemsep1pt\parskip0pt\parsep0pt
  \item
    Business Value: 5
  \item
    Story Points: 2
  \item
    Criterio de Aceptación:

    \begin{itemize}
    \itemsep1pt\parskip0pt\parsep0pt
    \item
      El atleta puede estipular ``Plazo'' dentro de la selección de
      ``Crear Plan''
    \item
      El atleta puede elegir un plazo estipulado para la finalización de
      cada uno de sus objetivos.
    \item
      Los planes consistirán de una fecha límite en la que se quiere
      lograr el objetivo.
    \end{itemize}
  \end{itemize}
\item
  Como atleta quiero poder ingresar mi frecuencia semanal con la que
  puedo entrenar.

  \begin{itemize}
  \itemsep1pt\parskip0pt\parsep0pt
  \item
    Business Value: 5
  \item
    Story Points: 1
  \item
    Criterio de Aceptación:

    \begin{itemize}
    \itemsep1pt\parskip0pt\parsep0pt
    \item
      El atleta podrá ingresar ``Frecuencia semanal'' dentro de sus
      datos de ``Mi Perfil''
    \item
      El atleta debe ingresar la frecuencia semanal con la que puede
      entrenar.
    \item
      Los valores ingresados deben ser una cantidad de días entre 1 y 7.
    \item
      La aplicación guardará registro del valor.
    \end{itemize}
  \end{itemize}
\item
  Como atleta quiero que las actualizaciones de posición sean acordes al
  nivel de batería seleccionado.

  \begin{itemize}
  \itemsep1pt\parskip0pt\parsep0pt
  \item
    Business Value: 5
  \item
    Story Points: 8
  \item
    Criterio de Aceptación:

    \begin{itemize}
    \itemsep1pt\parskip0pt\parsep0pt
    \item
      La posición se actualiza cada 10 segundos si el nivel de consumo
      batería elegido es alto.
    \item
      La posición se actualiza cada minuto si el nivel de consumo de
      batería es bajo.
    \item
      Para los demás niveles de batería también se indicará una
      frecuencia de actualización de posición al momento de la
      implementación.
    \end{itemize}
  \end{itemize}
\item
  Como atleta quiero que las notificaciones de velocidad de la
  aplicación sean acordes al de batería seleccionado.

  \begin{itemize}
  \itemsep1pt\parskip0pt\parsep0pt
  \item
    Business Value: 5
  \item
    Story Points: 8
  \item
    Criterio de Aceptación:

    \begin{itemize}
    \itemsep1pt\parskip0pt\parsep0pt
    \item
      Si el atleta eligió un consumo bajo, las notificaciones son
      pitidos y ocurren cada 1 minuto.
    \item
      Si el atleta eligió un consumo alto de batería, las notificaciones
      son temas musicales preelegidos por la app y ocurren cada 10
      segundos.
    \item
      Para otros niveles de batería se determinará también una
      frecuencia de notificaciones y calidad de las mismas en el momento
      de la implementación.
    \end{itemize}
  \end{itemize}
\item
  Como atleta quiero poder publicar los recorridos de los entrenamientos
  en redes sociales y aplicaciones de geolocalización.

  \begin{itemize}
  \itemsep1pt\parskip0pt\parsep0pt
  \item
    Business Value: 1
  \item
    Story Points: 10
  \item
    Criterio de Aceptación:

    \begin{itemize}
    \itemsep1pt\parskip0pt\parsep0pt
    \item
      El atleta puede seleccionar en qué red social o aplicación
      publicar.
    \item
      El atleta puede seleccionar qué recorrido puede publicar.
    \item
      El atleta puede escribir un mensaje a agregar además de los datos
      de su entrenamiento.
    \item
      La publicación es visible por los demás miembros de la red social
      de acuerdo a las reglas de privacidad de la misma.
    \item
      Solo los datos explícitamente indicados por el usuario son
      publicados en la red social correspondiente.
    \item
      El atleta puede decidir si quiere que se muestre el recorrido que
      realizó, la velocidad con la que corrió, etc., para cada tipo de
      dato a compartir.
    \end{itemize}
  \end{itemize}
\end{enumerate}


\subsection{Sprint backlog}

A continuaci\'on detallamos las \textit{user stories} que incluimos en el actual \textit{sprint}. Las mismas son:

\begin{itemize}
	\item 3) Como atleta quiero que la aplicación siga mi fase dentro del plan para que me avise si lo estoy siguiendo o tengo que modificar mi marcha.
	\item 13) Como corredor quiero poder ver la velocidad promedio y la duración de cada fase de un entrenamiento para saber el criterio con el que la aplicación mide mi performance corriendo.
	\item 14) Como atleta quiero poder ver mi posición en el mapa en tiempo real.
	\item 16) Como corredor quiero poder inicializar el seguimiento de un entrenamiento de los que me dio la aplicación y que están dentro del plazo para poder empezar a correr bajo el plan obtenido.	
\end{itemize}

y a continuaci\'on incluimos el detalles de las tareas involucradas y la dificultad asociada a cada una.

\begin{itemize}
	\itemsep 0em
	\item Investigar cómo reproducir una canción en cada formato estándar (mp3,wav, etc).
	\item[] \hfill \textit{Dificultad: 3 - Horas: 5}
	\item Investigar cómo guardar canciones en el teléfono y como volver a leerlas de almacenamiento permanente.
	\item[] \hfill \textit{Dificultad: 3 - Horas: 8}
	\item Codificar la lógica para que según el tipo de alerta se elija un sonido a reproducir (potencialmente leyendo de almacenamiento el mismo) y se lo reproduzca. 
	\item[] \hfill \textit{Dificultad: 5 - Horas: 8}
	\item Codificar la lógica para que si la velocidad no esta en el rango, se envíe una alerta y se muestre en pantalla la diferencia y según ese rang, cuan ``grave'' es la alerta. 
	\item[] \hfill \textit{Dificultad: 3 - Horas: 8}
	\item Testear para los 3 tipos de condiciones de marcha válida e inválida.
	\item[] \hfill \textit{Dificultad: 2 - Horas: 5}
	\item Investigar cómo mostrar datos numéricos de velocidad y duración por la interfaz del celular, y como actualizar la vista cuando estos cambian.
	\item[] \hfill \textit{Dificultad: 3 - Horas: 2}
	\item Investigar un algoritmo para lograr calcular la velocidad promedio a medida que llegan los datos.
	\item[] \hfill \textit{Dificultad: 2 - Horas: 2}
	\item Testear que el promedio calculado es correcto incluso considerando actualizaciones de velocidad y tiempo poco frecuentes 
		(por ejemplo en un modo de batería bajo).
	\item[] \hfill \textit{Dificultad: 5 - Horas: 3}
	\item Crear una vista para mostrar los datos.
	\item[] \hfill \textit{Dificultad: 3 - Horas: 2}
	\item Implementar la lógica para calcular los datos de velocidad promedio.
	\item[] \hfill \textit{Dificultad: 5 - Horas: 2}
	\item  Investigar como obtener el tiempo actual del celular.
	\item[] \hfill \textit{Dificultad: 3 - Horas: 1}
	\item Implementar la lógica para obtener la posición actual del teléfono.
	\item[] \hfill \textit{Dificultad: 8 - Horas: 2}
	\item Incorporar el uso de mapas de otras fuentes dentro de la aplicación
	\item[] \hfill \textit{Dificultad: 3 - Horas 1}
	\item Agregar una vista con un mapa en la pantalla.
	\item[] \hfill \textit{Dificultad: 5 - Horas: 1}
	\item Implementar lógica para centrar ese mapa en una posición indicada.
	\item[] \hfill \textit{Dificultad: 3 - Horas: 1}
	\item Investigar qué pasa cuando se pasa la aplicación al \textit{background} y cuando vuelve, 
			implementar la lógica que mantenga actualizando a la aplicación incluso en background.
	\item[] \hfill \textit{Dificultad: 3 - Horas: 2}
	\item Testear que efectivamente la posición se actualice al desplazar el teléfono.
	\item[] \hfill \textit{Dificultad: 2 - Horas: 2}
	\item Testear que el mapa se centre correctamente.
	\item[] \hfill \textit{Dificultad: 2 - Horas: 1}
	\item Implementar la lógica para dibujar un recorrido en el mapa dados los puntos y un \textit{timestamp} para cada uno.
	\item[] \hfill \textit{Dificultad: 5 - Horas: 5}
	\item Implementar una vista de seguimientos filtrados por plazo	
	\item[] \hfill \textit{Dificultad: 3 - Horas: 3}
	\item Implementar que la selecci\'on del entrenamiento inicie la aplicación de seguimiento para el mismo.
	\item[] \hfill \textit{Dificultad: 5 - Horas: 5}
	\item Implementar que se pueda detener un plan de entrenamiento en cualquier momento del mismo.
	\item[] \hfill \textit{Dificultad: 3 - Horas: 3}
	\item Investigar maneras de ordenar la lista según relevancia de entrenamientos.
	\item[] \hfill \textit{Dificultad: 2 - Horas 5}
\end{itemize}

\begin{enumerate}
\def\labelenumi{\arabic{enumi})}
\setcounter{enumi}{1}
\item
  Como atleta quiero poder ver mi posición en el mapa en tiempo real
  mientras estoy en un seguimiento

  \begin{itemize}
  \itemsep1pt\parskip0pt\parsep0pt
  \item
    Business Value: 34
  \item
    Story Points: 13
  \item
    Criterio de Aceptación:

    \begin{itemize}
    \itemsep1pt\parskip0pt\parsep0pt
    \item
      El atleta puede ver su posición actualizada a intervalos regulares
      en la pantalla.
    \item
      Si se apaga la pantalla o bloquea el teléfono, al reanudar la
      aplicación la actualización de la posición se reanuda en forma
      automática.
    \item
      Si se pierde señal de geolocalización, se notifica al usuario.
    \item
      El atleta puede ver un \emph{timestamp} en cada lugar donde se
      actualizó su posición.
    \item
      El atleta puede desactivar geolocalización si lo desea.
    \end{itemize}
  \item
    Tareas
  \end{itemize}

  \begin{longtable}[c]{@{}ll@{}}
  \hline\noalign{\medskip}
  \begin{minipage}[b]{0.92\columnwidth}\raggedright
  Tarea
  \end{minipage} & \begin{minipage}[b]{0.08\columnwidth}\raggedright
  Tiempo
  \end{minipage}
  \\\noalign{\medskip}
  \hline\noalign{\medskip}
  \begin{minipage}[t]{0.92\columnwidth}\raggedright
  Diseñar las pantallas y vistas de la aplicación
  \end{minipage} & \begin{minipage}[t]{0.08\columnwidth}\raggedright
  1h
  \end{minipage}
  \\\noalign{\medskip}
  \begin{minipage}[t]{0.92\columnwidth}\raggedright
  Instalar y preparar el entorno de desarrollo
  \end{minipage} & \begin{minipage}[t]{0.08\columnwidth}\raggedright
  2h
  \end{minipage}
  \\\noalign{\medskip}
  \begin{minipage}[t]{0.92\columnwidth}\raggedright
  Armar proyecto git
  \end{minipage} & \begin{minipage}[t]{0.08\columnwidth}\raggedright
  30min
  \end{minipage}
  \\\noalign{\medskip}
  \begin{minipage}[t]{0.92\columnwidth}\raggedright
  Investigar el lenguaje Objective-C y la plataforma iOS
  \end{minipage} & \begin{minipage}[t]{0.08\columnwidth}\raggedright
  2h
  \end{minipage}
  \\\noalign{\medskip}
  \begin{minipage}[t]{0.92\columnwidth}\raggedright
  Investigar como obtener el tiempo actual del celular
  \end{minipage} & \begin{minipage}[t]{0.08\columnwidth}\raggedright
  1h
  \end{minipage}
  \\\noalign{\medskip}
  \begin{minipage}[t]{0.92\columnwidth}\raggedright
  Implementar la lógica para obtener la posición actual del teléfono
  \end{minipage} & \begin{minipage}[t]{0.08\columnwidth}\raggedright
  1h
  \end{minipage}
  \\\noalign{\medskip}
  \begin{minipage}[t]{0.92\columnwidth}\raggedright
  Agregar una vista con un mapa en la pantalla
  \end{minipage} & \begin{minipage}[t]{0.08\columnwidth}\raggedright
  1h
  \end{minipage}
  \\\noalign{\medskip}
  \begin{minipage}[t]{0.92\columnwidth}\raggedright
  Implementar lógica para centrar ese mapa en una posición indicada
  \end{minipage} & \begin{minipage}[t]{0.08\columnwidth}\raggedright
  1h
  \end{minipage}
  \\\noalign{\medskip}
  \begin{minipage}[t]{0.92\columnwidth}\raggedright
  Investigar qué pasa cuando se pasa la aplicación al background y
  cuando vuelve, implementar la lógica que mantenga actualizando la
  aplicación incluso en background.
  \end{minipage} & \begin{minipage}[t]{0.08\columnwidth}\raggedright
  2h
  \end{minipage}
  \\\noalign{\medskip}
  \begin{minipage}[t]{0.92\columnwidth}\raggedright
  Testear que el mapa se centre correctamente.
  \end{minipage} & \begin{minipage}[t]{0.08\columnwidth}\raggedright
  30min
  \end{minipage}
  \\\noalign{\medskip}
  \begin{minipage}[t]{0.92\columnwidth}\raggedright
  Testear que efectivamente la posición se actualice al desplazar el
  teléfono.
  \end{minipage} & \begin{minipage}[t]{0.08\columnwidth}\raggedright
  30min
  \end{minipage}
  \\\noalign{\medskip}
  \begin{minipage}[t]{0.92\columnwidth}\raggedright
  Implementar la lógica para dibujar un recorrido en el mapa dados los
  puntos y un timestamp de los mismos.
  \end{minipage} & \begin{minipage}[t]{0.08\columnwidth}\raggedright
  3h
  \end{minipage}
  \\\noalign{\medskip}
  \hline
  \end{longtable}
\item
  Como atleta quiero que la aplicación siga mi fase dentro del plan para
  que me avise si lo estoy siguiendo o tengo que modificar mi marcha.

  \begin{itemize}
  \item
    Business Value: 34
  \item
    Story Points: 8
  \item
    Criterio de Aceptación:

    \begin{itemize}
    \itemsep1pt\parskip0pt\parsep0pt
    \item
      El atleta debe poder saber que plan y que entrenamiento esta
      actualmente siguiendo.
    \item
      El atleta debe poder saber en que fase de que entrenamiento esta
      en este momento.
    \item
      El atleta debe saber a que velocidad esta corriendo actualmente.
    \item
      El atleta debe recibir una notificación de que debe aumentar la
      marcha si está yendo más lento que la fase.
    \item
      El atleta debe recibir una notificación de que debe disminuir la
      marcha si está yendo más despacio que la fase.
    \item
      El atleta debe recibir una notificación en intervalos regulares si
      se está manteniendo en una marcha aceptable.
    \item
      Cada alerta por marcha inválida se repetirá a intervalos regulares
      mientras persista la condición.
    \item
      En la interfaz gráfica debe aparecer la velocidad a la que el
      atleta debiera correr para llegar a la distancia implícita
      requerida (resultado de multiplicar la velocidad media de la fase
      por la duración de la misma).
    \end{itemize}
  \item
    Tareas
  \end{itemize}

  \begin{longtable}[c]{@{}ll@{}}
  \hline\noalign{\medskip}
  \begin{minipage}[b]{0.92\columnwidth}\raggedright
  Tarea
  \end{minipage} & \begin{minipage}[b]{0.08\columnwidth}\raggedright
  Tiempo
  \end{minipage}
  \\\noalign{\medskip}
  \hline\noalign{\medskip}
  \begin{minipage}[t]{0.92\columnwidth}\raggedright
  Implementar la lógica para calcular los datos de velocidad promedio
  \end{minipage} & \begin{minipage}[t]{0.08\columnwidth}\raggedright
  2h
  \end{minipage}
  \\\noalign{\medskip}
  \begin{minipage}[t]{0.92\columnwidth}\raggedright
  Investigar un algoritmo para lograr calcular la velocidad promedio a
  medida que llegan los datos.
  \end{minipage} & \begin{minipage}[t]{0.08\columnwidth}\raggedright
  2h
  \end{minipage}
  \\\noalign{\medskip}
  \begin{minipage}[t]{0.92\columnwidth}\raggedright
  Codificar la lógica para obtener la velocidad necesaria para llegar a
  la velocidad requerida.
  \end{minipage} & \begin{minipage}[t]{0.08\columnwidth}\raggedright
  2h
  \end{minipage}
  \\\noalign{\medskip}
  \begin{minipage}[t]{0.92\columnwidth}\raggedright
  Codificar la lógica para que si la velocidad no esta en el rango, se
  envié una alerta correspondiente a cuan ``grave'' es el nivel de
  alerta.
  \end{minipage} & \begin{minipage}[t]{0.08\columnwidth}\raggedright
  5h
  \end{minipage}
  \\\noalign{\medskip}
  \begin{minipage}[t]{0.92\columnwidth}\raggedright
  Investigar como reproducir una canción en cada formato estandar (mp3,
  wav, etc.)
  \end{minipage} & \begin{minipage}[t]{0.08\columnwidth}\raggedright
  2h
  \end{minipage}
  \\\noalign{\medskip}
  \begin{minipage}[t]{0.92\columnwidth}\raggedright
  Codificar la vista para que muestre la velocidad objetivo, duración y
  distancia recorrida en esta fase
  \end{minipage} & \begin{minipage}[t]{0.08\columnwidth}\raggedright
  5h
  \end{minipage}
  \\\noalign{\medskip}
  \begin{minipage}[t]{0.92\columnwidth}\raggedright
  Crear un mock de un entrenamiento de una fase para poder realizar las
  pruebas de seguimiento de velocidad y actualización de la vista
  \end{minipage} & \begin{minipage}[t]{0.08\columnwidth}\raggedright
  30min
  \end{minipage}
  \\\noalign{\medskip}
  \begin{minipage}[t]{0.92\columnwidth}\raggedright
  Crear un mock de batería y configurador para testear los tipos de
  notificaciones y la frecuencia de actualización
  \end{minipage} & \begin{minipage}[t]{0.08\columnwidth}\raggedright
  1h
  \end{minipage}
  \\\noalign{\medskip}
  \begin{minipage}[t]{0.92\columnwidth}\raggedright
  Codificar un sistema de administración de actualizaciones para proveer
  los datos de posición a los distintos módulos
  \end{minipage} & \begin{minipage}[t]{0.08\columnwidth}\raggedright
  2h
  \end{minipage}
  \\\noalign{\medskip}
  \hline
  \end{longtable}
\item
  Como Secretaría de Deportes de la ciudad de Balvanera y San Cristóbal,
  quiero tener un documento detallado del diseño de la aplicación

  \begin{itemize}
  \itemsep1pt\parskip0pt\parsep0pt
  \item
    Business Value: 21
  \item
    Story Points: 13
  \item
    Criterio de Aceptación:

    \begin{itemize}
    \itemsep1pt\parskip0pt\parsep0pt
    \item
      El diagrama corresponderá al diseño orientado a objetos de la
      aplicación.
    \item
      Se incluirá un diagrama de clases mostrando la taxonomía de los
      objetos a implementar.
    \item
      Se incluirá un diagrama de secuencia para mostrar pasos
      algorítmicos de interés.
    \item
      El mismo utilizará la sintaxis de la materia Ingeniería del
      Software II de la carrera de Ciencias de la Computación de la UBA.
    \item
      Se incluirá un informe detallado de las decisiones de diseño
      tomadas.
    \end{itemize}
  \item
    Tareas:
  \end{itemize}

  \begin{longtable}[c]{@{}ll@{}}
  \hline\noalign{\medskip}
  \begin{minipage}[b]{0.92\columnwidth}\raggedright
  Tarea
  \end{minipage} & \begin{minipage}[b]{0.08\columnwidth}\raggedright
  Tiempo
  \end{minipage}
  \\\noalign{\medskip}
  \hline\noalign{\medskip}
  \begin{minipage}[t]{0.92\columnwidth}\raggedright
  Hacer el diagrama de secuencia del sistema de notificaciones en
  seguimiento
  \end{minipage} & \begin{minipage}[t]{0.08\columnwidth}\raggedright
  1h
  \end{minipage}
  \\\noalign{\medskip}
  \begin{minipage}[t]{0.92\columnwidth}\raggedright
  Realizar el informe dei diseño justificando las decisiones
  \end{minipage} & \begin{minipage}[t]{0.08\columnwidth}\raggedright
  6h
  \end{minipage}
  \\\noalign{\medskip}
  \begin{minipage}[t]{0.92\columnwidth}\raggedright
  Determinar cuales son y realizar los diagramas de secuencia
  pertinentes
  \end{minipage} & \begin{minipage}[t]{0.08\columnwidth}\raggedright
  8h
  \end{minipage}
  \\\noalign{\medskip}
  \begin{minipage}[t]{0.92\columnwidth}\raggedright
  Hacer diagrama de clases para el mecanismo de actualización de datos
  en un seguimiento
  \end{minipage} & \begin{minipage}[t]{0.08\columnwidth}\raggedright
  8h
  \end{minipage}
  \\\noalign{\medskip}
  \begin{minipage}[t]{0.92\columnwidth}\raggedright
  Hacer el diagrama de clases de los controladores y su relación con los
  objetos de negocio
  \end{minipage} & \begin{minipage}[t]{0.08\columnwidth}\raggedright
  8h
  \end{minipage}
  \\\noalign{\medskip}
  \begin{minipage}[t]{0.92\columnwidth}\raggedright
  Hacer el diagrama de clases para el mecanismo de creación de batería y
  su impacto
  \end{minipage} & \begin{minipage}[t]{0.08\columnwidth}\raggedright
  3h
  \end{minipage}
  \\\noalign{\medskip}
  \begin{minipage}[t]{0.92\columnwidth}\raggedright
  Hacer el diagrama para el mecanismo de craeción de planes básicos
  \end{minipage} & \begin{minipage}[t]{0.08\columnwidth}\raggedright
  2h
  \end{minipage}
  \\\noalign{\medskip}
  \begin{minipage}[t]{0.92\columnwidth}\raggedright
  Hacer diagrama de clases para el mecanismo de estadísticas de
  entrenamientos y sus almacenamientos.
  \end{minipage} & \begin{minipage}[t]{0.08\columnwidth}\raggedright
  3h
  \end{minipage}
  \\\noalign{\medskip}
  \hline
  \end{longtable}
\item
  Como corredor quiero poder inicializar el seguimiento de un
  entrenamiento de los que me dio la aplicación para poder empezar a
  correr bajo el plan obtenido.

  \begin{itemize}
  \itemsep1pt\parskip0pt\parsep0pt
  \item
    Business Value: 21
  \item
    Story Points: 13
  \item
    Criterio de Aceptación:

    \begin{itemize}
    \itemsep1pt\parskip0pt\parsep0pt
    \item
      El atleta puede elegir un plan de la lista de disponibles.
    \item
      El atleta puede seleccionar que el plan empiece a correr, y el
      mismo empezara el seguimiento en la primer fase.
    \item
      El atleta puede detener el entrenamiento en cualquier momento.
    \item
      El seguimiento termina de acuerdo a las fases del plan.
    \item
      El atleta puede elegir un entrenamiento si este esta disponible,
      es decir si el plazo para el objetivo no ha expirado.
    \end{itemize}
  \item
    Tareas
  \end{itemize}

  \begin{longtable}[c]{@{}ll@{}}
  \hline\noalign{\medskip}
  \begin{minipage}[b]{0.92\columnwidth}\raggedright
  Tarea
  \end{minipage} & \begin{minipage}[b]{0.08\columnwidth}\raggedright
  Tiempo
  \end{minipage}
  \\\noalign{\medskip}
  \hline\noalign{\medskip}
  \begin{minipage}[t]{0.92\columnwidth}\raggedright
  Implementar el controlador para que al apretar el boton de iniciar el
  seguimiento, inicie el seguimiento para el entrenamiento indicado con
  los datos del mismo
  \end{minipage} & \begin{minipage}[t]{0.08\columnwidth}\raggedright
  5h
  \end{minipage}
  \\\noalign{\medskip}
  \begin{minipage}[t]{0.92\columnwidth}\raggedright
  Implementar que se pueda detener un plan de entrenamiento en cualquier
  momento del mismo
  \end{minipage} & \begin{minipage}[t]{0.08\columnwidth}\raggedright
  3h
  \end{minipage}
  \\\noalign{\medskip}
  \begin{minipage}[t]{0.92\columnwidth}\raggedright
  Investigar maneras de ordenar la lista según relevancia de
  entrenamientos
  \end{minipage} & \begin{minipage}[t]{0.08\columnwidth}\raggedright
  2h
  \end{minipage}
  \\\noalign{\medskip}
  \begin{minipage}[t]{0.92\columnwidth}\raggedright
  Crear un mock de un okan ara testear a vsta de entrenamientos e
  indicar para empezar
  \end{minipage} & \begin{minipage}[t]{0.08\columnwidth}\raggedright
  3h
  \end{minipage}
  \\\noalign{\medskip}
  \begin{minipage}[t]{0.92\columnwidth}\raggedright
  Utilizar el mock de entrenamientos en el mock de planes para asociar
  un plan a entrenamientos de prueba
  \end{minipage} & \begin{minipage}[t]{0.08\columnwidth}\raggedright
  30min
  \end{minipage}
  \\\noalign{\medskip}
  \begin{minipage}[t]{0.92\columnwidth}\raggedright
  Implementar la vista para permitir iniciar un entrenamiento y mostrar
  sus fases y datos, con un botón para iniciar.
  \end{minipage} & \begin{minipage}[t]{0.08\columnwidth}\raggedright
  2h
  \end{minipage}
  \\\noalign{\medskip}
  \begin{minipage}[t]{0.92\columnwidth}\raggedright
  Implementar la vista de planes disponibles para el usuario
  \end{minipage} & \begin{minipage}[t]{0.08\columnwidth}\raggedright
  1h
  \end{minipage}
  \\\noalign{\medskip}
  \begin{minipage}[t]{0.92\columnwidth}\raggedright
  Testear que el seguimiento sea inicializado con los datos del
  entrenamiento seleccionado
  \end{minipage} & \begin{minipage}[t]{0.08\columnwidth}\raggedright
  1h
  \end{minipage}
  \\\noalign{\medskip}
  \hline
  \end{longtable}
\item
  Como Secretaría de Deportes de la ciudad de Balvanera y San Cristóbal,
  quiero tener un documento detallado del diseño de la aplicación

  \begin{itemize}
  \itemsep1pt\parskip0pt\parsep0pt
  \item
    Business Value: 21
  \item
    Story Points: 13
  \item
    Criterio de Aceptación:

    \begin{itemize}
    \itemsep1pt\parskip0pt\parsep0pt
    \item
      El diagrama corresponderá al diseño orientado a objetos de la
      aplicación.
    \item
      Se incluirá un diagrama de clases mostrando la taxonomía de los
      objetos a implementar.
    \item
      Se incluirá un diagrama de secuencia para mostrar pasos
      algorítmicos de interés.
    \item
      El mismo utilizará la sintaxis de la materia Ingeniería del
      Software II de la carrera de Ciencias de la Computación de la UBA.
    \item
      Se incluirá un informe detallado de las decisiones de diseño
      tomadas.
    \end{itemize}
  \item
    Tareas
  \end{itemize}

  \begin{longtable}[c]{@{}lr@{}}
  \hline\noalign{\medskip}
  \begin{minipage}[b]{0.89\columnwidth}\raggedright
  Tarea
  \end{minipage} & \begin{minipage}[b]{0.11\columnwidth}\raggedleft
  Tiempo
  \end{minipage}
  \\\noalign{\medskip}
  \hline\noalign{\medskip}
  \begin{minipage}[t]{0.89\columnwidth}\raggedright
  Conseguir dispositivo de prueba
  \end{minipage} & \begin{minipage}[t]{0.11\columnwidth}\raggedleft
  2h
  \end{minipage}
  \\\noalign{\medskip}
  \begin{minipage}[t]{0.89\columnwidth}\raggedright
  Preparar la presentación del producto
  \end{minipage} & \begin{minipage}[t]{0.11\columnwidth}\raggedleft
  2h
  \end{minipage}
  \\\noalign{\medskip}
  \begin{minipage}[t]{0.89\columnwidth}\raggedright
  Preparar informe de funcionalidades
  \end{minipage} & \begin{minipage}[t]{0.11\columnwidth}\raggedleft
  3h
  \end{minipage}
  \\\noalign{\medskip}
  \begin{minipage}[t]{0.89\columnwidth}\raggedright
  Generar y probar el instalable de la demo para el dispositivo
  \end{minipage} & \begin{minipage}[t]{0.11\columnwidth}\raggedleft
  30min
  \end{minipage}
  \\\noalign{\medskip}
  \begin{minipage}[t]{0.89\columnwidth}\raggedright
  Preparar informe de seguimiento de trabajo realizado
  \end{minipage} & \begin{minipage}[t]{0.11\columnwidth}\raggedleft
  2h
  \end{minipage}
  \\\noalign{\medskip}
  \begin{minipage}[t]{0.89\columnwidth}\raggedright
  Preparar la charla de presentación
  \end{minipage} & \begin{minipage}[t]{0.11\columnwidth}\raggedleft
  2h30min
  \end{minipage}
  \\\noalign{\medskip}
  \begin{minipage}[t]{0.89\columnwidth}\raggedright
  Preparar la restrospectiva sobre el proyecto
  \end{minipage} & \begin{minipage}[t]{0.11\columnwidth}\raggedleft
  1h
  \end{minipage}
  \\\noalign{\medskip}
  \hline
  \end{longtable}
\item
  Como corredor quiero poder ver la velocidad promedio y la duración de
  cada fase de un entrenamiento para saber el criterio con el que la
  aplicación mide mi performance corriendo.

  \begin{itemize}
  \itemsep1pt\parskip0pt\parsep0pt
  \item
    Business Value: 13
  \item
    Story Points: 8
  \item
    Criterio de Aceptación:

    \begin{itemize}
    \itemsep1pt\parskip0pt\parsep0pt
    \item
      El atleta podrá entrar a los planes que tiene creados.
    \item
      El atleta debe poder ver los entrenamientos que tiene listos
      dentro de ese plan.
    \item
      El atleta debe poder elegir un entrenamiento de los que la
      aplicación ha preparado.
    \item
      El atleta debe poder examinar las fases de un entrenamiento.
    \item
      El atleta debe poder elegir una fase para ver el detalle de la
      misma.
    \item
      El atleta debe ver para la fase elegida un rango de velocidades en
      km/h que son aceptables.
    \item
      El atleta debe poder ver para la fase elegida cuanto tiempo dura
      la misma en minutos.
    \end{itemize}
  \item
    Tareas
  \end{itemize}

  \begin{longtable}[c]{@{}ll@{}}
  \hline\noalign{\medskip}
  \begin{minipage}[b]{0.92\columnwidth}\raggedright
  Tarea
  \end{minipage} & \begin{minipage}[b]{0.08\columnwidth}\raggedright
  Tiempo
  \end{minipage}
  \\\noalign{\medskip}
  \hline\noalign{\medskip}
  \begin{minipage}[t]{0.92\columnwidth}\raggedright
  Implementar modelos Corredor, Plan, Fase y Entrenamiento
  \end{minipage} & \begin{minipage}[t]{0.08\columnwidth}\raggedright
  1h
  \end{minipage}
  \\\noalign{\medskip}
  \begin{minipage}[t]{0.92\columnwidth}\raggedright
  Mockear datos de los modelos
  \end{minipage} & \begin{minipage}[t]{0.08\columnwidth}\raggedright
  30min
  \end{minipage}
  \\\noalign{\medskip}
  \begin{minipage}[t]{0.92\columnwidth}\raggedright
  Crear una vista para mostrar los datos de las fases de entrenamiento
  que incluyan la duración y rango de velocidades.
  \end{minipage} & \begin{minipage}[t]{0.08\columnwidth}\raggedright
  2h
  \end{minipage}
  \\\noalign{\medskip}
  \begin{minipage}[t]{0.92\columnwidth}\raggedright
  Investigar como mostrar datos numéricos de velocidad y duración por la
  interfaz del celular.
  \end{minipage} & \begin{minipage}[t]{0.08\columnwidth}\raggedright
  30min
  \end{minipage}
  \\\noalign{\medskip}
  \begin{minipage}[t]{0.92\columnwidth}\raggedright
  Crear una lista que muestre y permita elegir los planes de
  entrenamiento
  \end{minipage} & \begin{minipage}[t]{0.08\columnwidth}\raggedright
  1h
  \end{minipage}
  \\\noalign{\medskip}
  \begin{minipage}[t]{0.92\columnwidth}\raggedright
  Crear una lista para que muestre y permita elegir los entrenamientos
  del usuario
  \end{minipage} & \begin{minipage}[t]{0.08\columnwidth}\raggedright
  1h
  \end{minipage}
  \\\noalign{\medskip}
  \begin{minipage}[t]{0.92\columnwidth}\raggedright
  Crear una lista que muestre las fases de un entrenamiento
  \end{minipage} & \begin{minipage}[t]{0.08\columnwidth}\raggedright
  1h
  \end{minipage}
  \\\noalign{\medskip}
  \begin{minipage}[t]{0.92\columnwidth}\raggedright
  Agregar funcionalidad para que al seleccionar un plan de entrenamiento
  de la lista, se muestren la lista de los entrenamientos relacionados
  al plan seleccionado.
  \end{minipage} & \begin{minipage}[t]{0.08\columnwidth}\raggedright
  30min
  \end{minipage}
  \\\noalign{\medskip}
  \begin{minipage}[t]{0.92\columnwidth}\raggedright
  Agregar funcionalidad para que al seleccionar un entrenamiento de la
  lista se muestren las fases asociadas a ese entrenamiento
  \end{minipage} & \begin{minipage}[t]{0.08\columnwidth}\raggedright
  30min
  \end{minipage}
  \\\noalign{\medskip}
  \hline
  \end{longtable}
\end{enumerate}


\subsection{Justificación del Sprint Backlog}

Decidimos utilizar este Sprint Backlog puesto que consideramos que realizamos el trabajo requerido por el trabajo práctico, al mismo tiempo que estamos dando valor al usuario. En total, con los estimativos de horas realizados, tenemos 119 horas y 30 minutos,  que considerando los 45 días para el trabajo práctico nos deja y descontando fines de semana nos deja entonces un total de aproximadamente 50 minutos por día para cada uno. Puesto que los cuatro miembros del grupo trabajan (sea en la industria o en la facultad como ayudantes) y cursan, nos pareció un buen estimativo para una cantidad razonable de trabajo. Esto por supuesto no considera la época de parciales, donde no nos fue posible seguir trabajando en el proyecto debido a compromisos de materias.

\section{Detalle de diseño}

\subsection{Diagrama de clases}

\subsubsection{Consideraciones Generales}
La figura \ref{diag_diseno} detalla el diagrama de clases realizado para el diseño orientado a objetos de la aplicación. El mismo fue realizado utilizado la semántica para un lenguaje dinámico e incluye tanto las clases correspondientes al modelo de negocios como las correspondientes a los controladores de las vistas del modelo \texttt{MVC}. Es necesario aclarar que éstas últimas no están completamente especificadas en el diagrama puesto que no nos pareció relevante para el trabajo práctico. Están por una cuestión de claridad de las distintas pantallas y cómo éstas, mediante los controladores, utilizan los objetos del modelo de negocios para lograr el objetivo de la aplicación. En el diagrama los controladores se muestran relacionados con aristas no dirigidas y sin aridad. Estas asociaciones representan ``saltos'' entre vistas y relaciones entre controladores, no la canónica relación de conocimiento que denotan las uniones en los diagramas de clases UML. Esto es porque en iOS esta relación de conocimiento entre controladores se implementa a través del \emph{NavigationController}. Incluirlo en el diseño no sólo acoplaría mucho el diseño a una plataforma particular, sino que elevaría su complejidad notablemente. Creemos que esta pequeña licencia en la semántica de UML nos permite expresar lo que deseamos (la relación entre los controladores del modelo \texttt{MVC}) manteniendo al diseño independiente de plataforma y acotada su complejidad.

Similarmente, no incluimos las clases correspondientes a las vistas propiamente dichas porque nos pareció que no aportaba en lo más mínimo a la claridad del diseño. Lo mismo aplica para la reproducción de sonido y la visualización en el mapa. 

Del mismo modo, tampoco incluimos clases tales como 
\begin{itemize}
	\item unReal
	\item unEntero
	\item unBoolean
	\item unaFecha
\end{itemize}
porque los consideramos partes de la implementación de \texttt{Cocoa/Objective C} utilizados para la demo en iPhone y que son parte de cualquier librería standard.

Nos parece sin embargo que hemos de realizar una aclaración sobre el \textit{Mapa}. El mismo tiene un constructor que toma una coordenada donde se centra inicialmente. Esto, si bien nos parece razonable aunque, no lo podemos discutir debido a que para mostrar los mapas utilizamos la API de \texttt{Google Maps SDK} que requiere construirlo de esta manera. Si bien el mapa ya es un \textit{wrapper} de este servicio, sigue requiriendo esto para funcionar.

El diagrama esta disponible para ver \textit{online} (dado que el diagrama tiene un tamaño considerable difícil de ver) mediante la URL: \url{https://cacoo.com/diagrams/3Bk8unCXdWQ8DFU0#}

\subsubsection{Clase \texttt{ServicioDeEstado}}
El objeto \texttt{ServicioDeEstado} tiene como principal función la de informar a los distintos componentes del programa el estado actual del teléfono en movimiento. A intervalos de tiempo regulares, configurados en última instancia por el consumo seleccionado de la batería, un objeto \texttt{ServicioDeEstado} le manda a todos los objetos a los que se hayan suscripto a él un paquete de actualización que incluye la velocidad a la que se está moviendo el dispositivo y su posición actual.

Esto corresponde al patrón \texttt{observer}. El propósito de esta decisión de diseño es que identificamos dos responsabilidades que no deseábamos acoplar: la obtención de una secuencia regular de mediciones y la utilización de dichas mediciones para diversos propósitos. En particular identificamos que dichas mediciones podían ser usadas para varios objetivos, tales como:
\begin{itemize}
	\item Mostrar en tiempo real en el mapa la posición actual.
	\item Obtener estadísticas sobre el entrenamiento.
	\item Mostrar la velocidad actual de desplazamiento.
	\item Determinar el pasaje de fases dentro de un entrenamiento.
	\item Mostrar la velocidad a la que debería desplazarse el corredor para lograr la distancia implícita pautada en la fase.
\end{itemize}
Dado que esto corresponde con el patrón observer, decidimos utilizarlo para resolver el problema \cite{Gamma}.

Se pueden observar ejemplos de uso de el \texttt{ServicioDeEstados} en los diagramas de secuencia \ref{diag_seq_crearServicioDeEstado} y \ref{diag_seq_actualizacionSeguimiento}. 

\subsubsection{Clase \texttt{Seguimiento}}
Un objeto \texttt{Seguimineto} tiene como funcionalidad principal la de mantener el estado de la fase actual del entrenamiento en curso. Para eso se apoya en otro objeto, el \texttt{IteradorEntrenamiento} que, como su nombre lo indica, es un iterador de un objeto \texttt{Entrenamiento}, que mantiene cuál es la fase actual. 

El patrón correspondiente a esto es el \texttt{iterator} (\cite{Gamma}). Decdimos utilizarlo puesto que consideramos que no es necesario que el seguimiento conozca al entrenamiento de forma total, sino que simplemente pueda iterar sus fases como corresponda. Esto reduce el acoplamiento entre el \texttt{Seguimiento} y el \texttt{Entrenamiento}, disminuyendo el impacto de futuras modificaciones a alguna de las antedichas clases.

El \texttt{Seguimiento} encapsula, por ejemplo, el cálculo de velocidad tentativa y además se ocupa de realizar las notificaciones auditivas funcionales a informar al corredor los cambios de velocidad necesarios y el cambio de fase dentro de un entrenamiento.

\subsubsection{Clase \texttt{Configurador}}
El \texttt{Configurador} encapsula la funcionalidad de modificar el comportamiento de la aplicación ante distintas configuraciones de nivel de batería. Al encapsular esta idea, se hace necesario que pueda influir sobre los objetos que realizan mediciones, para personalizar un \emph{trade-off} entre precisión y consumo. 

Para lograr esto, se utilizó el patrón \texttt{Factory} (\cite{Gamma}) para generar distintos objetos \texttt{Posicionador}, \texttt{Timer}, \texttt{Notificador} y \texttt{MedidorDeVelocidad}. Nuestro adaptación de este patrón al problema particular del TP fue realizar tres subclases de configurador, una para cada tipo de batería. Esto provee extensibilidad, ya que si se quisiera agregar otra configuración de batería, sólo sería necesario crear una nueva clase que implemente la interfaz definida por \texttt{Configurador}. 

Entendemos que esto no es la mejor forma de resolverlo, ya que dificulta el cambio dinámico de estado de batería, puesto que hay que destruir el configurador y crearlo de nuevo. Podría mejorarse implementando un patrón \texttt{Decorator} (sea recursivo o iterando sobre una lista de métodos a aplicar). Sin embargo, dado que el usuario no puede cambiar la configuración de la batería mientras corre y que sea de la forma actual o con \texttt{Decorator} hay que destruir y crear un nuevo \texttt{Posicionador}, \texttt{Timer}, etc. decidimos dejarlo con el diseño actual, que es mucho más sencillo y, por ende, fácil de entender.

\subsubsection{Estadísticas}
La representación de las estadísticas en el diseño presentado tiene una complejidad no menor en pos de proveer extensibilidad y usabilidad. 

En primer lugar se define una interfaz \texttt{Estadistica} que define que toda estadística debe implementar un método que reporte su valor. Además, las clase \texttt{Estadistica} hereda de \texttt{SuscriptorDeSerticioDeEstado}. Esto implica que todas las estadísticas deben implementar el método \emph{actualizar (unEstado)} para poder suscribirse a las notificaciones del \texttt{ServicioDeEstados} y actualizarse pertinentemente (consideramos que es responsabilidad de cada estadística saber cómo actualizarse cuando le llega un nuevo paquete de \texttt{Estado}). 

Por ejemplo, la estadística de velocidad máxima reporta como resultado final el máximo de las velocidades a las que se desplazó el corredor y su actualización cuando llega un nuevo paquete de \texttt{Estado} compara la nueva velocidad con la almacenada y se queda con la mayor. 

La interfaz \texttt{Estadística} hereda dos subclases: \texttt{EstadisticaGeneral} y \texttt{EstadisticaDeEntrenamiento}, que, como sus nombres los indican, representan las estadísticas generales y las asociadas a un entrenamiento particular respectivamente. El motivo de esta separación es que podría haber estadísticas que no tengan sentido considerarlas para un entrenamiento particular (por ejemplo, si en el futuro se implementara almacenar información climática y nutricional a los entrenamientos, uno podría querer una estadística del tipo: ``Velocidad promedio los martes lluviosos en los que entrené después de haber comido pastaflora''). No pondemos estadísticas para las fases porque nos pareció que en este punto no tenemos suficiente conocimiento del dominio como para determinar si las estadísticas por fase eran relevantes.

Para agregar un nuevo tipo de estadística, sólo es necesario determinar su tipo (general o de entrenamiento) y crear una nueva clase que implemente los métodos \emph{actualizar(unEstado)} y \emph{resultado() : unReal}.

Para mostrar las estadísticas se implementa la clase \texttt{EstadisticaMostrable}, que se construye con un objeto de la clase \texttt{Estadistica} y un nombre y tiene como único método ``mostrar()'', que devuelve un objeto apto para imprimir por pantalla en la plataforma que se esté implementando el modelo (en nuestro caso actual, un NSString). Se consideró hacer más extensible este diseño modificando la forma en la que se muestra una estadística utilizando un patrón \texttt{visitor}. Esto permitiría mucha más flexibilidad a la hora de mostrar o exportar una estadística. Esta idea no fue descartada por completo, pero la consideramos demasiado compleja para la primer iteración del \texttt{Scrum}. Es uno de los cambios a realizar en siguientes iteraciones.

\subsubsection{Fases, Entrenamiento, Planes y Planificador}
Para abstraer la lógica de planes del resto del sistema, decidimos utilizar una interfaz que denominamos \texttt{Planificador} que se ocupa de devolver un \texttt{Plan}. Un \texttt{Plan} consiste en una secuencia de \texttt{Entrenamientos} y un plazo que es el que le determina su validez (por ejemplo, cuando ya no logramos el objetivo en el plazo planteado, no tiene sentido mantener este plan en la lista de los realizables para el usuario). Cada \texttt{Entrenamiento} consiste en una serie de \texttt{Fases}. Cada \texttt{Fase} contiene una velocidad mínima y máxima que forman el rango aceptable para correr, y una duración. De esta manera, no tenemos acoplamiento entre la lógica que produce un plan y el resto del sistema, porque solo se conocen en \texttt{Plan} que es algo desconectado de la lógica de negocio que lo trajo al mundo.

Dado que desconocemos del dominio lo suficiente como para poder realizar una implementación fidedigna de acuerdo a parámetros médicos y deportivos, decidimos tratar de proveer un dise\~no extensible pero dentro de las limitaciones indicadas en el trabajo práctico. Para ello incluimos en el dise\~no un planificador básico.

El \texttt{Planificador Básico} emplea los datos indicados en el trabajo práctico para obtener el plan: Las características de la persona (peso y altura), un objetivo(que consiste en indicar cuantos kilómetros se desea correr, y opcionalmente en cuanto tiempo desea poder correrlo), su disponibilidad (en horas por semana), un plazo estipulado (como una fecha límite) y su estado actual (cuantos kilómetros fue lo máximo que corrió y en cuanto tiempo lo hizo).

Decidimos restringirnos a este diseño puesto que no nos pareció que tuviéramos el suficiente conocimiento como para realizar más detalles, y no tomamos una postura de pensar más en extensibilidad puesto que el trabajo práctico no indica que pueda haber cambios en este área.

Algunos de estos factores nos pareció pertinente incluirlos como parte de un \texttt{PerfilCorredor} y tomarlo como parte del \texttt{Corredor} (que además de esto incluye todos sus planes). Los demás factores son específicos a cuando se crea un plan, por lo tanto son conocidos y creados por el \texttt{NuevoPlanControlador} que es el que toma estos datos al preguntarle al usuario cuando este desea crear un nuevo plan.

El objetivo puede o no tener un tiempo asignado de corrida. Para esto, delegamos esta responsabilidad en un objeto \texttt{TiempoParaDistancia}. Las subclases de este implementan distinto comportamiento. Dado el conocimiento que tenemos del dominio, decidimos que lo que nos interesaba de este valor es compararlo con otro, para saber si es menor o igual que otro. En este caso, un limite sin restricciones es mayor a cualquiera y nunca es menor a otro. De esta manera se puede implementar cualquier tipo de algoritmo en el que sea necesario utilizar rangos de tiempos (sea cual sea).

\subsection{Diagramas de secuencia}


Para clarificar algunas de las partes del código que consideramos más difíciles de plasmar en un diagrama estático, incluimos diagramas de secuencia para diversas partes del trabajo práctico. Estas partes corresponden a diversas situaciones donde mostramos como podría ser una implementación de la lógica secuencial que se desarrolla.

\subsubsection{Inicio de un seguimiento}\label{diag_inicioSeguimiento}
% \begin{center}
% 	\includegraphics[scale=0.5]{images/diagSeq.png}
% \end{center}
Este diagrama de secuencia muestra el proceso de inicio de seguimiento. A grandes razgos, el controlador de fase crea un servicio de estado (que será le encargado de mandar las actualizaciones de estado a otros componentes). Esta interacción es un poco compleja para ponerla en el mismo diagrama, así que se la separó en el diagrama ``Crear un servicio de estados'', que se puede ver en la sección \ref{diag_crearUnServicioDeEstado}. 


Luego se procede a suscribir a todos los objetos que deben recibir actualizaciones de estado al controlador (creándolos si no existían), para lo que se le manda repetidas veces el mensaje \emph{suscribir (unObjeto)} al \texttt{ServicioDeEstado}. 


Finalmente, se inicializa el servicio de estados, enviándole el mensaje \emph{iniciarLectura ()}, lo que provoca que éste (que implementa la interfaz \texttt{SuscriptorDeTimer}), le envíe un mensaje \emph{suscribir (self)} al timer. De esa forma queda suscripto el notificador de estados al timer y éste le enviará notificaciones a intervalos regulares.


Suponiendo que el usuario ya realizó la configuración inicial y tiene algún plan con entrenamientos seteados, el camino por el que debe navegar para llegar a este diagrama de secuencia es:
\begin{itemize}
	\item Seleccionar uno de sus planes de la lista de planes disponibles.
	\item Seleccionar un entrenamiento de la lista de entrenamientos del plan.
	\item En esa pantalla (donde se listan las fases correspondientes al entrenamiento) se presiona ``comenzar''.
\end{itemize}

\subsubsection{Actualización dentro de un seguimiento}
% \begin{center}
% 	\includegraphics[scale=0.5]{images/diagSeq.png}
% \end{center}
Este diagrama muestra la secuencia de mensajes intercambiados a la hora de realizar una actualiación dentro de una fase, mientras el usuario está corriendo. Los pasos para llegar a este mensaje son los mismos que el anterior, esperando el tiempo correspondiente a la frecuencia de actualización que se haya realizado.


En primer lugar, el objeto \texttt{Timer} recibe un mensaje \emph{tic ()}. Este mensaje está en el diseño para que éste quede agnóstico a la implementación de la plataforma sobre la que se está trabajando. En el caso particualr de \emph{iOS}, este mensaje no existe sino que se le especifica al sistema operativo que un método en particular (en este caso el \emph{actualizar}) debe ser invocado regularmente cada cierto tiempo. Cuando el \texttt{Timer} recibe ese mensaje, le informa al \texttt{ServicioDeEstado} (pues es el único que está suscripto al \texttt{Timer}) enviándole el mensaje \emph{actualizar ()}.


Al recibir este mensaje, el \texttt{ServicioDeEstado} le solicita a sus colaboradores internos la velocidad y posición actual del teléfono (mediante los mensajes \emph{velocidadActual ()} y \emph{posición ()} respectivamente) y crea un nuevo \texttt{Estado} con esa información.


Finalmente le envía a cada uno de sus sucriptores el mensaje \emph{actualizar (unEstado)}, para que estos puedan realizar sus respectivas acciones con la nueva información del estado. 

\subsubsection{Crear un servicio de estados}\label{diag_crearUnServicioDeEstado}
% \begin{center}
% 	\includegraphics[scale=0.5]{images/diagSeq.png}
% \end{center}
En este diagrama se puede observar el proceso de creación de un objeto \texttt{ServicioDeEstado}, que se crea al comenzar el entrenamiento, como se puede ver en la sección \ref{diag_inicioSeguimiento}. Para crear el \texttt{ServicioDeEstado}
 se pasa como parámetro un \texttt{Configurador}, del que se obtienen y setean como colaboradores internos el \texttt{Timer}, el \texttt{Posicionador} y el \texttt{MedidorVelocidad}. El configurador crea dinámicamente estos elementos a medida que se le son solicitados y los devuelve.

 Una vez que todos los elementos están seteados, se devuelve el recién creado \texttt{ServicioDeEstado}, sin que éste se suscriba al \texttt{Timer}.  


Para la creación del servicio de estado se nos planteó una disyuntiva:
\begin{itemize}
	\item Una opción era no guardar el \texttt{Timer} como colaborador interno del \texttt{ServicioDeEstado}, sino que simplemente cuando a la hora de crear el \texttt{ServicioDeEstado}, se pida al \texttt{Configurador} el \texttt{Timer} y el \texttt{ServicioDeEstado} se suscriba en ese momento. Esta alternativa (no elegida) tenía la ventaja de que evitaba almacenar como colaborador interno al \texttt{Timer} y nos ahorraba el método \emph{iniciarLectura ()}. Sin embargo tiene un defecto: si la frecuencia de actualización era demasiado rápida, podía llegar a llamarse al mensaje \emph{actualizar ()} del servicio de estado (por un \emph{tic ()} del sistema operativo) antes de que todos sus colaboradores internos estuvieran registrados, potencialmente perdiendo información de los primeros estados.
	\item La alternativa que se nos ocurrió para esto es que se almacene el \texttt{Timer} como colaborador interno del \texttt{ServicioDeEstado} y que una vez que el controlador termine de suscribir a quienes corresponda, llame a un método del \texttt{ServicioDeEstado}(\emph{iniciarLectura ()}) que se ocupe de suscribirlo al \texttt{Timer}. Elegimos esta alternativa porque pensamos que en un futuro puede existir algún objeto (por ejemplo una nueva \texttt{Estadistica}) para el que sea importante no perder los estados iniciales. 
\end{itemize}


\subsubsection{Finalización de un seguimiento}
\subsubsection{Listado de estadísticas del usuario}
\subsubsection{Notificar que la velocidad del usuario es demasiado baja dentro de un seguimiento}
\subsubsection{Crear un plán básico}


\section{Retrospectiva y conclusiones} 

En este sprint, consideramos positivo que logramos obtener valor para el \textit{stakeholder} en la forma de una demo, y logramos implementar funcionalidad que
terminaría en un proyecto final. También consideramos positivo que ganamos experiencia en como realizar SCRUM y por lo tanto nos vemos mejor preparados para una posible próxima iteración, con respecto al estimar horas de trabajo y asignar las tareas.

Tuvimos problemas con la cantidad de horas y tareas a asignar al principio del Sprint. Nos resultó difícil extraer las \textit{stories} a realizar del enunciado del trabajo práctico, y esto nos alenteció especialmente en las primeras semanas. Al mismo tiempo tuvimos dificultades para mantener un ritmo más estable de trabajo debido a responsabilidades como por ejemplo los parciales de las materias y las responsabilidades laborales. Esto impactó en que las semanas siguientes, en especial la semana después del parcial de Ingeniería II, estuvieran mucho más cargadas en horas de trabajo. Por suerte esto lo tuvimos en cuenta al momento de armar el Sprint y no nos vimos excesivamente sobrecargados hacia el final del mismo y pudimos llegar a los objetivos claves planteados.

Nos costó también el poder equilibrar la necesidad de un diseño orientado a objetos completo de la aplicación con una metodología ágil como lo es SCRUM, que se supone que trata de evitar los problemas conocidos de \textit{Big Design Up Front}. Esto nos parece impacta en el diseño de manera negativa, puesto que solo un pedazo del trabajo esta implementado con lo cual nos parece que no tenemos el \textit{feedback} de realizar la implementación que consideramos es de gran ayuda al momento de validar el diseño.

Otro factor que impacto negativamente fue que las tareas de programación fueron asignadas en su casi totalidad a una persona del grupo, mientras el resto de las tareas fueron asignadas a todo el resto. La conclusión que obtuvimos es que esto es desfavorable para ambas partes, puesto que requiere de una coordinación entre diseño y código que no sería tan necesaria si todos los miembros del grupo están involucrados en ambas tareas. Esto fue complejizado por la tecnología elegida (iOS) que requiere de un \textit{setup} más complicado que otras, especialmente en tema herramientas.

En subsecciones anteriores señalamos algunos detalles del diseño que podrían mejorarse en próximos Sprints. Consideramos por ejemplo que sería positivo lograr mejorarel diseño del subsistema que se ocupa del almacenado de estadísticas, para que por ejemplo incluya estadísticas sobre las fases. Al mismo tiempo, se podría lograr una mejor separación entre el Seguimiento y el controlador de seguimiento en matería de roles. En estos puntos donde consideramos que no teníamos suficiente conocimiento de dominio como para justificar la elección de una alternativa de diseño por sobre otra, es donde hay más oportunidad de expansión al menos en el tema diseño con un poco más de experiencia en el tema.

\addcontentsline{toc}{section}{Referencias}
\begin{thebibliography}{8}
\raggedright

\bibitem{Gamma}
	E.~Gamma, R.~Helm, R.~Johnson, and J.~Vlissides.
	\newblock {\em Design Patterns: Elements of Reusable Object-Oriented Software}.
	\newblock Professional Computing. Addison-Wesley, 1995.

\end{thebibliography}

\end{document}
