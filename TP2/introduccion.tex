\section{Introducción}

El presente informe constituye la planificación de \textbf{Corré por tu
Vida} (CV), que dada la popularidad de su previa iteración ha visto su
potencial aumentado de manera importante. El nuevo tamaño del proyecto
motiva entonces el uso de la metodología RUP (\emph{Rational Unified
Process}).

\subsection{Caso de negocio}

El caso de negocio se modificó respecto al trabajo anterior
principalmente en la escala de dispositivos y usos para el mismo. La
aplicación debe integrar una gran variedad de dispositivos y deberá
proveer los datos a redes sociales, a consumidores externos de datos
físicos, etc. Consideramos que esto es lo que motiva el importante
cambio de escala del mismo.

A continuación incluimos los \emph{drivers} del proyecto, las
restricciones y los grados de libertad del mismo.

\subsubsection{Drivers del proyecto}

Consideramos que se deben lograr los siguientes objetivos:

\begin{itemize}
\itemsep1pt\parskip0pt\parsep0pt
\item
  La aplicación debe funcionar con anteojos de realidad aumentada,
  considerando que los mismos pueden ser fabricados por una amplia
  variedad de vendedores. También debe incluir dispositivos menos
  novedosos como mp3, relojes de pulsera, etc. ajustando la
  funcionalidad a lo que corresponda.
\item
  Los datos de los entrenamientos deben estar disponibles para ser
  compartidos en redes sociales, y que se pueda interactuar con los
  corredores enviandoles mensajes mediante estas redes. También debe
  incluirse esta noción de ``amigos'' en la aplicación mediante el uso
  de posiciones en un mapa.
\item
  Se mostrarán, durante una carrera, todo tipo de datos relacionados con
  la misma, \emph{on demand} mediante voz o indicaciones al dispositivo.
\item
  Se expanden las posibilidades de entrenamiento clásico con nuevas
  ``competencias virtuales'' contra los registros de otros corredores en
  la zona actual.
\item
  Los corredores profesionales deben poder recibir actualizaciones de
  sus entrenadores durante los entrenamientos, dandoles prioridad a
  estos avisos.
\item
  La información de los corredores debe ser recolectada en tiempo real
  para que la Secretaría de Deportes pueda procesar los datos físicos en
  tiempo real.
\item
  Esta información también debe ser disponible mediante una API
  (\emph{Application Programming Interface}) para consumidores de los
  datos de la aplicación.
\end{itemize}

\subsubsection{Restricciones}

El proyecto tiene las siguientes restricciones

\begin{itemize}
\itemsep1pt\parskip0pt\parsep0pt
\item
  Se deben utilizar estándares abiertos para minimizar la dependencia
  tecnológica. En particular debe seguirse el estándar OpenSocial en la
  medida de lo posible.
\item
  La aplicación debe adaptarse a las normativas de accesibilidad para
  personas con discapacidades, con lo que las maneras de mostrar
  información y generar notificaciones debe ser adecuadas a estas
  necesidades.
\item
  Se debe utilizar encriptación para proteger los datos de los
  corredores que se provean por la API. Existe un algoritmo de
  encriptación homomórfica a considerar pero es experimental y puede ser
  necesario ser cambiarlo.
\item
  Los datos enviados a la Secretaría deben ser procesados en una nube
  OpenStack que es local a la misma.
\item
  Para tener ganancias a corto plazo es necesario que se pueda mostrar
  publicidad \emph{targeteada} a los usuarios de acuerdo al contexto.
\end{itemize}

\subsubsection{Grados de libertad}

\begin{itemize}
\itemsep1pt\parskip0pt\parsep0pt
\item
  Si bien se indicó que no era posible contratar gente, no esta indicada
  la composición del grupo como lo decidimos nosotros. Por lo tanto
  consideramos una composición de un grupo de 4 personas.
\item
  La duración de las fases de RUP no esta especificada. Por lo tanto
  asumimos que las mismas son de 15 días y que tomando un día de 8 horas
  laborales y considerando 4 personas en el grupo eso nos da 480 horas
  hombre de trabajo para la iteración.
\item
  No se especifica si el \emph{reconocedor de voz} debe ser desarrollado
  \emph{in house} o no, por lo tanto decidimos que vamos a utilizar un
  servicio externo (por ejemplo \emph{Google Voice API}). El mismo es
  Open Source y por lo tanto no conflictua con ninguna restricción del
  proyecto.
\end{itemize}
