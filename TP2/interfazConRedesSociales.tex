\subsection{Interfaz con redes sociales}
\subsubsection{Diagrama}

\ig{l}{0.9}{interfazRedesSociales.png}{Diagrama de componentes y
conectores de la interfaz con redes sociales.}

\subsubsection{Detalle}

En el diagrama anterior incluimos un detalle de componentes y 
conectores para la interfaz entre el sistema y las redes sociales.

En particular, utilizamos un sistema basado en OpenSocial y redes
sociales compatibles con la especificación de la API. 

Con este proposito usamos también OAuth para autenticación de los usuarios (que se determina al modelo de configuración). Estos repositorios son almacenados por la aplicación de manera de poder validar el 
usuario en redes sociales de manera transparente. 

Los repositorios con certificados son actualizados cada cierto
tiempo de manera, ya que los mismos expiran.

La interfaz OpenSocial nos permite por defecto cualquier red social
que sea compatible con la misma. 

Los principales sistemas consumidores de las redes sociales son
parte del módulo de interfaz contextual, ya que queremos obtener
los comentarios nuevos para mostrarle al corredor y además
queremos obtener sus amigos para mostrar sus posiciones en el mapa.
