%BEGIN COPYPASTE EL INFORME DEL INFO
\documentclass[10pt, a4paper,english,spanish]{article}
\usepackage{subfig}

\parindent=20pt
\parskip=8pt
\usepackage[width=15.5cm, left=3cm, top=2.5cm, height= 24.5cm]{geometry}

% \usepackage{ccfonts,eulervm} 
% \usepackage[T1]{fontenc}

\usepackage{longtable}
\usepackage{ccfonts,eulervm} 
\usepackage[T1]{fontenc}
% \usepackage{ccfonts,eulervm} 
% \usepackage[T1]{fontenc}
\usepackage{longtable}
\usepackage{epigraph}
\usepackage{amsmath}
\usepackage{amsfonts}
\usepackage{amssymb}
\usepackage[activeacute, spanish,english]{babel}
\usepackage{cancel}
\usepackage[utf8]{inputenc}
\usepackage{algorithm}
%\usepackage{algpseudocode}
\usepackage{afterpage}
\usepackage{caratula}
\usepackage{url}
\usepackage{fancyhdr}
\usepackage{listings}
\usepackage{ulem}
\usepackage{dashrule}
\usepackage{pdflscape}
\usepackage{pgf}
\usepackage{tikz}
\usetikzlibrary{arrows,automata}


\floatname{algorithm}{Algoritmo}

\newtheorem{theorem}{Teorema}[section]
\newtheorem{lemma}[theorem]{Lema}
\newtheorem{proposition}[theorem]{Proposici\'on}
\newtheorem{corollary}[theorem]{Corolario}

\newcommand{\Var}{\textbf{var }}
\newcommand{\True}{\textbf{true }}
\newcommand{\False}{\textbf{false }}
\newcommand{\Break}{\textbf{break }}
\newcommand{\Continue}{\textbf{continue }}
\newcommand{\Param}{\textbf{param }}
% \newcommand{\ig}[3]{
% 	\begin{landscape}
% 		\begin{figure}[c]
% 			\label{diag_diseno}
% 			\includegraphics[scale=#1]{images/#2}
% 			\caption{#3}
% 		\end{figure} 
% 	\end{landscape}
% 	\newpage
% }

\newcommand{\ig}[4]{
	\ifthenelse{\equal{#1}{l}}{\begin{landscape}
		\begin{figure}[c]
			\label{diag_diseno}
			\includegraphics[scale=#2]{images/#3}
			\caption{#4}
		\end{figure} 
	\end{landscape}
	\newpage}{
	\ifthenelse{\equal{#1}{p}}{
		\begin{figure}[c]
			\label{diag_diseno}
			\includegraphics[scale=#2]{images/#3}
			\caption{#4}
		\end{figure} 
	\newpage}
	}	
	
}

% \parindent 0em
%\algrenewcommand{\algorithmiccomment}[1]{//\textit{#1} }

\renewcommand{\emph}[1]{\textit{#1}}
\pagestyle{fancy}
\thispagestyle{fancy}
\addtolength{\headheight}{1pt}
\lhead{IS2 - TP2}
\rhead{Grupo 8}
\cfoot{\thepage}
\renewcommand{\footrulewidth}{0.4pt}
\newcommand{\hblacksquare}{\hfill \blacksquare}
%FIN COPYPASTE EL INFORME DEL INFO
\begin{document}

\materia{Ingenier\'ia del Software 2}
\submateria{Segundo Cuatrim\'estre de 2013}
\titulo{Trabajo Pr\'actico 2}
\subtitulo{Programming \textit{in the large} - Tutor: Nicol\'as Rinaldi}
\grupo{Grupo 8}
\integrante{Juli\'an Sackmann}{540/09}{jsackmann@gmail.com}
\integrante{Juan Pablo Darago}{272/10}{jpdarago@gmail.com}
\integrante{Vanesa Stricker}{159/09}{vanesastricker@gmail.com}
% \integrante{Mat\'ias Barbeito}{179/08}{matiasbarbeito@gmail.com}

\maketitle
\pagebreak

\tableofcontents
\pagebreak

\section{Introducción}

El presente informe constituye la planificación de \textbf{Corré por tu
Vida} (CV), que dada la popularidad de su previa iteración ha visto su
potencial aumentado de manera importante. El nuevo tamaño del proyecto
motiva entonces el uso de la metodología RUP (\emph{Rational Unified
Process}).

\subsection{Caso de negocio}

El caso de negocio se modificó respecto al trabajo anterior
principalmente en la escala de dispositivos y usos para el mismo. La
aplicación debe integrar una gran variedad de dispositivos y deberá
proveer los datos a redes sociales, a consumidores externos de datos
físicos, etc. Consideramos que esto es lo que motiva el importante
cambio de escala del mismo.

A continuación incluimos los \emph{drivers} del proyecto, las
restricciones y los grados de libertad del mismo.

\subsubsection{Drivers del proyecto}

Consideramos que se deben lograr los siguientes objetivos:

\begin{itemize}
\itemsep1pt\parskip0pt\parsep0pt
\item
  La aplicación debe funcionar con anteojos de realidad aumentada,
  considerando que los mismos pueden ser fabricados por una amplia
  variedad de vendedores. También debe incluir dispositivos menos
  novedosos como mp3, relojes de pulsera, etc. ajustando la
  funcionalidad a lo que corresponda.
\item
  Los datos de los entrenamientos deben estar disponibles para ser
  compartidos en redes sociales, y que se pueda interactuar con los
  corredores enviandoles mensajes mediante estas redes. También debe
  incluirse esta noción de ``amigos'' en la aplicación mediante el uso
  de posiciones en un mapa.
\item
  Se mostrarán, durante una carrera, todo tipo de datos relacionados con
  la misma, \emph{on demand} mediante voz o indicaciones al dispositivo.
\item
  Se expanden las posibilidades de entrenamiento clásico con nuevas
  ``competencias virtuales'' contra los registros de otros corredores en
  la zona actual.
\item
  Los corredores profesionales deben poder recibir actualizaciones de
  sus entrenadores durante los entrenamientos, dandoles prioridad a
  estos avisos.
\item
  La información de los corredores debe ser recolectada en tiempo real
  para que la Secretaría de Deportes pueda procesar los datos físicos en
  tiempo real.
\item
  Esta información también debe ser disponible mediante una API
  (\emph{Application Programming Interface}) para consumidores de los
  datos de la aplicación.
\end{itemize}

\subsubsection{Restricciones}

El proyecto tiene las siguientes restricciones

\begin{itemize}
\itemsep1pt\parskip0pt\parsep0pt
\item
  Se deben utilizar estándares abiertos para minimizar la dependencia
  tecnológica. En particular debe seguirse el estándar OpenSocial en la
  medida de lo posible.
\item
  La aplicación debe adaptarse a las normativas de accesibilidad para
  personas con discapacidades, con lo que las maneras de mostrar
  información y generar notificaciones debe ser adecuadas a estas
  necesidades.
\item
  Se debe utilizar encriptación para proteger los datos de los
  corredores que se provean por la API. Existe un algoritmo de
  encriptación homomórfica a considerar pero es experimental y puede ser
  necesario ser cambiarlo.
\item
  Los datos enviados a la Secretaría deben ser procesados en una nube
  OpenStack que es local a la misma.
\item
  Para tener ganancias a corto plazo es necesario que se pueda mostrar
  publicidad \emph{targeteada} a los usuarios de acuerdo al contexto.
\end{itemize}

\subsubsection{Grados de libertad}

\begin{itemize}
\itemsep1pt\parskip0pt\parsep0pt
\item
  Si bien se indicó que no era posible contratar gente, no esta indicada
  la composición del grupo como lo decidimos nosotros. Por lo tanto
  consideramos una composición de un grupo de 4 personas.
\item
  La duración de las fases de RUP no esta especificada. Por lo tanto
  asumimos que las mismas son de 15 días y que tomando un día de 8 horas
  laborales y considerando 4 personas en el grupo eso nos da 480 horas
  hombre de trabajo para la iteración.
\item
  No se especifica si el \emph{reconocedor de voz} debe ser desarrollado
  \emph{in house} o no, por lo tanto decidimos que vamos a utilizar un
  servicio externo (por ejemplo \emph{Google Voice API}). El mismo es
  Open Source y por lo tanto no conflictua con ninguna restricción del
  proyecto.
\end{itemize}


\section{Casos de Uso}

A continuación detallamos los casos de uso que consideramos para el proyecto.

\subsection{Publicidad}
En este diagrama se muestran los casos de uso que conciernen a la publicidad dentro del sistema.

\ig{0.8}{CDU_Publicidad.pdf}{Casos de uso de publicidad}

\begin{itemize}
	\item \textbf{Logueándose al sistema}: un sujeto ya registrado inicia sesión desde su dispositivo para comenzar a utilizar el sistema. Para eso debe ingresar su nombre de usuario y password.
	\item \textbf{Añadiendo/modificando publicidad al sistema}: una entidad publicitaria ya registrada y logueada añade o modifica nueva publicidad al sistema.
	\item \textbf{Registrando/modificando entidad publicitaria}: luego de que se sigan todos los procesos burocráticos pertinentes, la secretaría de deportes registra en el sistema a una entidad publicitaria para que agregue publicidad al sistema.
	\item \textbf{Consumiendo publicidad en carrera}: el corredor consume publicidad contextual.
	\item \textbf{Emitiendo publicidad contextual}: el dispositivo, de acuerdo a su ubicación emite publicidad contextual. Esta puede presentarse en forma auditiva o visual, de acuerdo a las características del dispositivo.
	\item \textbf{Iniciando un entrenamiento}: un corredor previamente autenticado inicia una sesión de entrenamiento. Esta puede o no ser virtual.
	% \item \textbf{Finalizando un entrenamiento} ?????? HACEMOS ESTO ?????? 
\end{itemize}

\subsection{Registración y autenticación}
En este diagrama se muestran los casos de uso que conciernen a la registración y autenticación (llamado ``loguearse'') de los distintos sujetos ante el sistema. 

\ig{0.8}{CDU_RegistrandoLogueando.pdf}{Casos de uso de registración y autenticación}

\begin{itemize}
	\item \textbf{Registrándose al sistema}: un sujeto aún no registrado puede utilizar su dispositivo para registrarse en el sistema con el fin de comenzar a utilizarlo. Se puede registrar como corredor o como entrenador, ofreciendo diferentes funciones a cada uno. Al momento de registrarse como corredor, el sistema ofrece ``linkear'' la cuenta recién creada con las distintas redes sociales, que permitirá luego publicar datos y obtener comentarios motivacionales de las mismas.
\end{itemize}


\subsection{Redes Sociales}
En este diagrama se muestran principalmente interacciones entre un corredor y sus distintos amigos a través de las redes sociales.

\ig{0.8}{CDU_RedesSociales.pdf}{Casos de uso de redes sociales}

\begin{itemize}
	\item \textbf{Mirando posición propia y de amigos en el mapa}: un usuario ya autenticado puede utilizar el dispositivo para mirar su posición actual y la de sus amigos en un mapa. Para eso, el sistema utiliza los servicios del geolocalizador. Esto puede ocurrir aún cuando el usuario no esté en un entrenamiento.
	\item \textbf{Consumiendo comentario motivacional}: un corredor, esté o no en un entrenamiento, puede acceder a los comentarios motivacionales escritos por sus amigos a través de su dispositivo. Dependiendo de las características del mismo y las circunstancias, este acceso puede darse en forma visual o auditiva.
	\item \textbf{Levantando contenidos motivacionales de las redes sociales}: Si el corredor al registrarse asoció su cuenta con una cuenta de una red social, el sistema periódicamente accede a la antedicha red social para obtener los comentarios motivadores que se dejaron en la misma. 
	\item \textbf{Compartiendo los datos de un entrenamiento en tiempo real}: un corredor, durante un entrenamiento puede seleccionar compartir sus datos de entrenamiento mediante las redes sociales que tenga asociadas a su cuenta directamente desde su dispositivo.
	\item \textbf{Mostrando datos y notificaciones del entrenamiento actual}: un corredor, durante un entrenamiento, accede a sus notificaciones y datos del mismo directamente desde su dispositivo. Dependiendo del dispositivo, este acceso puede presentarse en forma visual o auditiva.
\end{itemize}


\subsection{Estado Físico}
En este diagrma se muestran casos de uso concernientes a las interacciones entre el entrenador y el corredor, y el estado físico de este último.

\ig{0.8}{CDU_EstadoFisico.pdf}{Casos de uso de Estados físicos}

\begin{itemize}
	\item \textbf{Enviando instrucciones a corredor}: un entrenador autenticado envía instrucciones a un corredor utilizando su dispositivo. Esto no necesariamente ocurre cuando el corredor está en un entrenamiento.
	\item \textbf{Consumiendo instrucciones de entrenador}: un corredor profesional autenticado, esté o no en un entrenamiento, puede consumir instrucciones de su entrenador mediante su dispositivo. 
	\item \textbf{Recibiendo notificación de necesidad de descanso}: luego de recibir datos del estado actual del corredor de un medidor biomédico, si es pertinente, el sistema el sistema informa al corredor la necesidad de que se tome un descanso. Esta notificación puede ser visual o auditiva, dependiendo del dispositivo y las circunstancias.
	\item \textbf{Informando estado físico del corredor}: un dispositivo biomédico informa al sistema del estado físico del corredor cuando éste está en un entrenamiento.
\end{itemize}


\subsection{API y datos exportados}
En este diagrma se muestran casos de uso concernientes a los datos y servicios que exporta el sistema.

\ig{0.8}{CDU_APIs.pdf}{Casos de uso de APIs y datos exportados}

\begin{itemize}
	\item \textbf{Utilizando la api de datos físicos almacenados}: Se expone una API para que sistemas externos puedan utilizar los datos biomédicos (que son anonimizados previamente y están tomados del corredor con su consentimiento) como deseen, por ejemplo hacer un estudio de calambres por parte de una organización médica. 
	\item \textbf{Enviando datos para procesar a la nube local}: La Secretaría de Deportes dispone de una nube OpenStack a la cual la aplicación le enviará datos de una manera que pueda usar para diversos tipos de procesamiento. 
	\item \textbf{Mostrando datos pedidos mediante voz}: El corredor puede, mientras usa la aplicación y va corriendo, darle indicaciones de voz a la aplicación para que la misma le provea otros datos. El reconocimiento de voz es realizado por fuera de la aplicación, utilizando una API externa (como por ejemplo Google Voice API) y la misma utiliza los datos procesados para responder al pedido de usuario.
\end{itemize}


\subsection{Usos especiales}
En este diagrama se muestran casos de uso concernientes a los usos ``especiales'' que se les da al sistema.

\ig{0.8}{CDU_UsosEspeciales.pdf}{Casos de uso de usos especiales}

\begin{itemize}
	\item \textbf{Corriendo una carrera virtual}: El corredor puede optar por, en vez de realizar un entrenamiento de los preparados por la aplicación, utilizar un entrenamiento de otro corredor en el mismo lugar geográfico, con los datos de tiempo almacenados, y correr contra este para ver si puede vencer su tiempo. Los datos de estas carreras también son mostrados por la aplicación como si fuese un entrenamiento común.
	\item \textbf{Usando aplicación con refuerzo para discapacidad}: Dado que se pretende que los usuarios con dificultades visuales o auditivas puedan utilizar la aplicación de manera satisfactoria, los mismos deben poder indicar como desean recibir las notificaciones, el tamaño de muestra de los mismos (asumiendo dispositivo con soporte visual), etc. para tener una experiencia más agradable de uso.
\end{itemize}


\section{Riesgos}

A continuación incluimos los riesgos más importantes que consideramos a la hora
de definir las iteraciones, junto con sus planes de contigencia y mitigación,
si probabilidad, clasificación, etc.

\begin{landscape}
\begin{table}
\centering
\begin{tabular}{| l | l | l | l | l | l | l | }
     \hline
P & Descripción & P & I & E & Plan & Clasificación \\ \hline

1 & Dado que las pruebas y testeos de la aplicación  & Media & Alto & Alta & [M] Durante el testeo y prueba de la aplicación,  & Técnico\\
  & se realizan a pequeña escala y se desconoce la   & & & &		generar un ambiente de testeo  de ese tipo, es  & \\
  & tecnología a usar, entonces probablemente si más  & & & &		decir, simular el uso de miles de usuarios & \\
  & de 1000 usuarios usan la aplicación 			  & & & & 		simultáneos, ver si se producen fallos y hacer & \\ 
  & simultáneamente el hardware no lo soportaría y se &&&			&lo necesario para evitarlos. & \\ 		  
  & producirían fallos y errores en la aplicación o &&& 			& [C] Tener a disposición un servicio de emergencia & \\
  & se caería todo el sistema. &&&									& que permita un uso reducido de la aplicación,  & \\
  &&&&																&pero que soporte esta cantidad de usuarios, en &\\ 
  &&&&																&caso de que se caiga el sistema general, levantar  &\\   
  &&&&																&el sistema de emergencia hasta tanto se solucionen&\\ 
  &&&&																&los conflictos, fallos y errores en el sistema &\\ 
  &&&&																&principal y se habilita el uso entero de la &\\ 
  &&&&																&aplicación soportando esa cantidad de usuarios. &\\ 
  \hline

2 & Dado que desconocemos la nube de la Secretaría  & Alta & Alto & Alta & [M] Destinar tiempo y recursos del proyecto para & Técnico\\
  &(OpenStack) entonces probablemente no se use   & & & &			obtener toda la información necesaria y gente & \\
  & adecuadamente, se sobrecargue y no sea segura en  & & & &		capacitada en todo lo relacionado con & \\
  & cuanto a la privacidad del usuario, o se caiga	  & & & & 		OpenStack para lograr el uso adecuado de la misma. & \\ 
  & el servicio. &&& 												&[C] Tener a disposición alguna tecnología de respaldo& \\ 		  &&&&																& en caso de que debido al mal uso de OpenStack, & \\
  &&&&																& como ser servidores de backup, y otros mecanismos & \\
  &&&&																& de protección de la información de los usuarios, &\\
  &&&&																& además de los usados sobre OpenStack. &\\ \hline

3 & Dado que se desconocen todas las tecnologías a & Media & Alto & Alta & [M] Destinar parte del presupuesto para el desarrollo & Comercial,\\
  & utilizar entonces probablemente no se haya   & & & &			del proyecto con nuevas tecnologías. & Management \\
  & contemplado en el presupuesto asignado al proyecto  & & & &		[C] Generar nuevas fuentes de ingreso (a través  & \\
  & todo lo necesario para poder trabajar y desarrollar & & & &		de la publicidad, por ejemplo) para obtener el & \\ 
  & con estas tecnologías y el mismo no sea suficiente. &&&			&presupuesto necesario para desarrollar con las & \\ 
  &&&&																& nuevas tecnologías.& \\ \hline
  
     \hline
\end{tabular}
\caption{M: Mitigación , C: Contingencia}
\end{table}
\end{landscape}

\begin{landscape}
\begin{table}
\centering
\begin{tabular}{| l | l | l | l | l | l | l | }
     \hline
P & Descripción & P & I & E & Plan & Clasificación \\ \hline

4 & Dado que el hardware será provisto por otra  & Media & Medio & Media & [M] Mantener contacto con la empresa proveedora del  & Técnico, \\
  & empresa, entonces probablemente no estén  & & & &				hardware para recordar e insistir en la entrega & Comercial\\
  & disponibles a tiempo para realizarse las pruebas  & & & &		del mismo para obtenerlo a tiempo. & \\
  & pertinentes y corroborar que la aplicación		  & & & & 		[C] Destinar tiempo adicional del proyecto para  & \\ 
  & funcione correctamente en los dispositivos. &&&					&realizar los testeos pertinentes para evitar fallos. & \\ \hline

5 & Dado que deben seguirse las especificaciones de & Baja & Medio & Medio & [M] Que la aplicación cuente con una interfaz sencilla  & Técnico\\
  & OpenSocial y optarse por estándares abiertos  & & & &		que se adapte a las especificaciones de OpenSocial,	& \\
  & cuando estén disponibles, si alguno de los  & & & &			que sea fácil de modificar  y mantener en caso de  & \\
  & estándares cambia, entonces probablemente dejaría  & & & &	cambios.		 & \\ 
  & de funcionar la parte “social” de la aplicación y &&&		&[C] Destinar los recursos del proyecto necesarios (tiempo,  & \\ 
  & los usuarios estarían descontentos. &&&					&  	desarrolladores, etc.) para adaptar las& \\ 
  &&&&														&	implementaciones a los cambios en las especificaciones &\\
  &&&&														&	de OpenSocial. &\\	   \hline
  
6 & Dado que la aplicación debe soportar diversos & Baja & Medio & Media & [M] Destinar tiempo del proyecto para lograr & Técnico\\
  & dispositivos entonces probablemente haya   & & & &				integrar el uso de los diversos dispositivos,& \\
  & incompatibilidad tecnológica entre los mismos y  & & & &		dentro de la aplicación. & \\
  & se retrase la fecha de finalización del proyecto.  & & & & 		[C] Contratar expertos en las diversas tecnologías & \\ 
  &&&&																&para desarrollar lo más rápido posibles la & \\ 		  &&&&																& infraestructura necesaria para lograr la & \\
  &&&&																& compatibilidad de todos los dispositivos y & \\
  &&&&																& minimizar el retraso. &\\ \hline
  
7 & Dado que se desconocen todos los dispositivos en  & Alta & Bajo & Medio & [M] Que los encargados de negociar con el cliente & Técnico\\
  & es experimental entonces probablemente no cumpla  & & & &	indaguen y extraigan toda la información necesaria	 & \\
  & los cuales deberá funcionar la aplicación,  & & & &			respecto de los dispositivos que se espera que & \\
  & entonces probablemente exista algún dispositivo	  & & & & 	funcione la aplicación.	 & \\ 
  & para el cual no funcione correctamente alguna &&&				&[C] Que la aplicación cuente con mensajes para el & \\
  & característica de la aplicación. &&&							& usuario de carácter informativo o de error que & \\ 
  &&&&																& de ser necesario informen que la aplicación no &\\ 
  &&&&																& está disponible para ese dispositivo aun y  &\\
  &&&&																& destinar parte del tiempo de mantenimiento del &\\ 
  &&&&																& producto para adaptar la aplicación a los  &\\ 
  &&&&																& dispositivos antes desconocidos. &\\ 
  \hline
     \hline
\end{tabular}
\caption{M: Mitigación , C: Contingencia}
\end{table}
\end{landscape}


\begin{landscape}
\begin{table}
\centering
\begin{tabular}{| l | l | l | l | l | l | l | }
	     \hline
P & Descripción & P & I & E & Plan & Clasificación \\ \hline

8 & Dado que la aplicación debe soportar tecnología & Media & Bajo & Media & [M] Tener recursos humanos dedicados a la investigación& Técnico\\
  & desconocida o aun no desarrollada entonces  & & & &				de las nuevas tecnologías y su uso. & \\
  & probablemente no haya capacidad técnica para  & & & &			[C] Informar a los usuarios para cuales & \\
  & trabajar con los nuevos dispositivos.			  & & & & 		dispositivos funciona la aplicación. Para los que & \\ 
  &&&&																&no funcione y debieran funcionar, informar a los & \\ 		  &&&&																& usuarios que próximamente estará disponible la& \\
  &&&&																& aplicación para esos dispositivos e invertir en & \\
  &&&&																& capacitaciones para los desarrolladores en las &\\
  &&&&																& nuevas tecnologías. &\\  \hline

9 & Dado que se desconocen cuáles son todas las redes & Baja & Bajo & Baja & [M] Desarrollar un adaptador que permita incorporar  & Técnico\\
  & sociales que debe soportar la aplicación entonces  & & & &		fácilmente el uso de una nueva red social. & \\
  & probablemente exista alguna red social que se  & & & &			[C] Destinar tiempo del proyecto para la & \\
  & quiera incorporar que no haya sido contemplada y	 & & & &	implementación necesaria del uso de las redes  & \\ 
  & sea de difícil implementación.&&&								&sociales que pudieran incorporarse. & \\ \hline

10 & Dado que el algoritmo de encriptación homomórfica & Baja & Bajo & Baja & [M] Incorporar más investigadores para trabajar con & Técnico\\
  & es experimental entonces probablemente no cumpla  & & & &		el desarrollo del algoritmo para lograr que éste & \\
  & con los estándares de encriptación o no funcione  & & & &		cumpla con los estándares requeridos y funcione de & \\
  & de la manera esperada. 							  & & & & 		la manera deseada. & \\ 
  &&&&																&[C] Descartar el uso del algoritmo de encriptación & \\ 		  &&&&																& homomórfica y usar un algoritmo conocido que cumpla & \\
  &&&&																& con los estándares de encriptación y funcione como se & \\
  &&&&																& espera. &\\ \hline

     \hline
\end{tabular}
\caption{M: Mitigación , C: Contingencia}
\end{table}
\end{landscape}


\section{Iteraciones del proyecto}

\subsection{Definición de los casos de uso por fase}

Los módulos que se deberían desarrollar en la fase de elaboración son
aquellos que eliminan o disminuyen los mayores riesgos del proyecto
junto con los que definen la arquitectura base del proyecto. Se decidió
que en primer fase de elaboración se desarrollaran los módulos de:

\begin{itemize}
\itemsep1pt\parskip0pt\parsep0pt
\item
  \textbf{Notificaciones y alertas}: Al tener que dar soporte no solo a
  distintos tipos de anteojos de realidad aumentada, sino también a
  otros dispositivos que nos son ahora desconocidos, creemos que este es
  uno de, sino el más, riesgoso de los módulos a implementar. El ser uno
  de los puntos fuertes de interacción con el sistema, y por lo tanto
  uno de los probables puntos de mayor \emph{carga} para el mismo,
  también es un riesgo que consideramos para tomar la decisión de elegir
  este módulo.
\item
  \textbf{Estado físico del corredor}: Consideramos que los módulos
  relacionados con la obtención y procesamiento de los datos médicos del
  corredor tienen algunos de los mayores riesgos asociados, debido a la
  heterogeneidad de datos que se podrían obtener y el desconocimiento de
  las características de procesamiento de la nube OpenStack de la
  secretaría.
\item
  \textbf{Integración con redes sociales}: El riesgo de la
  heterogeneidad de dispositivos, que también contribuye al problema de
  las notificaciones, y la falta de indicación de que redes sociales son
  las que se desea soportar nos llevan a considerar este módulo como
  riesgoso y por lo tanto a ponerlo en la primera iteración de
  elaboración. También hemos de considerar la falta de conocimiento por
  parte de los miembros del equipo del sistema Open Social.
\end{itemize}

Por lo tanto decidimos el siguiente plan de proyecto, indicado con los
casos de uso que se implementarán en cada uno, y luego incluiremos el
detalle de la primera iteración.

\begin{longtable}[c]{@{}llll@{}}
\hline\noalign{\medskip}
\begin{minipage}[b]{0.17\columnwidth}\raggedright
Fase
\end{minipage} & \begin{minipage}[b]{0.12\columnwidth}\raggedright
Número
\end{minipage} & \begin{minipage}[b]{0.58\columnwidth}\raggedright
Casos de uso
\end{minipage} & \begin{minipage}[b]{0.12\columnwidth}\raggedright
Horas
\end{minipage}
\\\noalign{\medskip}
\hline\noalign{\medskip}
\begin{minipage}[t]{0.17\columnwidth}\raggedright
Elaboración
\end{minipage} & \begin{minipage}[t]{0.12\columnwidth}\raggedright
Primera
\end{minipage} & \begin{minipage}[t]{0.58\columnwidth}\raggedright
Mostrando datos y notificaciones del entrenamiento actual.
\end{minipage} & \begin{minipage}[t]{0.12\columnwidth}\raggedright
84h
\end{minipage}
\\\noalign{\medskip}
\begin{minipage}[t]{0.17\columnwidth}\raggedright
\end{minipage} & \begin{minipage}[t]{0.12\columnwidth}\raggedright
\end{minipage} & \begin{minipage}[t]{0.58\columnwidth}\raggedright
Obteniendo posición propia y de amigos en el dispositivo empleado.
\end{minipage} & \begin{minipage}[t]{0.12\columnwidth}\raggedright
72h
\end{minipage}
\\\noalign{\medskip}
\begin{minipage}[t]{0.17\columnwidth}\raggedright
\end{minipage} & \begin{minipage}[t]{0.12\columnwidth}\raggedright
\end{minipage} & \begin{minipage}[t]{0.58\columnwidth}\raggedright
Tomando datos de estado físico del corredor.
\end{minipage} & \begin{minipage}[t]{0.12\columnwidth}\raggedright
84h
\end{minipage}
\\\noalign{\medskip}
\begin{minipage}[t]{0.17\columnwidth}\raggedright
\end{minipage} & \begin{minipage}[t]{0.12\columnwidth}\raggedright
\end{minipage} & \begin{minipage}[t]{0.58\columnwidth}\raggedright
Enviando datos a procesar a la nube local.
\end{minipage} & \begin{minipage}[t]{0.12\columnwidth}\raggedright
52h
\end{minipage}
\\\noalign{\medskip}
\begin{minipage}[t]{0.17\columnwidth}\raggedright
\end{minipage} & \begin{minipage}[t]{0.12\columnwidth}\raggedright
\end{minipage} & \begin{minipage}[t]{0.58\columnwidth}\raggedright
Compartiendo datos en tiempo real de un entrenamiento.
\end{minipage} & \begin{minipage}[t]{0.12\columnwidth}\raggedright
56h
\end{minipage}
\\\noalign{\medskip}
\begin{minipage}[t]{0.17\columnwidth}\raggedright
\end{minipage} & \begin{minipage}[t]{0.12\columnwidth}\raggedright
\end{minipage} & \begin{minipage}[t]{0.58\columnwidth}\raggedright
Logueando al usuario en el dispositivo.
\end{minipage} & \begin{minipage}[t]{0.12\columnwidth}\raggedright
112h
\end{minipage}
\\\noalign{\medskip}
\begin{minipage}[t]{0.17\columnwidth}\raggedright
Elaboración
\end{minipage} & \begin{minipage}[t]{0.12\columnwidth}\raggedright
Segunda
\end{minipage} & \begin{minipage}[t]{0.58\columnwidth}\raggedright
Levantando comentarios motivacionales de redes sociales.
\end{minipage} & \begin{minipage}[t]{0.12\columnwidth}\raggedright
60h
\end{minipage}
\\\noalign{\medskip}
\begin{minipage}[t]{0.17\columnwidth}\raggedright
\end{minipage} & \begin{minipage}[t]{0.12\columnwidth}\raggedright
\end{minipage} & \begin{minipage}[t]{0.58\columnwidth}\raggedright
Mostrando datos pedidos mediante voz.
\end{minipage} & \begin{minipage}[t]{0.12\columnwidth}\raggedright
80h
\end{minipage}
\\\noalign{\medskip}
\begin{minipage}[t]{0.17\columnwidth}\raggedright
\end{minipage} & \begin{minipage}[t]{0.12\columnwidth}\raggedright
\end{minipage} & \begin{minipage}[t]{0.58\columnwidth}\raggedright
Recibiendo notificación de necesidad de descanso.
\end{minipage} & \begin{minipage}[t]{0.12\columnwidth}\raggedright
40h
\end{minipage}
\\\noalign{\medskip}
\begin{minipage}[t]{0.17\columnwidth}\raggedright
\end{minipage} & \begin{minipage}[t]{0.12\columnwidth}\raggedright
\end{minipage} & \begin{minipage}[t]{0.58\columnwidth}\raggedright
Mostrando publicidad contextual.
\end{minipage} & \begin{minipage}[t]{0.12\columnwidth}\raggedright
120h
\end{minipage}
\\\noalign{\medskip}
\begin{minipage}[t]{0.17\columnwidth}\raggedright
Construcción
\end{minipage} & \begin{minipage}[t]{0.12\columnwidth}\raggedright
Tercera
\end{minipage} & \begin{minipage}[t]{0.58\columnwidth}\raggedright
Usando aplicación con refuerzo para discapacidad.
\end{minipage} & \begin{minipage}[t]{0.12\columnwidth}\raggedright
120h
\end{minipage}
\\\noalign{\medskip}
\begin{minipage}[t]{0.17\columnwidth}\raggedright
\end{minipage} & \begin{minipage}[t]{0.12\columnwidth}\raggedright
\end{minipage} & \begin{minipage}[t]{0.58\columnwidth}\raggedright
Iniciando un entrenamiento.
\end{minipage} & \begin{minipage}[t]{0.12\columnwidth}\raggedright
60h
\end{minipage}
\\\noalign{\medskip}
\begin{minipage}[t]{0.17\columnwidth}\raggedright
\end{minipage} & \begin{minipage}[t]{0.12\columnwidth}\raggedright
\end{minipage} & \begin{minipage}[t]{0.58\columnwidth}\raggedright
Recibiendo notificación de descanso.
\end{minipage} & \begin{minipage}[t]{0.12\columnwidth}\raggedright
80h
\end{minipage}
\\\noalign{\medskip}
\begin{minipage}[t]{0.17\columnwidth}\raggedright
\end{minipage} & \begin{minipage}[t]{0.12\columnwidth}\raggedright
\end{minipage} & \begin{minipage}[t]{0.58\columnwidth}\raggedright
Utilizando la API de datos físicos almacenados.
\end{minipage} & \begin{minipage}[t]{0.12\columnwidth}\raggedright
60h
\end{minipage}
\\\noalign{\medskip}
\begin{minipage}[t]{0.17\columnwidth}\raggedright
Construcción
\end{minipage} & \begin{minipage}[t]{0.12\columnwidth}\raggedright
Cuarta
\end{minipage} & \begin{minipage}[t]{0.58\columnwidth}\raggedright
Enviando instrucciones al corredor.
\end{minipage} & \begin{minipage}[t]{0.12\columnwidth}\raggedright
120h
\end{minipage}
\\\noalign{\medskip}
\begin{minipage}[t]{0.17\columnwidth}\raggedright
\end{minipage} & \begin{minipage}[t]{0.12\columnwidth}\raggedright
\end{minipage} & \begin{minipage}[t]{0.58\columnwidth}\raggedright
Registrando al usuario con el dispositivo.
\end{minipage} & \begin{minipage}[t]{0.12\columnwidth}\raggedright
80h
\end{minipage}
\\\noalign{\medskip}
\begin{minipage}[t]{0.17\columnwidth}\raggedright
\end{minipage} & \begin{minipage}[t]{0.12\columnwidth}\raggedright
\end{minipage} & \begin{minipage}[t]{0.58\columnwidth}\raggedright
Consumiendo comentario motivacional.
\end{minipage} & \begin{minipage}[t]{0.12\columnwidth}\raggedright
60h
\end{minipage}
\\\noalign{\medskip}
\begin{minipage}[t]{0.17\columnwidth}\raggedright
\end{minipage} & \begin{minipage}[t]{0.12\columnwidth}\raggedright
\end{minipage} & \begin{minipage}[t]{0.58\columnwidth}\raggedright
Corriendo carrera virtual.
\end{minipage} & \begin{minipage}[t]{0.12\columnwidth}\raggedright
40h
\end{minipage}
\\\noalign{\medskip}
\begin{minipage}[t]{0.17\columnwidth}\raggedright
\end{minipage} & \begin{minipage}[t]{0.12\columnwidth}\raggedright
\end{minipage} & \begin{minipage}[t]{0.58\columnwidth}\raggedright
Añadiendo publicidad al sistema.
\end{minipage} & \begin{minipage}[t]{0.12\columnwidth}\raggedright
120h
\end{minipage}
\\\noalign{\medskip}
\begin{minipage}[t]{0.17\columnwidth}\raggedright
Construcción
\end{minipage} & \begin{minipage}[t]{0.12\columnwidth}\raggedright
Quinta
\end{minipage} & \begin{minipage}[t]{0.58\columnwidth}\raggedright
Enviando instrucciones al corredor.
\end{minipage} & \begin{minipage}[t]{0.12\columnwidth}\raggedright
80h
\end{minipage}
\\\noalign{\medskip}
\begin{minipage}[t]{0.17\columnwidth}\raggedright
\end{minipage} & \begin{minipage}[t]{0.12\columnwidth}\raggedright
\end{minipage} & \begin{minipage}[t]{0.58\columnwidth}\raggedright
Consumiendo instrucciones del entrenador.
\end{minipage} & \begin{minipage}[t]{0.12\columnwidth}\raggedright
80h
\end{minipage}
\\\noalign{\medskip}
\begin{minipage}[t]{0.17\columnwidth}\raggedright
\end{minipage} & \begin{minipage}[t]{0.12\columnwidth}\raggedright
\end{minipage} & \begin{minipage}[t]{0.58\columnwidth}\raggedright
Mostrando publicidad en carrera.
\end{minipage} & \begin{minipage}[t]{0.12\columnwidth}\raggedright
120h
\end{minipage}
\\\noalign{\medskip}
\begin{minipage}[t]{0.17\columnwidth}\raggedright
\end{minipage} & \begin{minipage}[t]{0.12\columnwidth}\raggedright
\end{minipage} & \begin{minipage}[t]{0.58\columnwidth}\raggedright
Iniciando un entrenamiento.
\end{minipage} & \begin{minipage}[t]{0.12\columnwidth}\raggedright
60h
\end{minipage}
\\\noalign{\medskip}
\hline
\end{longtable}

Con lo cual tenemos un total de 480 horas, que considerando la
conformación de equipo dada anteriormente, corresponde a la cantidad de
horas hombre disponibles.

\subsection{Detalle de tareas para la primera iteración}

\subsubsection{Casos de uso y tareas}

A continuación incluimos las tareas a realizar con sus dependencias,
para la primer iteración (correspondiente a la primera fase de
elaboración presentada anteriormente).

\begin{longtable}[c]{@{}lll@{}}
\hline\noalign{\medskip}
\begin{minipage}[b]{0.28\columnwidth}\raggedright
Caso de uso
\end{minipage} & \begin{minipage}[b]{0.62\columnwidth}\raggedright
Tarea
\end{minipage} & \begin{minipage}[b]{0.10\columnwidth}\raggedright
Horas
\end{minipage}
\\\noalign{\medskip}
\hline\noalign{\medskip}
\begin{minipage}[t]{0.28\columnwidth}\raggedright
Logueando al usuario en el dispositivo
\end{minipage} & \begin{minipage}[t]{0.62\columnwidth}\raggedright
Investigar la arquitectura de login en Open Social.
\end{minipage} & \begin{minipage}[t]{0.10\columnwidth}\raggedright
8h
\end{minipage}
\\\noalign{\medskip}
\begin{minipage}[t]{0.28\columnwidth}\raggedright
\end{minipage} & \begin{minipage}[t]{0.62\columnwidth}\raggedright
Validar con los \emph{stakeholders} que tipo de sistemas de entrada
deben soportar los dispositivos.
\end{minipage} & \begin{minipage}[t]{0.10\columnwidth}\raggedright
4h
\end{minipage}
\\\noalign{\medskip}
\begin{minipage}[t]{0.28\columnwidth}\raggedright
\end{minipage} & \begin{minipage}[t]{0.62\columnwidth}\raggedright
Implementar capa de abstracción para interactuar con los dispositivos a
soportar y determinar sus características.
\end{minipage} & \begin{minipage}[t]{0.10\columnwidth}\raggedright
16h
\end{minipage}
\\\noalign{\medskip}
\begin{minipage}[t]{0.28\columnwidth}\raggedright
\end{minipage} & \begin{minipage}[t]{0.62\columnwidth}\raggedright
Implementar sistema de \emph{login} para cada dispositivo a soportar.
\end{minipage} & \begin{minipage}[t]{0.10\columnwidth}\raggedright
16h
\end{minipage}
\\\noalign{\medskip}
\begin{minipage}[t]{0.28\columnwidth}\raggedright
\end{minipage} & \begin{minipage}[t]{0.62\columnwidth}\raggedright
Diseñar e implementar mecanismos de logueo externos para soportar
dispositivos sin interfaz de entrada.
\end{minipage} & \begin{minipage}[t]{0.10\columnwidth}\raggedright
28h
\end{minipage}
\\\noalign{\medskip}
\begin{minipage}[t]{0.28\columnwidth}\raggedright
\end{minipage} & \begin{minipage}[t]{0.62\columnwidth}\raggedright
Implementar capa de interacción con redes sociales a soportar para
permitir el registro.
\end{minipage} & \begin{minipage}[t]{0.10\columnwidth}\raggedright
16h
\end{minipage}
\\\noalign{\medskip}
\begin{minipage}[t]{0.28\columnwidth}\raggedright
\end{minipage} & \begin{minipage}[t]{0.62\columnwidth}\raggedright
Diseñar módulos de interfaz con el dispositivo.
\end{minipage} & \begin{minipage}[t]{0.10\columnwidth}\raggedright
8h
\end{minipage}
\\\noalign{\medskip}
\begin{minipage}[t]{0.28\columnwidth}\raggedright
\end{minipage} & \begin{minipage}[t]{0.62\columnwidth}\raggedright
Documentar las interfaces esperadas de los dispositivos.
\end{minipage} & \begin{minipage}[t]{0.10\columnwidth}\raggedright
4h
\end{minipage}
\\\noalign{\medskip}
\begin{minipage}[t]{0.28\columnwidth}\raggedright
\end{minipage} & \begin{minipage}[t]{0.62\columnwidth}\raggedright
Testeo y \emph{debugging} de las implementaciones particulares y de la
interfaz general
\end{minipage} & \begin{minipage}[t]{0.10\columnwidth}\raggedright
4h
\end{minipage}
\\\noalign{\medskip}
\begin{minipage}[t]{0.28\columnwidth}\raggedright
\end{minipage} & \begin{minipage}[t]{0.62\columnwidth}\raggedright
Separar los tipos de usuarios entre corredores, entrenadores y
comentadores.
\end{minipage} & \begin{minipage}[t]{0.10\columnwidth}\raggedright
4h
\end{minipage}
\\\noalign{\medskip}
\begin{minipage}[t]{0.28\columnwidth}\raggedright
Mostrando datos y notificaciones
\end{minipage} & \begin{minipage}[t]{0.62\columnwidth}\raggedright
Validar con los \emph{stakeholders} que tipos de sistemas de aviso de
información han de soportar los dispositivos.
\end{minipage} & \begin{minipage}[t]{0.10\columnwidth}\raggedright
4h
\end{minipage}
\\\noalign{\medskip}
\begin{minipage}[t]{0.28\columnwidth}\raggedright
\end{minipage} & \begin{minipage}[t]{0.62\columnwidth}\raggedright
Implementar una capa de detección y abstracción de características de
notificación al usuario para los dispositivos.
\end{minipage} & \begin{minipage}[t]{0.10\columnwidth}\raggedright
30h
\end{minipage}
\\\noalign{\medskip}
\begin{minipage}[t]{0.28\columnwidth}\raggedright
\end{minipage} & \begin{minipage}[t]{0.62\columnwidth}\raggedright
Determinar que tipo de datos notificar al usuario sin que este los
demande.
\end{minipage} & \begin{minipage}[t]{0.10\columnwidth}\raggedright
4h
\end{minipage}
\\\noalign{\medskip}
\begin{minipage}[t]{0.28\columnwidth}\raggedright
\end{minipage} & \begin{minipage}[t]{0.62\columnwidth}\raggedright
Implementar una interfaz para soportar distintos tipos de datos a
notificar.
\end{minipage} & \begin{minipage}[t]{0.10\columnwidth}\raggedright
8h
\end{minipage}
\\\noalign{\medskip}
\begin{minipage}[t]{0.28\columnwidth}\raggedright
\end{minipage} & \begin{minipage}[t]{0.62\columnwidth}\raggedright
Implementar para cada dispositivo la metodología de envio de
notificaciones que se ajuste a sus capacidades técnicas.
\end{minipage} & \begin{minipage}[t]{0.10\columnwidth}\raggedright
30h
\end{minipage}
\\\noalign{\medskip}
\begin{minipage}[t]{0.28\columnwidth}\raggedright
\end{minipage} & \begin{minipage}[t]{0.62\columnwidth}\raggedright
Documentar las interfaces esperadas de los dispositivos.
\end{minipage} & \begin{minipage}[t]{0.10\columnwidth}\raggedright
4h
\end{minipage}
\\\noalign{\medskip}
\begin{minipage}[t]{0.28\columnwidth}\raggedright
\end{minipage} & \begin{minipage}[t]{0.62\columnwidth}\raggedright
Testeo y \emph{debugging} de las implementaciones particulares y de la
interfaz general.
\end{minipage} & \begin{minipage}[t]{0.10\columnwidth}\raggedright
4h
\end{minipage}
\\\noalign{\medskip}
\begin{minipage}[t]{0.28\columnwidth}\raggedright
Compartiendo datos en tiempo real de un entrenamiento
\end{minipage} & \begin{minipage}[t]{0.62\columnwidth}\raggedright
Investigar como compartir datos bajo Open Social
\end{minipage} & \begin{minipage}[t]{0.10\columnwidth}\raggedright
8h
\end{minipage}
\\\noalign{\medskip}
\begin{minipage}[t]{0.28\columnwidth}\raggedright
\end{minipage} & \begin{minipage}[t]{0.62\columnwidth}\raggedright
Implementar sistema de \emph{social sharing} para cada red social a dar
soporte en esta instancia.
\end{minipage} & \begin{minipage}[t]{0.10\columnwidth}\raggedright
16h
\end{minipage}
\\\noalign{\medskip}
\begin{minipage}[t]{0.28\columnwidth}\raggedright
\end{minipage} & \begin{minipage}[t]{0.62\columnwidth}\raggedright
Integrar sistema de \emph{login} a este módulo.
\end{minipage} & \begin{minipage}[t]{0.10\columnwidth}\raggedright
8h
\end{minipage}
\\\noalign{\medskip}
\begin{minipage}[t]{0.28\columnwidth}\raggedright
\end{minipage} & \begin{minipage}[t]{0.62\columnwidth}\raggedright
Validar políticas de privacidad de datos con \emph{stakeholders}
\end{minipage} & \begin{minipage}[t]{0.10\columnwidth}\raggedright
4h
\end{minipage}
\\\noalign{\medskip}
\begin{minipage}[t]{0.28\columnwidth}\raggedright
\end{minipage} & \begin{minipage}[t]{0.62\columnwidth}\raggedright
Testeo y \emph{debugging} del módulo.
\end{minipage} & \begin{minipage}[t]{0.10\columnwidth}\raggedright
4h
\end{minipage}
\\\noalign{\medskip}
\begin{minipage}[t]{0.28\columnwidth}\raggedright
Obteniendo posición propia y de amigos en el dispositivo empleado.
\end{minipage} & \begin{minipage}[t]{0.62\columnwidth}\raggedright
Validar con los \emph{stakeholders} que tipo de geolocalización se se
desea soportar.
\end{minipage} & \begin{minipage}[t]{0.10\columnwidth}\raggedright
4h
\end{minipage}
\\\noalign{\medskip}
\begin{minipage}[t]{0.28\columnwidth}\raggedright
\end{minipage} & \begin{minipage}[t]{0.62\columnwidth}\raggedright
Implementación de obtención de datos de geolocalización.
\end{minipage} & \begin{minipage}[t]{0.10\columnwidth}\raggedright
24h
\end{minipage}
\\\noalign{\medskip}
\begin{minipage}[t]{0.28\columnwidth}\raggedright
\end{minipage} & \begin{minipage}[t]{0.62\columnwidth}\raggedright
Implementación de sistema de notificación de geolocalización.
\end{minipage} & \begin{minipage}[t]{0.10\columnwidth}\raggedright
24h
\end{minipage}
\\\noalign{\medskip}
\begin{minipage}[t]{0.28\columnwidth}\raggedright
\end{minipage} & \begin{minipage}[t]{0.62\columnwidth}\raggedright
Integración con el módulo de \emph{social sharing}.
\end{minipage} & \begin{minipage}[t]{0.10\columnwidth}\raggedright
8h
\end{minipage}
\\\noalign{\medskip}
\begin{minipage}[t]{0.28\columnwidth}\raggedright
\end{minipage} & \begin{minipage}[t]{0.62\columnwidth}\raggedright
Testeo y \emph{debugging} del módulo.
\end{minipage} & \begin{minipage}[t]{0.10\columnwidth}\raggedright
4h
\end{minipage}
\\\noalign{\medskip}
\begin{minipage}[t]{0.28\columnwidth}\raggedright
Tomando datos de estado físico del corredor.
\end{minipage} & \begin{minipage}[t]{0.62\columnwidth}\raggedright
Validar que tipos de dispositivos médicos se van a soportar.
\end{minipage} & \begin{minipage}[t]{0.10\columnwidth}\raggedright
4h
\end{minipage}
\\\noalign{\medskip}
\begin{minipage}[t]{0.28\columnwidth}\raggedright
\end{minipage} & \begin{minipage}[t]{0.62\columnwidth}\raggedright
Implementar el módulo de detección de capacidades de medición para datos
biomédicos
\end{minipage} & \begin{minipage}[t]{0.10\columnwidth}\raggedright
32h
\end{minipage}
\\\noalign{\medskip}
\begin{minipage}[t]{0.28\columnwidth}\raggedright
\end{minipage} & \begin{minipage}[t]{0.62\columnwidth}\raggedright
Implementar módulo de medición para los dispositivos médicos.
\end{minipage} & \begin{minipage}[t]{0.10\columnwidth}\raggedright
32h
\end{minipage}
\\\noalign{\medskip}
\begin{minipage}[t]{0.28\columnwidth}\raggedright
\end{minipage} & \begin{minipage}[t]{0.62\columnwidth}\raggedright
Documentación de las interfaces esperadas.
\end{minipage} & \begin{minipage}[t]{0.10\columnwidth}\raggedright
4h
\end{minipage}
\\\noalign{\medskip}
\begin{minipage}[t]{0.28\columnwidth}\raggedright
\end{minipage} & \begin{minipage}[t]{0.62\columnwidth}\raggedright
Testeo y \emph{debugging} de la implementación.
\end{minipage} & \begin{minipage}[t]{0.10\columnwidth}\raggedright
8h
\end{minipage}
\\\noalign{\medskip}
\begin{minipage}[t]{0.28\columnwidth}\raggedright
Enviando datos a procesar a la nube local.
\end{minipage} & \begin{minipage}[t]{0.62\columnwidth}\raggedright
Investigar la plataforma Open Stack.
\end{minipage} & \begin{minipage}[t]{0.10\columnwidth}\raggedright
4h
\end{minipage}
\\\noalign{\medskip}
\begin{minipage}[t]{0.28\columnwidth}\raggedright
\end{minipage} & \begin{minipage}[t]{0.62\columnwidth}\raggedright
Implementar adaptador para llegada de datos a la nube de Secretaría.
\end{minipage} & \begin{minipage}[t]{0.10\columnwidth}\raggedright
8h
\end{minipage}
\\\noalign{\medskip}
\begin{minipage}[t]{0.28\columnwidth}\raggedright
\end{minipage} & \begin{minipage}[t]{0.62\columnwidth}\raggedright
Conseguir especificación técnica del software de procesamiento de datos
de la Secretaría.
\end{minipage} & \begin{minipage}[t]{0.10\columnwidth}\raggedright
16h
\end{minipage}
\\\noalign{\medskip}
\begin{minipage}[t]{0.28\columnwidth}\raggedright
\end{minipage} & \begin{minipage}[t]{0.62\columnwidth}\raggedright
Implementar módulo de envío de datos a la Secretaría.
\end{minipage} & \begin{minipage}[t]{0.10\columnwidth}\raggedright
8h
\end{minipage}
\\\noalign{\medskip}
\begin{minipage}[t]{0.28\columnwidth}\raggedright
\end{minipage} & \begin{minipage}[t]{0.62\columnwidth}\raggedright
Documentar la interfaz con el sistema.
\end{minipage} & \begin{minipage}[t]{0.10\columnwidth}\raggedright
8h
\end{minipage}
\\\noalign{\medskip}
\begin{minipage}[t]{0.28\columnwidth}\raggedright
\end{minipage} & \begin{minipage}[t]{0.62\columnwidth}\raggedright
Testeo y \emph{debugging} de la implementación.
\end{minipage} & \begin{minipage}[t]{0.10\columnwidth}\raggedright
4h
\end{minipage}
\\\noalign{\medskip}
\hline
\end{longtable}

En total tenemos entonces 460 horas, lo que nos da un poco de
\textbf{slack} en la iteración considerando la posibilidad de que surga
un evento inesperado o nuestras estimaciones sean incorrectas.

\subsubsection{Diagrama GANTT de la iteración}


\section{Arquitectura}
\textbf{Aclaración importante}: en los diagramas se omitió dibujar los puertos en los componentes por una cuestión de claridad y dimensiones.

\subsection{Conexión encriptada con ssl}\label{conexionEncriptadaConSSL}
\subsubsection{Diagrama}
% \ig{l}{0.8}{}{}


\subsubsection{Detalle}
\textbf{Nota}: se está asumiendo que en el repositorio de certificados ya se tienen los correspondientes a las autoridades certificantes usadas.

El \textbf{\texttt{handler de conexión}}, del lado del cliente es el encargado de inicializar la conexión con el servidor. Cuando recibe instrucciones de establecer una \texttt{conexión encriptada} con un servidor (esto se hace mediante un primer paquete con un formato especial en la \texttt{cola de paquetes}, no es necesaria una conexión adicional con el handler de mensajes), se comunica con el \texttt{manejador de certificados} mediante un \texttt{call return} para averiguar si ya tiene almacenado el certificado correspondiente al servidor. Para averiguar esto, el \texttt{manejador de certificados} lee el \texttt{repositorio de certificados SSL}, donde se almacenan estos certificados. Si lo tiene, lo devuelve.

En el caso de que no lo tenga, se comunica con el servidor mediante un \texttt{canal confiable no encriptado} y le solicita su certificado. Este certificado tiene numerosos datos, uno de los cuales es cuál es la autoridad certificante que lo valida. Una vez obtenido el certiciado, se comunica nuevamente con el \textbf{manejador de certificados} y le solicita la clave pública de la autoridad certificante correspondiente (por la suposición anterior, este dato está en el repositorio). El certificado de la autoridad certificante tiene, entre otros datos, su clave pública. Usando eso, se comunica en forma segura con la autoridad certificante y solicita que valide el certificado que envió el servidor destino. Una vez validado el certificado, se extrae de este su clave pública y se la utiliza para cifrar una serie de mensajes que serán intercambiados con el \texttt{handler de conexión}, del lado del servidor. En este intercambio de mensajes ambos \texttt{\emph{handlers}}, acuerdan una \textbf{clave de sesión} y un \textbf{algoritmo simétrico}.

Obtenida esta clave de sesión, el \texttt{handler de conexión} inicializa el componente \texttt{codificador simétrico} con este par \emph{(clave,algoritmo)} comienza el proceso de obtener los siguientes paquetes de la \texttt{cola de paquetes} y pasárselos mendiante una cola al \texttt{codificador simétrico}. Una vez que este los encripta con la clave y algoritmo correspondientes se los devuelve (también mediante una cola) al \texttt{\emph{handler}}, que se ocupa de enviarlos, mediante un \texttt{canal confiable no encriptado} (aunque estos paquetes ya están cifrados) al \texttt{handler de conexión (servidor)}. Éste, cuando los recibe, los encola en su \texttt{decodificador simétrico} (que ya oportunamente inicializó con el mismo par \emph{(clave, algoritmo)} acordado). A medida que son decodificados los va encolando en la \texttt{cola de mensajes} correspondiente.

\paragraph{Actualización de repositorio}
Para manejar los certificados presentes en el \texttt{repositorio de certificados SSL}, existe un componente (el \texttt{updater de repositorio de certificados}), que recibe pedidos del \texttt{updater global de certificados SSL} (que está corriendo en los servidores de la Secretaría de Deportes), para agregar, modificar o revocar certificados existentes. 


\subsection{Notificador con prioridad}\label{notificadorConPrioridad}
\subsubsection{Diagrama}
% \ig{l}{0.9}{}{}

\subsubsection{Detalle}
El conector \textbf{\texttt{notificador con prioridades}} tiene como objetivo recibir numerosos pedidos de notificaciones de distintos componentes y enviárselos a quien corresponda, siguiendo un orden de prioridades. De esta forma, se busca evitar que, por ejemplo, notificaciones del \texttt{sistema de comentarios} opaquen o retrasen notificaciones más importantes, tales como la del \texttt{subsistema de datos biomédicos}.

Para realizar esto, el conector permite muchas conexiones entrantes que se le envían sus respectivos paquetes al \texttt{handler de prioridades} mediante respectivas colas. Al recibir un paquete, el \emph{handler} revisa su repositorio de prioridades para asignarle una prioridad al mensaje. En este momento hay definidos 4 niveles de prioridades (\emph{P0}, \emph{P1}, \emph{P2} y \emph{P3}), pero esto puede ser modificado sencillamente en \emph{updates} futuros. En el caso de que el mensaje provenga de un emisor no registrado en el repositorio, se levanta una alerta y no se pasa la correspondiente notificación. De esta forma, el canal no sólo sirve para establecer prioridades, sino que como medida de \textbf{seguridad}, permite filtrar intentos de notificaciones de emisores no registrados. En el caso que el mensaje provenga de un emisor registrado, el \emph{handler} le asigna una prioridad al mensaje, lo \emph{taggea} y lo envía al \texttt{notificador} mediante la cola correspondiente.

El \texttt{notificador} tiene como función ir revisando los mensajes que le lleguen por las diferentes colas, siempre privilegiando las colas de mayor prioridad. En este caso no nos preocupa el \emph{starvation}: consideramos aceptable que se pierdan (o demoren infinitamente) notificaciones de más baja prioridad en pos de notificaciones de prioridad más elevada (por ejemplo, una notificación de instrucciones del entrenador tiene más prioridad que un comentario motivacional de una red social). En este momento, las prioridades definidas son:
\begin{itemize}
	\item \textbf{P0}: Notificaciones del \texttt{sistema de notificaciones por dispositivo biomédico}. 
	\item \textbf{P1}: Notificaciones del \texttt{sistema de notificaciones del entrenador}.
	\item \textbf{P2}: Notificaciones de: 
	\begin{itemize}
		\item \texttt{sistema de seguimiento de entrenamiento}.
		\item \texttt{sistema de seguimiento de carrera virtual}.
	\end{itemize}
	\item \textbf{P3}: Notificaciones del \texttt{sistema de comentarios} (para redes sociales).
\end{itemize}

\paragraph{Actualización de repositorio}
Como se dijo, las prioridades definidas pueden cambiar con \emph{updates} futuros. Consideramos que no es importante tener un módulo encargado de actualizar remota y dinámicamente este repositorio en este momento, dado que tanto las actualizaciones posibles (quitado o modificación de un componente existente o agregado de uno nuevo) sólo tienen sentido cuando se modifica alguno de los antedichos componentes. Consideramos que para esta primera etapa (quizás esto se cambie en un \emph{update} futuro), el repositorio contiene la información en forma estática, cargada al momento de instalar y sólo modificada con sucesivos \emph{updates} de la aplicación.


\addcontentsline{toc}{section}{Referencias}
\begin{thebibliography}{8}
\raggedright

\bibitem{Gamma}
	E.~Gamma, R.~Helm, R.~Johnson, and J.~Vlissides.
	\newblock {\em Design Patterns: Elements of Reusable Object-Oriented Software}.
	\newblock Professional Computing. Addison-Wesley, 1995.

\end{thebibliography}

\end{document}
