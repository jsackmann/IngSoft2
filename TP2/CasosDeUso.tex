\section{Casos de Uso}

A continuación detallamos los casos de uso que consideramos para el proyecto.

\subsection{Publicidad}
En este diagrama se muestran los casos de uso que conciernen a la publicidad dentro del sistema.

\ig{0.8}{CDU_Publicidad.pdf}{Casos de uso de publicidad}

\begin{itemize}
	\item \textbf{Logueándose al sistema}: un sujeto ya registrado inicia sesión desde su dispositivo para comenzar a utilizar el sistema. Para eso debe ingresar su nombre de usuario y password.
	\item \textbf{Añadiendo/modificando publicidad al sistema}: una entidad publicitaria ya registrada y logueada añade o modifica nueva publicidad al sistema.
	\item \textbf{Registrando/modificando entidad publicitaria}: luego de que se sigan todos los procesos burocráticos pertinentes, la secretaría de deportes registra en el sistema a una entidad publicitaria para que agregue publicidad al sistema.
	\item \textbf{Consumiendo publicidad en carrera}: el corredor consume publicidad contextual.
	\item \textbf{Emitiendo publicidad contextual}: el dispositivo, de acuerdo a su ubicación emite publicidad contextual. Esta puede presentarse en forma auditiva o visual, de acuerdo a las características del dispositivo.
	\item \textbf{Iniciando un entrenamiento}: un corredor previamente autenticado inicia una sesión de entrenamiento. Esta puede o no ser virtual.
	% \item \textbf{Finalizando un entrenamiento} ?????? HACEMOS ESTO ?????? 
\end{itemize}

\subsection{Registración y autenticación}
En este diagrama se muestran los casos de uso que conciernen a la registración y autenticación (llamado ``loguearse'') de los distintos sujetos ante el sistema. 

\ig{0.8}{CDU_RegistrandoLogueando.pdf}{Casos de uso de registración y autenticación}

\begin{itemize}
	\item \textbf{Registrándose al sistema}: un sujeto aún no registrado puede utilizar su dispositivo para registrarse en el sistema con el fin de comenzar a utilizarlo. Se puede registrar como corredor o como entrenador, ofreciendo diferentes funciones a cada uno. Al momento de registrarse como corredor, el sistema ofrece ``linkear'' la cuenta recién creada con las distintas redes sociales, que permitirá luego publicar datos y obtener comentarios motivacionales de las mismas.
\end{itemize}


\subsection{Redes Sociales}
En este diagrama se muestran principalmente interacciones entre un corredor y sus distintos amigos a través de las redes sociales.

\ig{0.8}{CDU_RedesSociales.pdf}{Casos de uso de redes sociales}

\begin{itemize}
	\item \textbf{Mirando posición propia y de amigos en el mapa}: un usuario ya autenticado puede utilizar el dispositivo para mirar su posición actual y la de sus amigos en un mapa. Para eso, el sistema utiliza los servicios del geolocalizador. Esto puede ocurrir aún cuando el usuario no esté en un entrenamiento.
	\item \textbf{Consumiendo comentario motivacional}: un corredor, esté o no en un entrenamiento, puede acceder a los comentarios motivacionales escritos por sus amigos a través de su dispositivo. Dependiendo de las características del mismo y las circunstancias, este acceso puede darse en forma visual o auditiva.
	\item \textbf{Levantando contenidos motivacionales de las redes sociales}: Si el corredor al registrarse asoció su cuenta con una cuenta de una red social, el sistema periódicamente accede a la antedicha red social para obtener los comentarios motivadores que se dejaron en la misma. 
	\item \textbf{Compartiendo los datos de un entrenamiento en tiempo real}: un corredor, durante un entrenamiento puede seleccionar compartir sus datos de entrenamiento mediante las redes sociales que tenga asociadas a su cuenta directamente desde su dispositivo.
	\item \textbf{Mostrando datos y notificaciones del entrenamiento actual}: un corredor, durante un entrenamiento, accede a sus notificaciones y datos del mismo directamente desde su dispositivo. Dependiendo del dispositivo, este acceso puede presentarse en forma visual o auditiva.
\end{itemize}


\subsection{Estado Físico}
En este diagrma se muestran casos de uso concernientes a las interacciones entre el entrenador y el corredor, y el estado físico de este último.

\ig{0.8}{CDU_EstadoFisico.pdf}{Casos de uso de Estados físicos}

\begin{itemize}
	\item \textbf{Enviando instrucciones a corredor}: un entrenador autenticado envía instrucciones a un corredor utilizando su dispositivo. Esto no necesariamente ocurre cuando el corredor está en un entrenamiento.
	\item \textbf{Consumiendo instrucciones de entrenador}: un corredor profesional autenticado, esté o no en un entrenamiento, puede consumir instrucciones de su entrenador mediante su dispositivo. 
	\item \textbf{Recibiendo notificación de necesidad de descanso}: luego de recibir datos del estado actual del corredor de un medidor biomédico, si es pertinente, el sistema el sistema informa al corredor la necesidad de que se tome un descanso. Esta notificación puede ser visual o auditiva, dependiendo del dispositivo y las circunstancias.
	\item \textbf{Informando estado físico del corredor}: un dispositivo biomédico informa al sistema del estado físico del corredor cuando éste está en un entrenamiento.
\end{itemize}


\subsection{API y datos exportados}
En este diagrma se muestran casos de uso concernientes a los datos y servicios que exporta el sistema.

\ig{0.8}{CDU_APIs.pdf}{Casos de uso de APIs y datos exportados}

\begin{itemize}
	\item \textbf{Utilizando la api de datos físicos almacenados}: Se expone una API para que sistemas externos puedan utilizar los datos biomédicos (que son anonimizados previamente y están tomados del corredor con su consentimiento) como deseen, por ejemplo hacer un estudio de calambres por parte de una organización médica. 
	\item \textbf{Enviando datos para procesar a la nube local}: La Secretaría de Deportes dispone de una nube OpenStack a la cual la aplicación le enviará datos de una manera que pueda usar para diversos tipos de procesamiento. 
	\item \textbf{Mostrando datos pedidos mediante voz}: El corredor puede, mientras usa la aplicación y va corriendo, darle indicaciones de voz a la aplicación para que la misma le provea otros datos. El reconocimiento de voz es realizado por fuera de la aplicación, utilizando una API externa (como por ejemplo Google Voice API) y la misma utiliza los datos procesados para responder al pedido de usuario.
\end{itemize}


\subsection{Usos especiales}
En este diagrama se muestran casos de uso concernientes a los usos ``especiales'' que se les da al sistema.

\ig{0.8}{CDU_UsosEspeciales.pdf}{Casos de uso de usos especiales}

\begin{itemize}
	\item \textbf{Corriendo una carrera virtual}: El corredor puede optar por, en vez de realizar un entrenamiento de los preparados por la aplicación, utilizar un entrenamiento de otro corredor en el mismo lugar geográfico, con los datos de tiempo almacenados, y correr contra este para ver si puede vencer su tiempo. Los datos de estas carreras también son mostrados por la aplicación como si fuese un entrenamiento común.
	\item \textbf{Usando aplicación con refuerzo para discapacidad}: Dado que se pretende que los usuarios con dificultades visuales o auditivas puedan utilizar la aplicación de manera satisfactoria, los mismos deben poder indicar como desean recibir las notificaciones, el tamaño de muestra de los mismos (asumiendo dispositivo con soporte visual), etc. para tener una experiencia más agradable de uso.
\end{itemize}
