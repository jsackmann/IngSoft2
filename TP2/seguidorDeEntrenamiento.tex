\subsection{Seguidor de entrenamiento y carrera virtual}
\subsubsection{Diagrama}

\ig{l}{0.9}{seguidorDeEntrenamiento.png}{Diagrama de componentes y
conectores del sistema de seguimiento de entrenamiento.}

\subsubsection{Detalle}

En el diagrama anterior detallamos los componentes y conectores del
sistema de seguimiento de entrenamiento. 

Este sistema se ocupa de seguir al corredor mientras realiza un 
entrenamiento (como fue especificado en el primer trabajo práctico
de la materia). Por un lado se ocupa de permitir crear entrenamientos
y por otro lado se ocupa de procesar un entrenamiento en carrera.

Los entrenamientos son almacenados del lado del servidor y el
\texttt{Procesador de estado de seguimiento} se ocupa de obtener
(y cachear) los mismos en el dispositivo. Los datos se obtienen del
\texttt{Sistema de cinemática} y con ellos se va formando el registro
de los entrenamientos realizados y se van enviando los datos 
generales para ser notificados al usuario y enviados al entrenador del
mismo si es pertinente.

También, al concluir, se commitean estos resultados a un 
\texttt{Centralizador de entrenamientos}, con el objetivo de que 
se almacenen los datos del entrenamiento del usuario para que 
puedan utilizarse para carreras virtuales.

También este sistema incluye un procesador de pedidos (bajo el nombre
\texttt{Sistema de querys de entrenamiento} que contesta consultas
más generales (y que no se muestran siempre para que la interfaz no
se llene de datos posiblemente irrelevantes) sobre el estado del
entrenamiento a los sistemas que lo necesiten (en particular, el
\texttt{Sistema de información contextual}.

\subsubsection{Diagrama}

\ig{l}{0.9}{carrerasVirtuales.png}{Diagrama de componentes y
conectores del sistema de carreras virtuales.}

\subsubsection{Detalle}

En el diagrama anterior se detalla los componentes y conectores del
sistema de carreras virtuales.

Dado que este sistema es análogo al sistema de seguimiento de
corredores, no lo detallaremos mucho más. La principal diferencia es
que se utiliza la información de cinemática (en particular la posición
actual) junto con un acceso a un servidor de carreras virtuales que
utiliza los entrenamientos de los otros corredores para crear carreras
virtuales.

Para permitir al sistema continuar operando incluso fuera de línea, se
tiene un cache local de carreras virtuales que guarda las ultimas
carreras virtuales corridas de manera que se puedan volver a realizar
(y en particular guarda la última para que se pueda realizar la misma
sin tener que ir a buscar los datos de la misma cada vez al servidor).

