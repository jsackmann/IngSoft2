\subsection{Información contextual}
\subsubsection{Diagrama}

\ig{l}{0.7}{informacionContextual.png}{Diagrama de componentes y conectores del sistema de información contextual.}

\subsubsection{Detalle}

En el diagrama anterior se incluye el detalle de componentes y
conectores del sistema de información contextual. Este sistema
se ocupa de paginar información de varias fuentes y utilizarla para
generar notificaciones pertinentes tanto para el usuario como 
potencialmente para los entrenadores del mismo (que necesitan seguir
su performance para poder darle consejos adecuados).

Otro aspecto de importancia para este sistema es que procesa los 
comandos relacionados con pedidos de información, de manera de 
proveerselos al usuario \textit{on demand} (de otra manear sería
necesario mostrar demasiada información, haciendo al sistema inusable).

Para ello, en primer lugar tenemos que al llegar un nuevo pedido, se
determinan las dependencias a satisfacer (es decir, que valores 
necesito mostrarle al usuario para poder responder su consulta).

Estos valores son pedidos al sistema mediante un 
\texttt{Colector de dependencias}. Este se ocupa de interactuar con
los distintos sistemas. Por ejemplo, se ocupa de conseguir las 
posiciones de los amigos interactuando con las posiciones históricas
de los mismos (dadas por el sistema \texttt{Posiciones de usuarios}).

Este Sistema también se ocupa de compaginar la información de los
sistemas de \texttt{Seguimiento} y de \texttt{Competencias virtuales}
para soportar consultas como por ejemplo ``Cuanto me lleva por delante
mi competidor virtual?''. 

Por otro lado, es necesario mantener la información propia. Para ello
se utiliza el \texttt{Driver de cinemática}, que devuelve valores como
por ejemplo la velocidad, dirección y posición del sistema a 
ciertos intervalos. Estos valores son mantenidos por el sistema de
información contextual de manera no solo de mostrarlos como 
notificaciones sino también actualizar con esto las publicidades que
se muestran al usuario (para dar contextualización), y mantener fresco
el sistema de posiciones y el cache de datos propios (para poder
responder consultas históricas). 

Los amigos del corredor son obtenidos utilizando su perfil (disponible
desde el \texttt{Sistema de configuración}) y la interfaz con las
redes sociales. Los amigos del corredor son cacheados para evitar
tener que acceder todas las veces a los datos. Este caché expira cada
cierto tiempo y debe ser renovado (de lo cual también se encarga este
componente).

Por ultimo, este subsistema utiliza un componente de cliente de 
servicio publicitario con datos de geolocalización y de entrenamiento
para mostrar publicidad especialmente dirigida a la actividad que
el corredor se encuentre realizando.
