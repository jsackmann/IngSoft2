\section{Comparación}

\subsection{UP vs Ágil}

La metodología \textbf{Unified Process (\texttt{UP})} es un proceso que se basa en el desarrollo de forma iterativa e incremental y se centra principalmente en la arquitectura, y los casos de uso, que son los que guían este desarrollo.
\texttt{UP} consta de 4 fases: Incepción, Elaboración, Construcción y Transición. Cada una de estas fases apunta a una etapa distinta del desarrollo. En \texttt{UP}, no es lo mismo el comienzo del proyecto, en donde se recolecta información del mismo, se genera cierta documentación , por ejemplo de requerimientos funcionales, y no funcionales y casos de uso, y se establece el caso de negocio, que por ejemplo la fase de Construcción, en la que se desarrolla y prueba el producto. 
A su vez, hay iteraciones. En cada iteración se obtiene un sistema funcionando pero no necesariamente es un entregable.

En cambio, la metodología \textbf{\texttt{Scrum}} (que es una metodología ágil) no tiene distinción de etapas, aunque si tiene iteraciones. Pero, en este caso, al finalizar cada iteración se obtiene un entregable.

\paragraph{Iteraciones}
Ambas metodologías buscan con las iteraciones obtener feedback del usuario, y de así, de ser necesario, realizar tempranamente los cambios que se crean oportunos, para lograr la satisfacción de los usuarios. En particular, en \texttt{Scrum}, se esepra que los pedidos de cambios se realizan al principio o fin de la iteración, pero no en el medio de la iteración.

\paragraph{Tipo de planificación} 
También en ambas se realiza planificación de cada iteración, pero es distinta para ambos casos. Mientras que en \texttt{UP}, se planifican los \textbf{Casos de uso}, y al comienzo del proyecto se planifica todo el desarrollo y se expresa en un cronograma, en \texttt{Scrum} se planifican \textbf{User Stories}, y solo se planifica la iteración actual. Con lo cual es más difícil tener una estimación precisa de cuando será la finalización del proyecto.

\paragraph{Prioridades}
Otra de las diferencias entre ambas metodologías tiene que ver con las prioridades. Para la metodología \texttt{UP} es clave tener en cuenta los riesgos, se busca evitarlos, o disminuirlos. Sin embargo, en \texttt{Scrum}, lo prioritario es aportarle valor al cliente, y esto se logra priorizando aquellas User Stories que le permitan ver al cliente más rápidamente las funcionalidades principales del sistema.

\paragraph{UP:grandes poyectos, Scrum: proyectos medianos}
Lo que podemos ver en cuanto a estas metodologías y sus características es que el tipo de planificación de \texttt{UP}, y las formas en las que se toman decisiones, las iteraciones y las etapas son útiles para proyectos muy grandes, de mucha complejidad y críticos. \texttt{Scrum}, sin embargo, es útil para proyectos no tan grandes, en los que se requiere flexibilidad y adaptabilidad, y es por esta razón que la elección de la metodología a usar dependerá del tipo de proyecto que se quiere realizar.


\subsection{Programming in the small vs Programming in the large}

Cuando se habla de \emph{\textbf{Programming-in-the-large}} se habla del desarrollo de proyectos de software que incluyen grandes grupos de personas o grupos pequeños de personas a lo largo de mucho tiempo. \cite{Brooks}

Debido a que el desarrollo de un proyecto a esta escala puede volverse muy complejo, es de vital importancia que hayan definidos procesos, y haya una planificación y un cronograma. Que esté definido el objetivo, prioridades, y atributos de calidad, para no perder el foco de lo que se busca y se quiere lograr y poder cumplir con la planificación. 
Es por esto que resulta muy útil definir la arquitectura para \emph{Programming -in-the-large}.

Al definir la arquitectura puede verse la estructura del sistema, como éste cumplirá con los atributos de calidad, y cuales son las tácticas usadas para lograrlo.
Además, resulta primordial el análisis de riesgos para prevenir y tratar de evitar lo antes posible los problemas que pudieran llegar a surgir, y que, de no ser previstos, podrían provocar que el sistema se vuelva inviable.

Al hablar de \emph{\textbf{Programming-in-the-small}}, se hace referencia al desarrollo de sistemas pequeños. Al ser pequeños, requieren de una menor planificación, y no es necesario hacer grandes análisis en cuanto a riesgos debido a que las consecuencias  pueden llegar a ser menos impactantes (igualmente esto depende del tipo específico de proyecto, y su criticidad), ni tampoco es necesario tener mucha documentación a la hora de desarrollar, debido a que, al ser más pequeño es más sencillo de entender.
Y es por esta razón que suele usarse diseño orientado a objetos que permite modelar el sistema a desarrollar.

Por estas razones, suele asociarse \emph{Programming-in-the-large} con metodologías como \texttt{UP}, y a \emph{Programming-in-the-small} con metodologías ágiles, como \texttt{Scrum}.