\subsection{Sistema de cinemática}
\subsubsection{Diagrama}

\ig{p}{0.55}{sistemaDeCinematica.png}{Diagrama de componentes y conectores del sistema de cinemática}

\subsubsection{Detalle}

El rol de este sistema, cuyo diagrama de componentes y conectores
incluimos previamente, es el de proveer información del dispositivo
que es resultado de mediciones cinemáticas del mismo: posición,
velocidad, dirección, etc. La posición por ejemplo, requerimos que sea
geolocalizada (ya que por ejemplo esto es necesario para lograr enviar
una ambulancia en caso de emergencia en el menor tiempo posible).

Decidimos utilizar una arquitectura basada en heartbeat del sistema
de cinemática: el mismo envía cada cierto tiempo (configurable por
el usuario) una actualización. La misma consiste de los datos que
puede medir el dispositivo en particular. Un ejemplo es que Google
Glass tiene giroscopios y velocímetros, con lo cual puede incluso
obtener la dirección de movimiento. Otros dispositivos puede que no,
pero estos datos pueden estimarse mediante otros (por ejemplo, la
posición geolocalizada) o usar dispositivos externos que tengan
habilitada esta funcionalidad (por ejemplo un celular).

Puesto que un servicio de geolocalización puede caerse, o puede no
soportar la tasa de refresco requerida por la configuración, también
se considera un caché de datos que permite realizar interpolaciones
de los datos entremedio de manera de dar un soporte lo más fiel 
posible al sistema (para que este no parezca caído, lo cual nos 
interesa dado que buscamos disponibilidad lo mayor posible del 
sistema).

Para esto tenemos un colector que escribe los valores en un 
repositorio local que es utilizado por el estimador. Cuando el 
colector es consultado por un valor que no posee, utiliza el estimador
para derivar una respuesta cercana. Este valor es también ingresado
al repositorio para permitir cotejarlo con nuevos valores que vayan
surgiendo de mediciones reales.
