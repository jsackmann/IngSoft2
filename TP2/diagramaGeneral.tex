\subsection{Diagrama general}
\subsubsection{Diagrama}

\subsubsection{Detalle}

El anterior diagrama corresponde a un diagrama a gran escala de la
aplicación, de manera de identificar los sistemas más importantes.

El diagrama se encuentra dividido en dos claras partes. Los sistemas
en la parte derecha corresponden a los distintos sistemas de la
aplicación. Estos sistemas interactuan entre si de diversas maneras pero lo 
principal que pretendíamos que se viera en este diagrama es que los mismos 
envían notificaciones al usuario, las cuales son priorizadas 
y mapeadas a las características del dispositivo y la configuración 
del usuario.

Algunos de los sistemas que indicamos en el diagrama los consideramos
de un tamaño muy reducido como para merecer un diagrama aparte. Por
ejemplo, en general los sistemas de notificaciones intermedios (por
ejemplo el \texttt{Sistema de notificaciones de entrenador}) no hace
más que poner unos tags de tiempo y otras operaciones lógicas a aplicar
sobre los comentarios de entrenadores) nos parecieron demasiado sencillos
como para merecer un diagrama aparte.

Otro ejemplo de similar razonamiento es para el sistema \texttt{Sistema de
posiciones de usuarios} que simplemente se ocupa de guardar las posiciones
que le son enviadas de manera de tener un registro de las mismas que se pueda
usar para mostrar como información contextual en el mapa.

Sin embargo, varios de los sistemas marcados si son importantes y por lo
tanto incluimos a continuación un breve detalle de los mismos. Se 
especificarán con otros diagramas a continuación también.

\begin{itemize}
	\item \textbf{Sistema de datos médicos}: Este sistema se ocupa de
	las mediciones de datos de condición física (ejemplo: presión
	arterial, palpitaciones, etc.) con sus mensajes de notificación
	asociados (``palpitaciones excedidas, disminuya la marcha'', etc.).
	\item \textbf{Interfaz de redes sociales}: Este sistema se ocupa
	de ser el puente entre las redes sociales que soporta la plataforma
	Open Social y el sistema en si. Por ejemplo, posteando estado en
	tiempo real de la carrera del corredor y permitiendo comentarios
	motivacionales por afuera.
	\item \textbf{Sistema de información contextual}: Este sistema se ocupa
	de obtener y centralizar datos a mostrar al corredor, sea de la 
	posición de sus amigos o procesar datos que pida mediante la
	interfaz del dispositivo o del teléfono con el que este conectado.
	\item \textbf{Sistema de seguimiento de entrenamiento}: Este sistema se
	ocupa de realizar el seguimiento del entrenamiento del corredor
	mientras usa el dispositivo, incluyendo procesar los comandos para crear
	un entrenamiento, pedir estado del seguimiento para mostrar 
	notificaciones en particular y mostrar datos generales del seguimiento.	
	\item \textbf{Sistema de seguimiento de carrera virtual}: Este sistema
	se ocupa de permitir al corredor realizar carreras virtuales contra 
	otros corredores que esten en la misma zona.
	\item \textbf{Frontend de indicaciones}: Este sistema se ocupa de
	obtener indicaciones de los entrenadores a sus corredores profesionales
	y mandar las mismas como notificaciones. Chequea por ejemplo los permisos
	y se ocupa de identificar a los entrenadores.
	\item \textbf{Sistema de adición de publicidad}: Este sistema se ocupa de 
	aceptar publicidad contextualizada de acuerdo a \textit{features} dados
	por el dispositivo y enviar una notificación a compaginar por el sistema 
	de información contextualizada. También se ocupa de chequear los permisos
	usados para ingresar publicidad de manera que no se puedan inyectar 
	datos al sistema,  mediante el uso de encripción.
\end{itemize}

Un subsistema que no consideramos que merece un diagrama aparte consiste en
el \texttt{Sistema de configuración}. Este sistema se ocupa de mantener la
configuración del usuario: Que tipo de necesidades diferentes tiene, que 
cuenta de OAuth posee, etc. Puesto que los componentes consisten basicamente
en entradas a una base de datos y salidas que leen de la misma, no nos
pareció pertinente detallarlo más alla de estas palabras.

También tenemos el \texttt{Sistema de entrada al usuario}. Dado que la 
cantidad de dispositivos de entrada posibles es grande, este sistema tiene 
que manejar distintos posibles de tipos de entrada y convertirlos a 
comandos útiles para el sistema. 
Por ello consideramos que su interacción merece un subsistema aparte.

Por último haremos mención del \texttt{Sistema de interfaz} con el usuario para el envio de las notificaciones. Esta interacción esta específicada en mayor
detalle mediante un diagrama de componentes y conectores posteriormente en
el trabajo práctico.

