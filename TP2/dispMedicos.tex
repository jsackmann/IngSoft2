\subsection{Dispositivos biomédicos} \label{dispMedicos}
\subsubsection{Diagrama}

\ig{l}{0.8}{dispMedicos.png}{Diagrama de componentes y conectores del
sistema de dispositivos biomédicos.}

\subsubsection{Detalle}

Este diagrama detalla la arquitectura del sistema de dispositivos médicos.

En este sistema tenemos por un lado el sistema de driver de dispositivo, que
toma una medición. Esta medición o señal es etiquetada (de manera de que
se identifique de que dispositivo es, datos sobre el usuario y un tag
de geolocalización en caso de emergencia) y luego enviada al router de
mediciones médicas que se ocupa de enviarlo al resto de los subsistemas que
necesiten hacer algo con el.

Estos subsistemas se dividen en dos. Por un lado tenemos el sistema de 
detección de mediciones peligrosas. Este sistema puede detectar que hay un
posible problema de salud a partir de la medición obtenida (o usar mediciones acumuladas, depende de la implementación particular que dejamos para 
posterior consideración). Debe entonces avisar al usuario de la gravedad del
problema notificandolo por su dispositivo y, si hay un riesgo de vida muy
serio, enviar un mensaje a emergencias con la posición que tenía la 
medición. Para poder garantizar que esto se produzca de la manera más rápida
posible, utilizamos redundancia activa para mantener dos sistemas que hacen
este análisis y posteriormente envian un mensaje a emergencias. Este mensaje
es enviado utilizando un sistema de envío confiable que considera la 
posibilidad de que la conexión con emergencias sufra fallas que impidan 
enviar bien el pedido si se hacen de manera incorrecta.

El problema de multiples pedidos se arregla utilizando para ello un 
\textit{timestamp} en los pedidos de manera que los sistemas instalados del
otro lado de emergencias puedan detectar los replays (porque además 
preferimos que lleguen dos pedidos de emergencias a que haya una mayor
posibilidad de que no llegue ninguno).

El otro sistema es el sistema de envío de datos. El mismo empieza enviando
los datos para procesar. Estos datos se envían encriptados (para evitar que
los datos sean capturados por un atacante externo) pero no se envían de forma
confiable pues perder datos no nos preocupa particularmente. Estos datos
son procesados utilizando varios sistemas (en particular, de acuerdo a la
especificación del trabajo práctico, este sistema es la misma nube de 
procesamiento OpenStack local a la secretaría, pero posteriormente podría
expandirse para incluir otras nubes). Para manejar estos sistemas y permitir
adicionar sistemas de manera horizontal (mejorando así la escalabilidad, y
para soportar uno de los atributos de \textit{modificabilidad} señalados)
primero utilizamos un balanceador de carga que se encarga de asignar trabajos
a los distintos módulos de procesamiento. Estos procesan pequeñas partes
del problema y luego envían los resultados a un colector, que se ocupa de
serializar y post procesar los datos obtenidos. Uno de estos pasos de post-
procesamiento es encriptar los datos para almacenarlos en la base de datos.
Al usarse encripción homomórfica\footnote{\texttt{http://blogs.teamb.com/craigstuntz/2010/03/18/38566/}} los datos se pueden procesar sin desencriptarlos,
manteniendo entonces la privacidad de los datos médicos de los corredores
(uno de los atributos de calidad que nos interesaban). 

Para permitir el acceso a estos datos solamente a entes autorizados, se 
utiliza una estructura cliente servidor con autenticación para los 
consumidores de la API. 

Puesto que el servidor solo debe verificar la 
identidad del consumidor revisando su identificación de clave de API y usar
el repositorio de datos obtenido, no nos parece pertinente detallar más el
módulo.

Para dar un poco más de detalle al sistema de procesamiento de datos en
la nube, incluimos el siguiente diagrama:

\ig{p}{0.8}{procDatosMedicos.png}{Diagrama de componentes y conectores de procesamiento de datos médicos.}

En este se puede ver los componentes y conectores internos al sistema de
procesamiento. Como podemos ver, el balanceador de carga toma la decisión
de a que nube de procesamiento enviar los datos utilizando un chequeo de
salud basado en la táctica de \textit{ping-echo}. El sistema envia 
periodicamente pedidos de estado de carga de cada uno. Si no recibe una
respuesta luego de un cierto tiempo asume que la nube eseta caída. Estos
datos son almacenados en un cache interno para que le sirvan al scheduler 
como manera de decidir a que nube asignar el trabajo. Adicionalmente estos
datos son almacenados en un auditor de sistema para poder ser accedidos
mediante el sistema de Dashboard de OpenStack y poder hacer diagnósticos.

Cada nube de procesamiento realiza un computo y envía el resultado al
colector (de manera confiable, para no gastar el cómputo realizado por
errores). El colector va realizando una serie de procesos sobre los datos,
desde determinar su numero de batch (asignado al principio por el 
scheduler de trabajos), aplicar una serie de reducciones a los datos del
batch (por eso necesita encolar parcialmente los datos), juntando los datos
en una base de datos intermedia (de manera de poder juntar todo un batch
y enviarlo apenas este) y luego encriptando los datos y escribiendolos en el
repositorio de datos. De esta manera se puede procesar los datos con una 
buena performance (de manera que los datos más frescos esten disponibles 
para los consumidores de la API).
