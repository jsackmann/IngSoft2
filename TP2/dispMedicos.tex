\subsection{Dispositivos biomédicos}
\subsubsection{Diagrama}
% \ig{l}{0.8}{}{}

\subsubsection{Detalle}

Este diagrama detalla la arquitectura del sistema de dispositivos médicos.

En este sistema tenemos por un lado el sistema de driver de dispositivo, que
toma una medición. Esta medición o señal es etiquetada (de manera de que
se identifique de que dispositivo es, datos sobre el usuario y un tag
de geolocalización en caso de emergencia) y luego enviada al router de
mediciones médicas que se ocupa de enviarlo al resto de los subsistemas que
necesiten hacer algo con el.

Estos subsistemas se dividen en dos. Por un lado tenemos el sistema de 
detección de mediciones peligrosas. Este sistema puede detectar que hay un
posible problema de salud a partir de la medición obtenida (o usar mediciones acumuladas, depende de la implementación particular que dejamos para 
posterior consideración). Debe entonces avisar al usuario de la gravedad del
problema notificandolo por su dispositivo y, si hay un riesgo de vida muy
serio, enviar un mensaje a emergencias con la posición que tenía la 
medición. Para poder garantizar que esto se produzca de la manera más rápida
posible, utilizamos redundancia activa para mantener dos sistemas que hacen
este análisis y posteriormente envian un mensaje a emergencias. Este mensaje
es enviado utilizando un sistema de envío confiable que considera la 
posibilidad de que la conexión con emergencias sufra fallas que impidan 
enviar bien el pedido si se hacen de manera incorrecta.

El problema de multiples pedidos se arregla utilizando para ello un 
\textit{timestamp} en los pedidos de manera que los sistemas instalados del
otro lado de emergencias puedan detectar los replays (porque además 
preferimos que lleguen dos pedidos de emergencias a que haya una mayor
posibilidad de que no llegue ninguno).

%% TODO: Explicar procesamiento de los datos
