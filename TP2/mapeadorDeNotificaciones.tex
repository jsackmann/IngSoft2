\subsection{Mapeador de notificaciones}
\subsubsection{Diagrama}

\ig{l}{0.9}{mapeadorDeNotificaciones.png}{Diagrama de componentes
y conectores del sistema de notificaciones.}

\subsubsection{Detalle}

En el anterior diagrama se incluye un detalle de los componentes y
conectores del mapeador de notificaciones de la aplicación. El rol
de este sistema es, dada la heterogeneidad de dispositivos a soportar,
queríamos separar la generación de eventos (o notificaciones como les
decimos en el trabajo) de la manera en la que estos se muestran. 
También era nuestro interés que los usuarios pudieran configurar la
manera en la que se les informaban las notificaciones, pero que el
sistema eligiera la mejor opción dadas las que hubiera (por ejemplo
usar visual si el usuario no manifiesta deseo por otra y si el 
dispositivo lo soporta).

Consideramos que las posibles salidas consisten en utilizar 
notificaciones auditivas, visuales, o enviar mediante un conector
externo la notificación a otro dispositivo (por ejemplo, usando
Bluetooth para enviar la notificación al teléfono celular del 
corredor).

El sistema de lógica de mapeo se encarga de, utilizando soporte de
una base de datos central para saber que características hay en
el dispositivo y como utilizarlas (por ello necesitamos un servidor
central de actualizaciones), y utilizando la configuración del usuario
para saber que organización dar a las notificaciones, va tomando 
estas notificaciones y creando los estímulos al dispositivo 
necesarios para mostrarla, teniendo en cuenta su nivel de prioridad 
y otros posibles factores.

Las notificaciones vienen del notificador priorizado (que se ocupa de
organizar las prioridades y otros factores, y se explica en un 
diagrama separado) y son etiquetadas con metadatos que permitan 
tomar acciones como descartarlas si hay demasiadas cosas que notificar
para el hardware, o dar datos como el tiempo en que se recibió la
notificación. 

Estas notificaciones son enviadas por el despachador, que bloquea
esperando a que el notificador determine que ya ha enviado la 
notificación (porque puede por ejemplo esperar a que la misma sea
mostrada durante un tiempo o hasta recibir una confirmación).
