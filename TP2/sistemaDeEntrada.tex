\subsection{Sistema de entrada de comandos}
\subsubsection{Diagrama}

\ig{p}{0.55}{sistemaDeEntrada.png}{Diagrama de componentes y conectores
del sistema de entrada al dispositivo.}

\subsubsection{Detalle}

En el diagrama anterior se incluye un diagrama de componentes y
conectores del sistema de entrada de un dispositivo, de manera de
que el sistema pueda recibir los comandos necesarios para trabajar.

Dada la heterogeneidad de dispositivos que se quieren soportar,
esta capa de sistema sirve para permitir separar la entrada de
comandos con los procesos que se generan a partir de ellos.

Por ejemplo, se soporta entrada por teclado, por voz, o por 
comandos externos (compatibles con una simple especificación de
comandos basada en texto, y que permiten dar ordenes al dispositivo
mediante un celular por ejemplo).

Los comandos enviados son procesados por un interprete. Este se 
ocupa de hacer procesamiento simbólico de manera de convertir el texto
del comando en un comando efectivo interpretable por el resto del
sistema. En el caso particular de instrucciones dadas por voz, se 
procesa esta entrada a texto utilizando para ello un servicio externo
basado en Google Speech (el mismo eventualmente puede ser reemplazado
por un servicio compatible, ya que esta en nuestros intereses eliminar
la dependencia tecnológica lo más posible).

Los comandos son comparados contra un repositorio de manera de 
realizar validación del mismo contra los comandos y sus distintas
reglas. Estas reglas son actualizadas periodicamente mediante un
sistema automático (de manera de involucrar lo menos posible 
en este procedimiento tedioso al usuario). Si hay un error, se 
muestra una notificación de mensaje de error en el dispositivo del
usuario (ya que nos interesa tratar de hacer al sistema lo más usable
posible, y para ello usamos la táctica de \textit{buenos mensajes de
error}). 

Si el comando es interpretado, el mismo es liberado a los distintos
otros sistemas. Dado que tenemos un conjunto de sistemas interesados
en ciertos tipos de mensajes (por ejemplo el modulo de seguimiento
quiere saber unicamente los que impliquen crear, iniciar o detener
un entrenamiento, etc.) y mensajes que publicar, utilizamos para esto
un bus de \textit{publish subscribe}. Si bien no hay más que un solo
publicador, existen una variedad no acotada (porque se pueden ir
agregando nuevos) mensajes, y nos interesa poder soportar eventualmente
nuevos comandos.

