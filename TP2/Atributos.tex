\subsection{Atributo de Usabilidad}

El sistema debe poder ser utilizado por personas con capacidades
diferentes de movibilidad, visión o audición de la manera más
satisfactoria posible.

\begin{itemize}
\itemsep1pt\parskip0pt\parsep0pt
\item
  Fuente: Usuario
\item
  Estímulo: El usuario desea utilizar su dispositivo habiendo indicado
  que desea usarlo en modo con refuerzo.
\item
  Artefacto: Sistema
\item
  Entorno: En condiciones normales.
\item
  Respuesta: El sistema adecua las notificaciones de acuerdo a las
  limitaciones del usuario.
\item
  Medida de respuesta: El 99\% de los usuarios puede entender las
  notificaciones recibidas en menos de 10 segundos sin ayuda de una
  persona externa.
\end{itemize}

\subsection{Atributo de Usabilidad}

Queremos que la aplicación sea intuitiva, elegante y fácil de usar, de
manera que el usuario no tenga problemas en encontrar como realizar una
tarea.

\begin{itemize}
\itemsep1pt\parskip0pt\parsep0pt
\item
  Fuente: Usuario nuevo
\item
  Estímulo: Empieza a utilizar el sistema en una carrera con su
  dispositivo.
\item
  Artefacto: Sistema
\item
  Entorno: En condiciones normales.
\item
  Respuesta: El sistema notifica al usuario las opciones y empieza a
  recibir instrucciones del mismo por los diversos medios que permita el
  dispositivo.
\item
  Medida de respuesta: El 70\% de los usuarios, sin ser avisado
  previamente, descubre las opciones de uso del sistema y como dar
  órdenes al mismo.
\end{itemize}

\subsection{Atributo de Usabilidad}

Las indicaciones dadas por el sistema, visuales auditivas o táctiles,
deben ser fáciles de entender por el usuario, y adaptarse al tipo de
dispositivo que se tiene.

\begin{itemize}
\itemsep1pt\parskip0pt\parsep0pt
\item
  Fuente: Usuario
\item
  Estímulo: El usuario realiza una acción que requiere una notificación
  por parte del Sistema.
\item
  Artefacto: Sistema
\item
  Entorno: En condiciones normales.
\item
  Respuesta: El sistema provee la notificación utilizando los medios
  disponibles por el dispositivo y priorizando medios más inmediatos
  sobre otros menos inmediatos (por ejemplo priorizando visual sobre
  táctil en la medida de lo posible).
\item
  Medida de respuesta: El 75\% de los usuarios entiende en menos de 10
  segundos, sin ayuda de terceros, la notificación presentada cuando la
  misma ocurre por primera vez.
\end{itemize}

\subsection{Atributo de Usabilidad}

Las instrucciones y mensajes de los entrenadores a los corredores
profesionales deben ser lo menos intrusivos posibles.

\begin{itemize}
\itemsep1pt\parskip0pt\parsep0pt
\item
  Fuente: Entrenador
\item
  Estímulo: Envía un mensaje a un corredor profesional que esta a su
  cargo.
\item
  Artefacto: Sistema
\item
  Entorno: Condiciones normales.
\item
  Respuesta: El sistema muestra el mensaje del entrenador al usuario.
\item
  Medida de respuesta: El 75\% de los usuarios puede entender la
  notificación en menos de 6 segundos, sin ser distraido de la tarea que
  se encuentra realizando al momento pero sin ser interrumpido por otras
  notificaciones de menor prioridad.
\end{itemize}

\subsection{Atributo de Usabilidad}

Se desea que el usuario pueda utilizar su dispositivo para dar avisos al
sistema mediante el uso de voz, o utilizando medios táctiles.

\begin{itemize}
\itemsep1pt\parskip0pt\parsep0pt
\item
  Fuente: Usuario
\item
  Estímulo: Desea dar una órden al dispositivo por medio de la voz o
  mediante uso táctil.
\item
  Artefacto: Sistema
\item
  Entorno: Condiciones normales.
\item
  Respuesta: Se procesa la órden del usuario y el sistema utiliza la
  información para dar notificaciones o para ejecutar comandos.
\item
  Medida de respuesta: El 90\% de las órdenes dadas por voz son
  interpretadas correctamente por el Sistema en menos de 3 reintentos.
  De las órdenes ingresadas, el 70\% en promedio son órdenes que el
  sistema puede procesar efectivamente.
\end{itemize}

\subsection{Atributo de Seguridad}

La información de salud sobre los corredores debe ser almacenada de
manera que solo personal autorizado pueda revisarla, y solo se puede
acceder a la información de la API que se haya autorizado previamente a
consumir.

\begin{itemize}
\itemsep1pt\parskip0pt\parsep0pt
\item
  Fuente: Agente no autorizado.
\item
  Estímulo: Intenta acceder a la base de datos de la Secretaría de
  Deportes.
\item
  Artefacto: Sistema
\item
  Entorno: En condiciones normales.
\item
  Respuesta: El atacante no puede acceder a datos privados de los
  corredores, los accesos no autorizados son auditados por el sistema.
\item
  Medida de respuesta: El 99.9999\% de los ataques son rechazados. Los
  ataques con éxito no pueden obtener información privada de los datos
  utilizando dispositivos con poder de cómputo actuales en menos de 1000
  años.
\end{itemize}

\subsection{Atributo de Seguridad}

Solo un entrenador puede enviar un mensaje prioritario a los corredores
profesionales que esta \emph{coucheando}.

\begin{itemize}
\itemsep1pt\parskip0pt\parsep0pt
\item
  Fuente: Agente no autorizado.
\item
  Estímulo: Intenta enviar un mensaje personalizado a un corredor que no
  lo ha autorizado.
\item
  Artefacto: Sistema
\item
  Entorno: En condiciones normales.
\item
  Respuesta: El comentario es descartado por el sistema sin dar aviso al
  usuario.
\item
  Medida de respuesta: La performance de sistema no se degrada, se
  rechaza el 99.9999\% de los mensajes no autorizados enviados, no se
  interrumpe al corredor mientras se realiza este filtrado.
\end{itemize}

\subsection{Atributo de Seguridad}

La información de datos físicos que midan los sensores de los
dispositivos no debe poder ser interceptada en su camino a la unidad de
procesamiento central.

\begin{itemize}
\itemsep1pt\parskip0pt\parsep0pt
\item
  Fuente: Agente no autorizado.
\item
  Estímulo: Intenta acceder a los datos enviados desde un dispositivo.
\item
  Artefacto: Sistema
\item
  Entorno: En condiciones normales.
\item
  Respuesta: La información que puede ser accedida por agentes externos
  en el camino pero solo los extremos de la comunicación pueden usarla
  para obtener los datos reales.
\item
  Medida de Respuesta: Un \emph{eavesdropper} no puede obtener los datos
  que tomó el dispositivo en menos de 1000 años utilizando equipos con
  poder de cómputo como los actualmente disponibles.
\end{itemize}

\subsection{Atributo de Performance}

El sistema de procesamiento debe procesar estadísticas de los
datos biomédicos de los corredores en tiempo real de manera que
estos esten disponibles para los consumidores de la API lo más
``frescos'' posible.

\begin{itemize}
\itemsep1pt\parskip0pt\parsep0pt
\item
	Fuente: Corredores del sistema
\item
	Estimulo: Ingresan datos médicos mediante medidores.
\item
	Artefacto: Sistema
\item
	Entorno: En condiciones normales
\item
	Respuesta: Los datos son procesados y se obtienen estadísticas de los mismos.
\item
	Medida de Respuesta: El tiempo de procesamiento desde la llegada del dato hasta su almacenamiento en un batch es de 1 segundo.
\end{itemize} 

\subsection{Atributo de Performance}

El sistema debe dar respuesta rápida ante situaciones de riesgo médico
detectadas mediante un dispositivo de medición de datos médicos.

\begin{itemize}
\itemsep1pt\parskip0pt\parsep0pt
\item
  Fuente: Dispositivo médico.
\item
  Estímulo: Se envía una señal que indica un riesgo potencial a la salud
  del corredor.
\item
  Artefacto: Sistema
\item
  Entorno: En condiciones normales.
\item
  Respuesta: Se le avisa al corredor de la situación, si la situación
  conlleva riesgo de vida se pide asistencia médica.
\item
  Medida de respuesta: El aviso de situación y las acciones correctivas
  se realizan en menos de 10 segundos de recibida la señal de riesgo
  potencial del medidor.
\end{itemize}

\subsection{Atributo de Disponibilidad}

El sistema debe lograr que un mensaje de pedido de emergencia llegue al
servicio de emergencias de manera que se disminuya en lo mínimo posible
el riesgo de vida del paciente.

\begin{itemize}
\itemsep1pt\parskip0pt\parsep0pt
\item
  Fuente: Dispositivo médico.
\item
  Estímulo: Se envía una señal que indica un riesgo potencial a la salud
  del corredor.
\item
  Artefacto: Sistema
\item
  Entorno: En condiciones normales.
\item
  Respuesta: Se le avisa al corredor de la situación, si la situación
  conlleva riesgo de vida se pide asistencia médica.
\item
  Medida de respuesta: El 99\% de las veces que se detecta una señal de
  riesgo de vida para el corredor, el pedido de ayuda llega a
  emergencias con datos que les permiten enviar una ambulancia al lugar
  del hecho.
\end{itemize}

\subsection{Atributo de Disponibilidad}

El sistema debe estar en funcionamiento todo el tiempo que sea posible.

\begin{itemize}
\itemsep1pt\parskip0pt\parsep0pt
\item
  Fuente: Usuarios
\item
  Estímulo: Se intenta enviar un pedido al sistema de información.
\item
  Artefacto: Sistema
\item
  Entorno: En condiciones normales.
\item
  Respuesta: El sistema responde con la información pertinente al
  usuario.
\item
  Medida de Respuesta: El sistema responde con una probabilidad que
  equivale a un \emph{downtime} de 30 segundos por día promediado a lo
  largo de un año.
\end{itemize}

\subsection{Atributo de Modificabilidad}

Se desea poder agregar nuevos modelos de dispositivos biométricos.

\begin{itemize}
\itemsep1pt\parskip0pt\parsep0pt
\item
  Fuente: Desarrollador
\item
  Estímulo: Se desea poder recolectar una nueva métrica de carácter
  médico a partir de un dispositivo.
\item
  Artefacto: Sistema
\item
  Entorno: En condiciones normales.
\item
  Respuesta: Las medidas del dispositivo pasan a ser recolectadas y
  enviadas al sistema, y los demás módulos continuan funcionando.
\item
  Medida de Respuesta: Desarrollar y probar el driver para recolectar
  esta información se realiza en menos de dos semanas hombre de trabajo.
\end{itemize}

\subsection{Atributo de Modificabilidad}

El sistema debe poder adaptarse a nuevo hardware de manera de poder
cambiar el hardware actual a otro para aumentar el poder de cómputo.

\begin{itemize}
\itemsep1pt\parskip0pt\parsep0pt
\item
  Fuente: Secretaría de Deportes
\item
  Estímulo: Se agrega nuevo hardware para procesamiento.
\item
  Artefacto: Sistema
\item
  Entorno: En condiciones normales.
\item
  Respuesta: El hardware agregado adicional recibe una proporción de los
  pedidos para procesar. Se sigue utilizando el resto del hardware con
  la proporción de pedidos modificada.
\item
  Medida de respuesta: El hardware agregado logra entrar en
  funcionamiento en menos de una semana de configuración y
  \emph{deployment}.
\end{itemize}

\subsection{Atributo de Modificabilidad}

El sistema debe permitir que se adapten todo tipo de dispositivos, no
solo anteojos sino también relojes de pulsera, mp3, etc.

\begin{itemize}
\itemsep1pt\parskip0pt\parsep0pt
\item
  Fuente: Desarrollador
\item
  Estímulo: Se desea que el sistema soporte el uso de un nuevo
  dispositivo.
\item
  Artefacto: Sistema
\item
  Entorno: En tiempo de ejecución.
\item
  Respuesta: Se agrega soporte para el dispositivo, no se modifica el
  resto del sistema.
\item
  Medida de respuesta: El cambio se desarrolla, testea y el dispositivo
  esta listo para uso con el sistema en menos de dos semanas hombre.
\end{itemize}
