\section{Atributos de calidad}

A continuación incluimos los atributos de calidad que identificamos en base al trabajo práctico y
al QAW (\textit{Quality Attribute Workshop}) realizado en clase. También incluimos una justificación de como
logramos los mismos en la arquitectura. Para ello justificamos las tácticas usadas mediante el libro de 
arquitectura de la materia~\cite{lenbass}.

\subsection{Atributo de Usabilidad}
El sistema debe poder ser utilizado por personas con capacidades diferentes de movibilidad, visión o audición de la manera más satisfactoria posible.

\begin{itemize}
  \item \textbf{Fuente}: Usuario
  \item \textbf{Estímulo}: El usuario desea utilizar su dispositivo habiendo indicado que desea usarlo en modo con refuerzo.
  \item \textbf{Artefacto}: Sistema
  \item \textbf{Entorno}: En condiciones normales.
  \item \textbf{Respuesta}: El sistema adecua las notificaciones de acuerdo a las limitaciones del usuario.
  \item \textbf{Medida} de respuesta: El 99\% de los usuarios puede entender las notificaciones recibidas en menos de 10 segundos sin ayuda de una persona externa.
\end{itemize}

\paragraph{Justificación en la arquitectura}
Este atributo se logra gracias a que el \texttt{sistema de configuración} envía datos al \texttt{mapper de notificaciones a estímulo}, indicándole, si hubiera, las restriciones de uso del usuario correspondiente. De esta forma, el mapper (utilizando el componente de \texttt{sistema de lógica de mapeo}) puede adaptar la notificación correspondiente a las discapacidades del usuario. Esto utiliza la técnica de \emph{separate user interface}


\subsection{Atributo de Usabilidad}
Las indicaciones dadas por el sistema, visuales auditivas o táctiles, deben ser fáciles de entender por el usuario, y adaptarse al tipo de dispositivo que se tiene.

\begin{itemize}
  \item \textbf{Fuente}: Usuario
  \item \textbf{Estímulo}: El usuario realiza una acción que requiere una notificación por parte del Sistema.
  \item \textbf{Artefacto}: Sistema
  \item \textbf{Entorno}: En condiciones normales.
  \item \textbf{Respuesta}: El sistema provee la notificación utilizando los medios disponibles por el dispositivo y priorizando medios más inmediatos sobre otros menos inmediatos (por ejemplo priorizando visual sobre táctil en la medida de lo posible).
  \item \textbf{Medida} de respuesta: El 75\% de los usuarios entiende en menos de 10 segundos, sin ayuda de terceros, la notificación presentada cuando la misma ocurre por primera vez.
\end{itemize}

\paragraph{Justificación en la arquitectura}
Al igual que el atributo anterior, este atributo se logra gracias al \texttt{mapper de notificaciones a estímulo}. El componente de \texttt{sistema de lógica de mapeo} puede determinar correctamente la forma en la que mostrar la notificación utilizando el \texttt{detector de características de dispositivo}. Este componente realiza numerosas \emph{system calls} al sisteam operativo del dispositivo en el que esté corriendo y puede informarle al \texttt{sistema de lógica de mapeo} cuáles son las \emph{features} del dispositivo y éste puede seleccionar la mejor forma de mostrar la notificación. 


\subsection{Atributo de Usabilidad}
Queremos que la aplicación sea intuitiva, elegante y fácil de usar, de manera que el usuario no tenga problemas en encontrar como realizar una tarea.

\begin{itemize}
  \item \textbf{Fuente}: Usuario nuevo
  \item \textbf{Estímulo}: Empieza a utilizar el sistema en una carrera con su dispositivo.
  \item \textbf{Artefacto}: Sistema
  \item \textbf{Entorno}: En condiciones normales.
  \item \textbf{Respuesta}: El sistema notifica al usuario las opciones y empieza a recibir instrucciones del mismo por los diversos medios que permita el dispositivo.
  \item \textbf{Medida} de respuesta: El 70\% de los usuarios, sin ser avisado previamente, descubre las opciones de uso del sistema y como dar órdenes al mismo.
\end{itemize}


\subsection{Atributo de Usabilidad}
Las instrucciones y mensajes de los entrenadores a los corredores profesionales deben ser lo menos intrusivos posibles, pero deben resaltar más que un comentario motivacional proveniente de una red social.

\begin{itemize}
  \item \textbf{Fuente}: Entrenador
  \item \textbf{Estímulo}: Envía un mensaje a un corredor profesional que esta a su cargo.
  \item \textbf{Artefacto}: Sistema
  \item \textbf{Entorno}: Condiciones normales.
  \item \textbf{Respuesta}: El sistema muestra el mensaje del entrenador al usuario.
  \item \textbf{Medida} de respuesta: El 75\% de los usuarios puede entender la notificación en menos de 6 segundos, sin ser distraido de la tarea que se encuentra realizando al momento pero sin ser interrumpido por otras notificaciones de menor prioridad.
\end{itemize}

\paragraph{Justificación en la arquitectura}
Este atributo se logra gracias a la interacción entre el \texttt{notificador con prioridad} y el \texttt{mapper de notificaciones a estímulo}. El primero le asigna una prioridad correspondiente a cada notificación que cada módulo quiere enviar al usuario. En el caso de las notificaciones de entrenador, se las considera de prioridad 1 (ver \ref{notificadorConPrioridad}). Al tener las notificaciones separadas por prioridad, el mapeador puede correctamente resaltar las más prioritarias, relegando (sea ignorando completamente o disminuyendo su importancia) las de menor prioridad.

También utiliza la técnica de \texttt{separate user interface}.


\subsection{Atributo de Usabilidad}
Se desea que el usuario pueda utilizar su dispositivo para dar avisos al sistema mediante el uso de voz, o utilizando medios táctiles.

\begin{itemize}
  \item \textbf{Fuente}: Usuario
  \item \textbf{Estímulo}: Desea dar una órden al dispositivo por medio de la voz o mediante uso táctil.
  \item \textbf{Artefacto}: Sistema
  \item \textbf{Entorno}: Condiciones normales.
  \item \textbf{Respuesta}: Se procesa la órden del usuario y el sistema utiliza la información para dar notificaciones o para ejecutar comandos.
  \item \textbf{Medida} de respuesta: El 90\% de las órdenes dadas por voz son interpretadas correctamente por el Sistema en menos de 3 reintentos. De las órdenes ingresadas, el 70\% en promedio son órdenes que el sistema puede procesar efectivamente.
\end{itemize}

\paragraph{Justificación en la arquitectura} %REVISAR ESTO
Este atributo de calidad se logra mediante la colaboración entre el \texttt{sistema de entrada del usuario} y el \texttt{sistema de instrucción de comandos}. Utilizando este primer componente, el sistema puede recibir \textit{input} del usuario; de acuerdo al tipo de dispositivo que esté utilizando el usuario, la forma para introducir \textit{input} será distinta (en un \emph{Google Glass} probablemente sea por voz, mientras que en un \emph{smartphone} puede ser tanto por voz como por pantalla táctil). En cualquiera de los casos, la interacción entre ambos componentes logra ocupa de convertir esa forma particular de \texttt{input} en comandos para el sistema.


\subsection{Atributo de Seguridad}
La información de salud sobre los corredores debe ser almacenada de manera que solo personal autorizado pueda revisarla, y solo se puede acceder a la información de la API que se haya autorizado previamente a consumir.

\begin{itemize}
  \item \textbf{Fuente}: Agente no autorizado.
  \item \textbf{Estímulo}: Intenta acceder a la base de datos de la Secretaría de Deportes.
  \item \textbf{Artefacto}: Sistema
  \item \textbf{Entorno}: En condiciones normales.
  \item \textbf{Respuesta}: El atacante no puede acceder a datos privados de los corredores, los accesos no autorizados son auditados por el sistema.
  \item \textbf{Medida} de respuesta: El 99.9999\% de los ataques son rechazados. Los ataques con éxito no pueden obtener información privada de los datos utilizando dispositivos con poder de cómputo actuales en menos de 1000 años.
\end{itemize}

\paragraph{Justificación en la arquitectura}
Para lograr este atributo se utilizan los conectores \texttt{conexión encriptada con ssl} (ver \ref{conexionEncriptadaConSSL}),  \texttt{sistema de interfaz a datos} y \texttt{cifrador homomórfico} (ver \ref{cifradorHomo}). El primero utiliza un sistema estándar de encriptación mediante \emph{SSL} para evitar que algún atacante realizando un ataque del tipo \emph{man in the middle} pueda interceptar información en tránsito. Además, como la conexión con \emph{SSL} se realiza sobre un canal \textbf{confiable} (el conector \texttt{módulo de transmisión confiable}), se garantiza que los datos no fueron almacenados en el camino. De esta forma, durante la transferencia de la información de salud de los coredores, los datos viajan seguros (suponiendo confianza en las autoridades certificantes). 

Una vez dentro de los repositorios, la seguridad se garantiza mediante el \texttt{sistema de interfaz a datos} (ver \ref{dispMedicos}), que requiere autenticación a la hora de acceder a los datos. Además, los datos son cifrados antes de ser almacenados mediante el \texttt{cifrador homomórfico}. De esta forma, por más que alguien tenga acceso directo a la base de datos (violentando por ejemplo la seguridad física de la Secretaría de Deportes), no podría obtener datos de la salud de los corredores.

Las tácticas utilizadas para esto son:
\begin{itemize}
  \item \emph{Authenticate users}
  \item \emph{Authorize users}
  \item \emph{Maintain data confidentiality}
  \item \emph{Maintain integrity}
\end{itemize}


\subsection{Atributo de Seguridad}
Solo un entrenador puede enviar un mensaje prioritario a los corredores profesionales que esta \emph{coucheando}.

\begin{itemize}
  \item \textbf{Fuente}: Agente no autorizado.
  \item \textbf{Estímulo}: Intenta enviar un mensaje personalizado a un corredor que no lo ha autorizado.
  \item \textbf{Artefacto}: Sistema
  \item \textbf{Entorno}: En condiciones normales.
  \item \textbf{Respuesta}: El comentario es descartado por el sistema sin dar aviso al usuario.
  \item \textbf{Medida} de respuesta: La performance de sistema no se degrada, se rechaza el 99.9999\% de los mensajes no autorizados enviados, no se interrumpe al corredor mientras se realiza este filtrado.
\end{itemize}

\paragraph{Justificación en la arquitectura}
Este atributo se logra utilizando el módulo de \texttt{notificador con prioridad} (ver \ref{notificadorConPrioridad}) para garantizar que las notificaciones se ordenen correspondientemente (por ejemplo, evita que un corredor deje de recibir notificaciones de su entrenador por un \emph{flooding} de comentarios motivacionales). Además, participan los módulos \texttt{conexión encriptada por SSL} (ver \ref{conexionEncriptadaConSSL}) y los sistemas de seguridad de entrenador (módulos \texttt{sistema de inscripción de entrenadores}, \texttt{chequeo de permisos} y \texttt{sistema de autenticación de entrenador}). Estos últimos módulos garantizan que nadie pueda hacerse pasar por un entrenador.


\subsection{Atributo de Seguridad}
La información de datos físicos que midan los sensores de los dispositivos no debe poder ser interceptada en su camino a la unidad de procesamiento central.

\begin{itemize}
  \item \textbf{Fuente}: Agente no autorizado.
  \item \textbf{Estímulo}: Intenta acceder a los datos enviados desde un dispositivo.
  \item \textbf{Artefacto}: Sistema
  \item \textbf{Entorno}: En condiciones normales.
  \item \textbf{Respuesta}: La información que puede ser accedida por agentes externos en el camino pero solo los extremos de la comunicación pueden usarla para obtener los datos reales.
  \item \textbf{Medida} de Respuesta: Un \emph{eavesdropper} no puede obtener los datos que tomó el dispositivo en menos de 1000 años utilizando equipos con poder de cómputo como los actualmente disponibles.
\end{itemize}

\paragraph{Justificación en la arquitectura}
Análogo al atributo anterior, los componentes involucrados en este caso son el módulo de \texttt{conexión encriptada por SSL} (ver \ref{conexionEncriptadaConSSL}) y el \texttt{cifrador homomórfico} (ver \ref{cifradorHomo}). Ambos utilizan algoritmos de cifrado (simétrico y/u homomórfico respectivamente) suficientemente seguros como para salvaguardar la confidencialidad de los datos.

La táctica utilizada para esto es la de \emph{maintain data confidenciality}.


\subsection{Atributo de Performance}
El sistema debe dar respuesta rápida ante situaciones de riesgo médico detectadas mediante un dispositivo de medición de datos médicos.

\begin{itemize}
  \item \textbf{Fuente}: Dispositivo médico.
  \item \textbf{Estímulo}: Se envía una señal que indica un riesgo potencial a la salud del corredor.
  \item \textbf{Artefacto}: Sistema
  \item \textbf{Entorno}: En condiciones normales.
  \item \textbf{Respuesta}: Se le avisa al corredor de la situación, si la situación conlleva riesgo de vida se pide asistencia médica.
  \item \textbf{Medida} de respuesta: El aviso de situación y las acciones correctivas se realizan en menos de 10 segundos de recibida la señal de riesgo potencial del medidor.
\end{itemize}

\paragraph{Justificación en la arquitectura}
Para lograr este atributo de performance se realizan varias tácticas que involucran a numerosos componentes: 
\begin{itemize}
  \item \texttt{Notificador con prioridad} (ver~\ref{conexionEncriptadaConSSL}).
  \item \texttt{Sistema de datos biomédicos}~\ref{dispMedicos}. Particularmente: 
  \begin{itemize}
    \item \texttt{Etiquetador de señal}.
    \item \texttt{Router de mediciones biomédicas}.
    \item \texttt{Rastreador de señales de riesgo}.
    \item \texttt{Sistema de determinación de acción}.
    \item \texttt{Módulo de transmisión confiable}.
  \end{itemize}
\end{itemize}

Dado que las notificaciones del dispositivo biomédico tienen la mayor de las prioridades (\emph{P0}), son \emph{handleadas} casi instantáneamente una vez que son recibidas por el \texttt{notificador con prioridad}. 

Las principal táctica utilizada es \emph{introduce concurrency}. 

\subsection{Atributo de Disponibilidad}

El sistema debe lograr que un mensaje de pedido de emergencia llegue al servicio de emergencias de manera que se disminuya en lo mínimo posible el riesgo de vida del paciente.

\begin{itemize}
  \item \textbf{Fuente}: Dispositivo médico.
  \item \textbf{Estímulo}: Se envía una señal que indica un riesgo potencial a la salud del corredor.
  \item \textbf{Artefacto}: Sistema
  \item \textbf{Entorno}: En condiciones normales.
  \item \textbf{Respuesta}: Se le avisa al corredor de la situación, si la situación conlleva riesgo de vida se pide asistencia médica.
  \item \textbf{Medida} de respuesta: El 99\% de las veces que se detecta una señal de riesgo de vida para el corredor, el pedido de ayuda llega a emergencias con datos que les permiten enviar una ambulancia al lugar del hecho.
\end{itemize}

\paragraph{Justificación en la arquitectura}

La principal táctica utilizada para esto es implementar transmisión confiable de manera que se reintente enviar
el mensaje tantas veces como sea necesario para que llegue. Esto lo podemos ver como un tipo de \textit{state
resincronization}. 

\subsection{Atributo de Disponibilidad}
El sistema debe estar en funcionamiento todo el tiempo que sea posible.

\begin{itemize}
  \item \textbf{Fuente}: Usuarios
  \item \textbf{Estímulo}: Se intenta enviar un pedido al sistema de información.
  \item \textbf{Artefacto}: Sistema
  \item \textbf{Entorno}: En condiciones normales.
  \item \textbf{Respuesta}: El sistema responde con la información pertinente al usuario.
  \item \textbf{Medida} de Respuesta: El sistema responde con una probabilidad que equivale a un \emph{downtime} de 30 segundos por día promediado a lo largo de un año.
\end{itemize}

\paragraph{Justificación en la arquitectura}

Para implementar esto utilizamos \texttt{shadow} en varias partes del trabajo práctico de manera que el sistema
parezca que esta andando (es decir, de respuesta al usuario) lo mayor posible. También utilizamos estrategias
de \textit{ping-echo} para salud del sistema de procesamiento de datos y utilizamos caches de datos para no 
necesitar acceder constantemente al servidor. También utilizamos control de frecuencia de sampleo para manejar
la lectura de datos de cinemática por parte del sistema y usamos colas en los servicios de salida para evitar
sobrecargar (porque usamos tranmisión con confirmación) a los servicios del sistema.

\subsection{Atributo de Modificabilidad}
Se desea poder agregar nuevos modelos de dispositivos biométricos.

\begin{itemize}
  \item \textbf{Fuente}: Desarrollador
  \item \textbf{Estímulo}: Se desea poder recolectar una nueva métrica de carácter médico a partir de un dispositivo.
  \item \textbf{Artefacto}: Sistema
  \item \textbf{Entorno}: En condiciones normales.
  \item \textbf{Respuesta}: Las medidas del dispositivo pasan a ser recolectadas y enviadas al sistema, y los demás módulos continuan funcionando.
  \item \textbf{Medida} de Respuesta: Desarrollar y probar el driver para recolectar esta información se realiza en menos de dos semanas hombre de trabajo.
\end{itemize}

\paragraph{Justificación en la arquitectura}

Para esto utilizamos un sistema de fronteo de servicios médicos que se ocupa de routear los mensajes a los
distintos sistemas (en particular el \texttt{Rastreador de señales de riesgo} y el sistema de envío de datos
médicos a procesar) y colecciona de los diversos dispositivos los datos que necesita. Esto separa la interfaz
de procesamiento de los datos de la generación de los mismos y facilita el trabajo de escribir drivers para los
dispositivos de manera que solo necesitemos formatear las mediciones a lo que necesita el \texttt{Router de
mediciones biomédicas}

\subsection{Atributo de Modificabilidad}
El sistema debe poder adaptarse a nuevo hardware de manera de poder cambiar el hardware actual a otro para aumentar el poder de cómputo.

\begin{itemize}
  \item \textbf{Fuente}: Secretaría de Deportes
  \item \textbf{Estímulo}: Se agrega nuevo hardware para procesamiento.
  \item \textbf{Artefacto}: Sistema
  \item \textbf{Entorno}: En condiciones normales.
  \item \textbf{Respuesta}: El hardware agregado adicional recibe una proporción de los pedidos para procesar. Se sigue utilizando el resto del hardware con la proporción de pedidos modificada.
  \item \textbf{Medida} de respuesta: El hardware agregado logra entrar en funcionamiento en menos de una semana de configuración y \emph{deployment}.
\end{itemize}

\paragraph{Justificación en la arquitectura}

Esto se logra instalando el \texttt{Balanceador de carga} que se encarga de mantener una flota de unidades de
procesamiento funcionando y asignarles trabajo, y luego un sistema de \texttt{Colector de procesamiento} que
junta los datos de estos. Por lo tanto para agregar hardware, solo es necesario conectarlo y agregar su 
existencia en los dos sistemas mencionados, lo cual permite escalar horizontalmente con un esfuerzo de 
configuración lo más mínimo posible.

\subsection{Atributo de Modificabilidad}
El sistema debe permitir que se adapten todo tipo de dispositivos, no solo anteojos sino también relojes de pulsera, mp3, etc.

\begin{itemize}
  \item \textbf{Fuente}: Desarrollador
  \item \textbf{Estímulo}: Se desea que el sistema soporte el uso de un nuevo dispositivo.
  \item \textbf{Artefacto}: Sistema
  \item \textbf{Entorno}: En tiempo de ejecución.
  \item \textbf{Respuesta}: Se agrega soporte para el dispositivo, no se modifica el resto del sistema.
  \item \textbf{Medida} de respuesta: El cambio se desarrolla, testea y el dispositivo esta listo para uso con el sistema en menos de dos semanas hombre.
\end{itemize}

\paragraph{Justificación en la arquitectura}

Esto se logra mediante dos objetivos:

\begin{itemize}
	\item 
		Separamos las entradas de dispositivos del sistema de interpretación de comandos, de manera que agregar
		un nuevo dispositivo solo requiere escribir el código para tomar su entrada y pasarla a un texto que
		pueda usar el interprete de comandos (y luego estos comandos son usados por los demás sistemas).
	\item 
		Separamos las funcionalidades de presentación (que denominamos estímulos en el trabajo) de las 
		notificaciones a mostrar y sus prioridades. De esta manera solo es 
\end{itemize}

De esta manera \textit{anticipamos cambios esperados}, \textit{abstraemos servicios comunes} y utilizamos
\textit{adhesion a protocolos determinados} para separar los detalles particulares del dispositivo lo mejor
posible.
